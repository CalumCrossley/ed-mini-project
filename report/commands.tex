\usepackage{tcolorbox}
\tcbuselibrary{theorems}
\tcbuselibrary{breakable}

\usepackage{microtype}

\usepackage{quiver}

\usepackage{pgfplots}
\pgfplotsset{width=\columnwidth,compat=1.13}

\usepackage{listings}
\definecolor{codegreen}{rgb}{0,0.6,0}
\definecolor{codegray}{rgb}{0.5,0.5,0.5}
\definecolor{codepurple}{rgb}{0.58,0,0.82}
\definecolor{backcolour}{rgb}{0.95,0.95,0.92}
\lstdefinestyle{mystyle}{
    backgroundcolor=\color{backcolour},   
    commentstyle=\color{codegreen},
    keywordstyle=\color{magenta},
    numberstyle=\tiny\color{codegray},
    stringstyle=\color{codepurple},
    basicstyle=\ttfamily\footnotesize,
    breakatwhitespace=false,         
    breaklines=true,                 
    captionpos=b,                    
    keepspaces=true,                 
    numbers=none,                    
    numbersep=5pt,                  
    showspaces=false,                
    showstringspaces=false,
    showtabs=false,                  
    tabsize=2
}
\lstset{style=mystyle}

\usepackage{mathtools}
\usepackage{amsthm}
\usepackage{amscd}
\usepackage{tikz-cd}

\usepackage{float}
\usepackage{titling}
\usepackage{caption}
\usepackage{subcaption}

\usepackage{hyperref}
\definecolor{maincolor}{RGB}{122, 17, 49}
\hypersetup{linkcolor=maincolor, linkbordercolor=maincolor, citebordercolor=maincolor} %colorlink removes boxes

\setcounter{section}{-1}
\setcounter{secnumdepth}{4} % how many sectioning levels to assign numbers to
\setcounter{tocdepth}{4}    % how many sectioning levels to show in ToC

\numberwithin{equation}{section}

\newtcbtheorem[number within=section]{theorem}{Theorem} {theorem style = plain, colback=maincolor!30!white, coltitle=black, colframe=white, fonttitle = \upshape\bfseries, fontupper=\itshape}{th}

\newtcbtheorem[use counter from = theorem]{definition}{Definition} {theorem style = plain, colback=maincolor!30!white, coltitle=black, colframe=white, fonttitle = \upshape\bfseries,  fontupper=\itshape}{dfn}

\newtcbtheorem[use counter from = theorem]{proposition}{Proposition} {theorem style = plain, colback=maincolor!30!white, coltitle=black, colframe=white, fonttitle = \upshape\bfseries,  fontupper=\itshape}{prop}

\newtcbtheorem[use counter from = theorem]{corollary}{Corollary} {theorem style = plain, colback=maincolor!30!white, coltitle=black, colframe=white, fonttitle = \upshape\bfseries,  fontupper=\itshape}{cor}

\newtcbtheorem[use counter from = theorem]{lemma}{Lemma} {theorem style = plain, colback=maincolor!30!white, coltitle=black, colframe=white, fonttitle = \upshape\bfseries,  fontupper=\itshape}{lemma}

\newtcbtheorem[no counter]{ob}{Observation} {theorem style = plain, colback=maincolor!30!white, coltitle=black, colframe=white, fonttitle = \upshape\bfseries,  fontupper=\itshape}{ob}

\newtcbtheorem[no counter]{conjecture}{Conjecture} {theorem style = plain, colback=maincolor!30!white, coltitle=black, colframe=white, fonttitle = \upshape\bfseries,  fontupper=\itshape}{conj}

\newtcbtheorem[use counter from = theorem]{example}{Example} {breakable, theorem style = plain, colback=maincolor!10!white, coltitle=black, colframe=white, fonttitle = \upshape\itshape}{ex}

\newtcbtheorem[no counter]{technicalities}{Technicalities} {theorem style = plain, colback=maincolor!10!white, coltitle=black, colframe=white, fonttitle = \upshape\itshape}{tech}

\newtcbtheorem[use counter from = theorem]{remark}{Remark} {theorem style = plain, colback=maincolor!10!white, coltitle=black, colframe=white, fonttitle = \upshape\itshape}{rem}

\DeclareMathOperator{\adj}{adj} % adjugate matrix
\DeclareMathOperator{\im}{im}
\DeclareMathOperator{\coker}{coker}
\DeclareMathOperator{\cone}{cone}
\DeclareMathOperator{\Tot}{Tot}
\DeclareMathOperator{\Hom}{Hom}
\DeclareMathOperator{\Ext}{Ext}
\DeclareMathOperator{\lExt}{\mathcal{E}\mathit{xt}} % local Ext
\DeclareMathOperator{\lHom}{\mathcal{H}\mathit{om}} % local Hom
\newcommand{\dL}{\mathbf{L}} % derived left
\newcommand{\dR}{\mathbf{R}} % derived right
\newcommand{\mL}{\mathbb{L}} % left mutation
\newcommand{\mR}{\mathbb{R}} % right mutation
\DeclareMathOperator{\RHom}{\dR Hom} % derived usual Hom
\DeclareMathOperator{\RlHom}{\dR\mathcal{H}\mathit{om}} % derived local Hom
\DeclareMathOperator{\RG}{\dR\Gamma} % derived global sections
\newcommand{\Lotimes}{\otimes^\dL} % derived tensor product
\newcommand{\lperp}[1]{\prescript{\perp}{}{#1}} % left orthogonal
\DeclareMathOperator{\Spec}{Spec}
\DeclareMathOperator{\Proj}{Proj}
\DeclareMathOperator{\Bl}{Bl}
\DeclareMathOperator{\Ch}{Ch}
\DeclareMathOperator{\Pic}{Pic}
\DeclareMathOperator{\Coh}{Coh}
\DeclareMathOperator{\QCoh}{QCoh}
\DeclareMathOperator{\Perf}{Perf}
\DeclareMathOperator{\DPerf}{\mathfrak{Perf}}
\DeclareMathOperator{\MF}{MF}
\newcommand{\Ab}{\mathrm{Ab}}

% letters
\newcommand{\N}{\mathbb{N}}
\newcommand{\Z}{\mathbb{Z}}
\newcommand{\Q}{\mathbb{Q}}
\newcommand{\R}{\mathbb{R}}
\newcommand{\C}{\mathbb{C}}
\newcommand{\A}{\mathbb{A}}
\renewcommand{\P}{\mathbb{P}}
\renewcommand{\O}{\mathcal{O}}
\newcommand{\bbH}{\mathbb{H}}
\newcommand{\calH}{\mathcal{H}}
\newcommand{\calA}{\mathcal{A}}
\newcommand{\calB}{\mathcal{B}}
\newcommand{\calC}{\mathcal{C}}
\newcommand{\calD}{\mathcal{D}}
\newcommand{\calE}{\mathcal{E}}
\newcommand{\calF}{\mathcal{F}}
\newcommand{\calG}{\mathcal{G}}
\newcommand{\calL}{\mathcal{L}}
\newcommand{\calP}{\mathcal{P}}
\newcommand{\calS}{\mathcal{S}}
\newcommand{\calI}{\mathcal{I}}
\newcommand{\calN}{\mathcal{N}}
\newcommand{\calT}{\mathcal{T}}
\newcommand{\calR}{\mathcal{R}}
\newcommand{\bfF}{\mathbf{F}}
\newcommand{\bfL}{\mathbf{L}}
\newcommand{\tbfF}{\widetilde{\mathbf{F}}}
\newcommand{\tbfL}{\widetilde{\mathbf{L}}}
\newcommand{\scrF}{\mathscr{F}}
\newcommand{\tscrF}{\widetilde{\mathscr{F}}}
\newcommand{\scrL}{\mathscr{L}}
\newcommand{\tscrL}{\widetilde{\mathscr{L}}}
\newcommand{\scrP}{\mathscr{P}}
\newcommand{\tscrP}{\widetilde{\mathscr{P}}}


% names
\newcommand{\id}{\mathrm{id}} % identity
\newcommand{\pt}{\mathrm{pt}} % point
\newcommand{\ev}{\mathrm{ev}} % evaluation
\newcommand{\qi}{\mathrm{q.i.}} % quasiisomorphism
\newcommand{\Sg}{\mathrm{Sg}} % singularities

\newcommand{\todo}[1]{{\color{red}\textbf{TODO!: #1}}}
