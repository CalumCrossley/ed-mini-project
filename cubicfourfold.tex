In this section, we will review the deep link between K3 surfaces and cubic fourfolds. 

 A theorem of Hasset describes the so called \emph{special cubic fourfolds}, which are cubic fourfolds containing a surface $T$ which is not a complete intersection. They form a countable union of divisors $\calC_d$ in the moduli space $\P H^0(\O(3))/\P GL(6)$ of cubic fourfolds which are expected to be the rational cubic fourfolds (the intuition is that for a cubic fourfold to be rational, one has to blow up a surface at some point) and have associated K3 surfaces: the primitive cohomology of a K3 surface embeds as the complement $\langle h^2, T\rangle^\perp$. 

Alternatively, Kuznetsov has considered the derived category of a cubic fourfold and found a component $\calA_X$. He showed that Pfaffian cubics and nodal cubics have $\calA_X$ derived equivalent to K3 surfaces and conjectured \cite{KuznetsovDerivedCubic} that $X$ is rational if and only if $\calA_X \simeq \calD(S)$ for some K3 surface $S$.

The two viewpoints, the Hodge theoretic of Hassett and derived categorical of Kuznetsov, were shown to be equivalent by \cite{addington_hodge_2014}: in other words, ...

In this section, we will briefly describe the Hodge theory and Torelli theorems for K3 surfaces and cubic fourfolds. After that, we will show that the Kuznetsov component is a Calabi-Yau 2 category and show that the cubic fourfolds containing a plane $\calC_8$ are derived equivalent to a twisted K3 surface, which is an honest K3 surface when $X\in \calC_8 \cap \calC_d$ for some other $d$ on Hasset's list. Finally, we quickly sketch Addington-Thomas' paper.



\subsection{A quick review of K3 surfaces}
\begin{definition}{K3 surface}{K3 surface}
A K3 surface is a compact complex surface with $\omega_{S}$ trivial and $H^1(S,\mathcal{O}_{S})=0$.
\end{definition}

Firstly, spaces derived equivalent to K3 surfaces are also K3 surfaces: their dimensions have to be the same because the Serre functors are isomorphic and hence the orders of the canonical bundles are the same. Similarly, by passage to cohomology, we know that $$\bigoplus_{p-q=i}H^{p,q}(S)\simeq \bigoplus_{p-q=i}H^{p,q}(S')$$
Hence, since $h^{0,1}=h^{1,2}$ by Hodge symmetry and Serre duality, we see that both are zero.

Computations using Riemann-Roch show that $$2=\chi(S,\O_S)=\frac{c_1^2+c_2}{12}\implies c_2(S)=e(S)=24$$
Then Poincare duality shows $b_2=22$. 

The Hodge decomposition tells us that $$H^2(S,\mathbb{C})=H^{2,0}\oplus H^{1,1}\oplus H^{0,2}$$But both $h^{2,0}=h^{0,2}=1$ and hence $h^{1,1}=20$.  The Hodge diamond is thus: \[\begin{tikzcd}[row sep=tiny, column sep=tiny]
    &   & 1  &   &   \\
    & 0 &    & 0 &   \\
  1 &   & 20 &   & 1 \\
    & 0 &    & 0 &   \\
    &   & 1  &   &  
  \end{tikzcd}\]

Moreover, the intersection lattice is $H^2(X,\mathbb{X})=\Lambda=(-E_8)^{\oplus 2}\oplus U^{\oplus 3}$ and $H^2(S,\mathbb{Z})=(1,20,1)$ has signature $(3,19)$.

The Hodge structure on a K3 surface is then essentially determined by $H^{2,0}(S)\subset H^2(S;\mathbb{C})=\Lambda\otimes \mathbb{C}$. Thus, we can consider the period map $$P: \mathcal{M} \rightarrow \P \Lambda \otimes \mathbb{C}$$that sends a K3 surface $S$ to the line $H^{2,0}(S)$.  In fact, a nonzero vector in $H^{2,0}$ has $u^2=0, u \overline{u}>0$, which is the defining equation for the period domain. The local Torelli theorem shows that this is surjective and a local homeomorphism. It is not a global isomorphism: one has to mod out by the automorphisms of the lattice $\Lambda$. In fact, it is known that K3 surfaces are determined by the period map, i.e. by their Hodge structure:

\begin{theorem}{Global Torelli}{Global Torelli}
Two K3 surfaces are isomorphic if and only if there is a Hodge isometry $H^2(X,\mathbb{C})\simeq H^2(Y,\mathbb{C})$.
\end{theorem}

%A Hodge isometry respects the intersection pairing and the Hodge decomposition. 

%Because the canonical bundle of $\mathbb{P}^1$ is $\mathcal{O}(-2)$ and is also isomorphic to the normal bundle by adjunction, we see that $$C\cdot C=\int _{C}c_{1}(\mathcal{O}(-2)) =-2$$
%Such curves define a reflection on the cohomology: $$\begin{gathered}
 
%s_{\delta}:H^2(X,\mathbb{Z})\xrightarrow{}H^2(X,\mathbb{Z})\\
%\alpha\mapsto \alpha + \langle \alpha ,\delta \rangle \delta
%\end{gathered}$$
%Is this induced by a spherical twist? Yes, I think the one given by $\mathcal{O}_{C}$. Let's see why this is true:

%Firstly, $\mathcal{O}_{C}$ is spherical because $$\mathrm{Ext^1}(\mathcal{O}_{C},\mathcal{O}_{C})=H^0(\mathcal{Ext}^1)=H^0(\mathcal{N}_{C/X})=H^0(\mathcal{O}(-2))=0$$The other Ext groups are one-dimensional, by Serre duality. 


%Now, the Todd class of a K3 surface $X$ is given as $(1,0,2)$ so its square root is $(1,0,1)$. Hence, for a vector bundle, we see that $$v(\mathcal{E})=(\mathrm{rk}\,\mathcal{E}, c_{1}\mathcal{E}, \mathrm{rk}(\mathcal{E})+c_{1}(\mathcal{E})^2/2-c_{2}(\mathcal{E}))$$
%So for a rational sphere given by the zero locus of a line bundle $\mathcal{L}$, we see by the resolution $$\mathcal{L}^\lor\xrightarrow{}\mathcal{O}_{X}\xrightarrow{}\mathcal{O}_{C}$$ that $\mathrm{ch}(\mathcal{O}_{C})=\mathrm{ch}(\mathcal{L}^\lor)-\mathrm{ch(\mathcal{O}_{X})}=-c_{1}(\mathcal{L})$. Moreover, $v(\mathcal{L}^\lor)=(1,-\delta,0), v(\mathcal{O}_{X})=(1,0,0)$ and hence $$v(\mathcal{O}_{C})=(0,\delta,0)$$ and this recovers the reflection on $H^2(X,\mathbb{Z})$. 

%Wait it's apparently $\mathcal{O}_{C}(-1)$?

%Next: derived equivalence of K3 surfaces iff Hodge isometry of the Mukai Hodge structure.

%Also: moduli of sheaves

\subsection{Hodge theory of cubic fourfolds}

Let us first think about the Hodge diamond of $X$. Embed $\mathbb{P}^5$ via the Veronese embedding in $\mathbb{P}V_{3,5+1}$, the space of degree 3 polynomials in 6 variables. Then, a hyperplane section of the image of $\mathbb{P}^5$ consists of a linear relation between the basis of monomials of this vector space, and hence is precisely a cubic fourfold $X$. The Lefschetz hyperplane theorem states that then $$H^\bullet(\mathbb{P}^5)\xrightarrow{}H^\bullet (X) \text{ iso for }\bullet<4$$
We thus see, by Poincare duality, that $b_{1}=b_{3}=b_{5}=b_{7}=0$. Similarly, $b_{2}=b_{6}=1$ and must be given by $h^{1,1}$ by Hodge theory. We have that $H^2(X;\mathbb{Z})=\mathbb{Z}$ and the LES from the exponential sequence tells us that $$H^1(X,\mathcal{O}_{X})\xrightarrow{}H^1(X,\mathcal{O}_{X}^*)\xrightarrow{}H^2(X,\mathbb{Z})\xrightarrow{}H^2(X,\mathcal{O}_{X})$$
But the first term is 0, since it is a summand of $H^1(X;\mathbb{C})=0$. The second term is $\mathrm{Pic}(X)$ and the map is $c_1$ which lands in $H^{1,1}$. Since $c_1(K_X)=-3$ it is nonzero, so since $\mathbb{C}=H^2(X;\mathbb{C})=H^{2,0}\oplus H^{1,1}\oplus H^{0,2}$ the only option is for $h^{2,0}=h^{0,2}=0$ and $\mathrm{Pic}(X)=\mathbb{Z}$ generated by a hyperplane. 

Now, the interesting bit is the fourth row, the middle dimensional cohomology. By using the normal bundle sequence, we have that $$c(T_{X})(1+3x)=(1+x)^6=\iota^* c(T_{\mathbb{P}^5})$$We compute $c_1=3x, c_2=6x^2, c_3=2x^3, c_4=9x^4$. Then by Gauss-Bonnet, $$b_{4}+4=\chi(X)=\langle 9x^4, [X] \rangle= 27 $$since $pd(X)=3\omega$. We conclude that $b_4=23$. 

We also have that $H^0(X, \Omega^4)=H^0(X, \mathcal{O}_{X}(-3))=0$ so $h^{4,0}=h^{0,4}=0$. The only bit left is to determine $h^{1,3}=h^{3,1}$. 

Now we can use Hirzebruch-Riemann-Roch for the cotangent bundle of $X$, by computing $$\chi(\Omega_{X})=-h^{1,1}-h^{1,3}=\int _{X}\mathrm{ch(\Omega_{X})}\mathrm{td}(T_{X})=3\int _{\mathbb{P}^5} \omega \,\mathrm{ch}(\Omega_{\mathbb{P}^5}-\mathcal{O}(-3)) \mathrm{td}(T_{\mathbb{P}^5}-\mathcal{O}(3))  $$
Because I am lazy, I used the Macaulay 2 code: 

\begin{lstlisting}[language=Python]
loadPackage "Schubert2"
P5  = flagBundle({1,5}, VariableNames=>{s,q})
T = tangentBundle(P5)
coT = cotangentBundle(P5)
O1 = dual(P5.Bundles#0)
w = chern(1,O1)
NX = O1^**3
coNX = dual(NX)
TX = T - NX
coTX = coT - coNX
Q = 3 *  w*ch(coTX)*todd(TX)
print  integral  Q
\end{lstlisting}
This gives the answer -2, which implies $h^{1,3}=1$.

Can try to compute $h^{2,2}$ similarly: again by Hirzebruch-Riemann-Roch
$$h^{2,2}=\chi(\Omega_{X}^2)=\int_{X} \mathrm{ch}(\Omega^2_{X})\mathrm{td}(TX)  $$
The Macaulay code is: 

\begin{lstlisting}[language=Python] 
coT2X = exteriorPower_2 coTX
A = 3*w*ch(coT2X)* todd(TX)
print integral A
\end{lstlisting}
%can add caption by declaring caption = ...
This gives the correct answer $21$! All in all, the Hodge diamond looks like: 

\begin{center}
\begin{tikzcd}[row sep=tiny, column sep=tiny]
    &   &   &   & 1  &   &   &   &   \\
    &   &   & 0 &    & 0 &   &   &   \\
    &   & 0 &   & 1  &   & 0 &   &   \\
    & 0 &   & 0 &    & 0 &   & 0 &   \\
  0 &   & 1 &   & 21 &   & 1 &   & 0 \\
    & 0 &   & 0 &    & 0 &   & 0 &   \\
    &   & 0 &   & 1  &   & 0 &   &   \\
    &   &   & 0 &    & 0 &   &   &   \\
    &   &   &   & 1  &   &   &   &  
  \end{tikzcd}
\end{center}

The primitive cohomology of the cubic fourfold looks like $(0,1,20,1,0)$ and has signature $(20,2)$ which looks similar to the K3 $H^2(S,\mathbb{Z})$, but that one has signature $(3,19)$. 

We can see that $H^{3,1}\simeq \mathbb{C}$ and similarly to the K3 case we have a period map which to each cubic fourfolds associates a line in the local period domain, which is one of the connected components of the quadric hypersurface $Q\subset \P \Lambda$ defined by $u^2=0$ together with the condition that $-(u, \overline{v})$ is a positive hermitian form. We factor out by automorphisms $\Gamma^+$ of $H^4(X,\mathbb{Z})$ which preserve $(-,-)$ and the hyperplane squared $h^2$, and moreover stabilize $\tilde{\calD}$. The resulting map is $$\mathcal{M}_{\text{cubic fourfolds}}\rightarrow \calD : = \tilde{\calD}/\Gamma^+$$

Voisin \cite{voisin_theoreme_2008} proved the following theorem, by considering the cubic fourfolds containing a plane:

\begin{theorem}{Global Torelli for cubic fourfolds}{Global Torelli for cubic fourfolds}
    The period map for cubic fourfolds is injective.
\end{theorem}

We finish off the Hodge theoretic section by a review of the results of Hassett's paper \cite{hassett_special_2000}. 

The essential idea is as follows: a generic cubic fourfold $X$ has $H^{2,2}_{prim}(X;\mathbb{Z})=0$, but there is a class of \emph{special cubic fourfolds} which contain some integral class $T$. This allows us to pass to a codimension $1$ subspace of $H^{2,2}(X)$ which has a chance of being Hodge isometric to the primitive cohomology of a K3 surface (recall that the issue was that a priori they have different signatures). Explicitly, we have $\langle h^2, T\rangle ^\perp \subset H^{2,2}(X)$ and $H^2_{prim}(S)$ both of signature $(2,19)$. Hassett determines exactly when they could be Hodge isometric:

\begin{theorem}{Hassett}{Hassett}
    There is a Noether-Lefschetz divisor $\calC_d$ in the moduli space of cubic fourfolds, where $d=disc\langle h^2, T\rangle$ for $T\subset X$ a surface not homologous to a complete intersection. This is nonempty precisely when $d>6$ and $d\equiv 0,2 \mod 6$. Moreover, these special cubic fourfolds have an associated K3 surface in the sense above precisely when $d$ is not divisible by $4,9$ or any odd prime $p\equiv -1 \mod 3$
\end{theorem}

The slightly awkward condition will be rephrased later, when we review Addington-Thomas.

\subsection{Derived view of the cubic fourfold}

We now shift gears and look at cubic fourfolds through a derived lens.

Take a cubic fourfold $X$ in $\mathbb{P}^5$. This has canonical bundle $K_{X}=-3H$ so is a Fano of index 3. By using Kodaira vanishing, one can show that $$\mathcal{O}_{X}(-2H), \mathcal{O}_{X}(-H), \mathcal{O}_{X}$$form an exceptional collection. The orthogonal complement is given by $$\mathcal{A}_{X}=\{\mathcal{F}\in \mathbf{D}^b(X)|\,\mathrm{Ext}^\bullet(\mathcal{O}_{X}(-iH), \mathcal{F})=0\}$$This component, named after Kuznetsov, is very interesting, as it looks like the derived category of a K3 surface.

We see that, by \ref{th:HKR}, the Hochschild homology is $HH_\bullet(X)=\mathbb{C}[-2]\oplus \mathbb{C}^{25}\oplus \mathbb{C}[2]$. On the other hand, Hochschild homology is additive with respect to semiorthogonal decomposition, and the three exceptional objects contribute to three copies of $\mathbb{C}$ in degree zero, so we see that $HH_\bullet(\mathcal{A}_X)=\mathbb{C}[-2]\oplus \mathbb{C}^{22}\oplus \mathbb{C}[2]$, which is exactly the Hochschild homology of a K3 surface.

These properties of the category $\mathcal{A}_X$ make it possible that there is an honest, geometric K3 surface associated to $X$. All known cases of cubic fourfolds with associated K3 surfaces are birational to $\mathbb{P}^4$, which led Kuznetsov to conjecture:

\begin{conjecture}{}{Kuznetsov's conjecture}
A cubic fourfold is rational if and only if it there is a K3 surface $S$ such that $\mathcal{D}(S)\simeq \mathcal{A}_X$.
\end{conjecture}

In the subsequent section, we will explain why the Kuznetsov component is a CY2 category and later on focus on the fundamental example of cubics containing a plane.

\subsubsection{The Kuznetsov component is CY2}

We now embark on the proof that the Kuznetsov component is a Calab-Yau 2 category. What this means is that it has a Serre functor which is just given by shifting by $2$, as would be the case if it were the derived category of an honest K3 surface.

%\begin{proposition}{Kuznetsov component is CY2}{Kuznetsov component is CY2}
 %   The category $\mathcal{A}_{X}$ is a CY2 category with Hochschild homology the same as that of a K3 surface. Therefore, it is indecomposable and the semiorthogonal decomposition is maximal.
%  \end{proposition}
    

Let us consider a general degree $d$ hypersurface $X \subset \mathbb{P}^{n+1}$. By adjunction, its canonical bundle is given by $\O(d-n-2)$ and hence its Serre functor is $\otimes \,\O(d-n-2)[n] $. Moreover, the line bundles $\O,\dots,\O(n+1-d)$ are all exceptional and have an orthogonal complement which gives $$\calD(X)=\langle \calA_X,\O,\dots,\O(n+1-d) \rangle $$

\paragraph*{The kernel of the Serre functor}

The left mutation with respect to the admissible subcategory spanned by the line bundles is $$\mL_{\langle \O, \dots, \O(n+1-d) \rangle}\simeq \mL_\O \circ \dots \circ \mL_{\O(n+1-d)}$$

Now we notice the following: the mutation $\mL_{\O(i)}$ fits into an exact triangle $$\Hom(\O(i), F)\otimes \O(i) \xrightarrow{ev} F \rightarrow \mL_{\O(i)} F$$
\todo{Give explanation why ev is the unit and why mutations compose}

On the other hand, we can think of the middle as $\Phi_{\O_\Delta}F$ and the left object as $\Phi_{\O(-i)\boxtimes \O(i)}F$. More precisely, by using the projection and base change formulas:
\begin{gather*}
    \Phi_{\O(-i)\boxtimes \O(i)}F=p_* (p^*\O(i)\otimes q^* \O(-i)q^*F)\simeq p_* q^* F(-i)\otimes \O(i)\simeq\\
    \simeq \O_X \otimes \RG (F(-i))\otimes \O(i)\simeq \RG \RlHom(\O(i),F) \otimes O(i)=\RHom(\O(i),F) \otimes \O(i)
\end{gather*}

Hence, we can conclude that $\mL_{\O(i)}$ is given by a Fourier-Mukai transform with kernel $$[\O(-i)\boxtimes \O(i)\rightarrow \O_\Delta]$$

Now, define $$\mathbf{O}:=\mL_\O \circ (-\otimes \O(1))$$
We see that by \ref{cor:Autoequivalences and mutations}, $$\mathbf{O}^{n+2-d}\simeq \mL_\O \circ \mL_{\O(1)}\circ ... \circ \mL_{\O(n+1-d)}\circ (-\otimes \O(n+2-d))=\mL_{\langle \O, \dots, \O(n+1-d) \rangle} \circ (-\otimes \omega ^{-1})$$

But by \ref{lemma:Serre functors of admissible subcategories}, we know that the inverse of the Serre functor of $\calA_X$ is given by $$\calS_{\calA_X}^{-1} =  \mL_{\langle \O, \dots, \O(n+1-d) \rangle} \circ \calS_X ^{-1}=\mL_{\langle \O, \dots, \O(n+1-d) \rangle} \circ (-\otimes \omega^{-1})[-n]$$

We can thus conclude that $$\calS_{\calA_X}^{-1} = \mathbf{O}^{n+2-d}\circ [-n]$$

If we put $T:=\mathbf{O}|_{\calA_X}$, then we can reinterpret this as $$\calS_{\calA_X}=T^{d-n-2}\circ [n]$$

Recall that $\mL_\O$ had kernel given by $[\O \boxtimes \O \rightarrow \O_\Delta]$. We can compose this with the kernel for $-\otimes \O(1)$ which is $\O_\Delta(1)$ to see that the kernel for $T$ is given by $K_1=[\O(1)\boxtimes \O \rightarrow \O_\Delta(1)]$, i.e. we have an exact triangle $K_1 \rightarrow  \O(1)\boxtimes \O \rightarrow \O_\Delta(1)$. 

Our ultimate aim is to show that a suitable power of the Serre functor is just a shift functor, so we need to understand the kernel for $T^i, i=1,2,\dots, $. For this purpose, we need to convolute $K_1$ with itself multiple times. 

\begin{proposition}{Kernel of Serre functor of Kuznetsov component}{Kernel of Serre functor of Kuznetsov component}   
The kernel $K_1^i$ fits into a sequence $$K_1^i\rightarrow \O(1)\boxtimes \Omega^{i-1}(i-1)\rightarrow \dots \rightarrow \O(i-1)\boxtimes \Omega(1)\rightarrow \O(i)\boxtimes \O \rightarrow \O_\Delta(i)$$
\end{proposition}
\begin{proof}
    We show this by induction.
    
    Firstly, composing with a Fourier-Mukai transform preserves exact triangles, so the same holds for the kernels. We can compose the triangle for $K_1$ with $\O(1)\boxtimes \O, \O_\Delta(1)$ on the left and $K_1$ on the right respectively to get three exact triangles \begin{gather*}
        (\O(1)\boxtimes \O) \circ K_1 \rightarrow (\O(1)\boxtimes \O)  \circ (\O(1)\boxtimes \O) \rightarrow (\O(1)\boxtimes \O) \circ \O_\Delta(1)\\
        \O_\Delta(1)\circ K_1 \rightarrow \O_\Delta(1) \circ (\O(1)\boxtimes \O)  \rightarrow \O_\Delta(1) \circ \O_\Delta(1)\\
        K_1 \circ K_1  \rightarrow (\O(1)\boxtimes \O)\circ K_1 \rightarrow \O_\Delta(1)\circ K_1
    \end{gather*}
    
    We can compute some of these: for example, the middle convolution on the first row is given by $$(\O(1)\boxtimes \O)  \circ (\O(1)\boxtimes \O) ={\pi_{13}}_*\big(\O(1)\boxtimes \O \boxtimes \O \otimes \O \boxtimes \O(1) \boxtimes O\big)=\big(\O(1)\boxtimes \O\big) \otimes H^\bullet(\O(1))$$
    as pushing down is the same as cohomology on the fibers. This is the only computation that involves any cohomology: the others are given by tensoring with the diagonal, which turns the first two triangles into: \begin{gather*}
        (\O(1)\boxtimes \O) \circ K_1 \rightarrow \big(\O(1)\boxtimes \O\big) \otimes H^\bullet(\O(1)) \rightarrow \O(1)\boxtimes \O(1)\\
        \O_\Delta(1)\circ K_1 \rightarrow \O(2)\boxtimes \O \rightarrow \O_\Delta(2)
    \end{gather*}
    
    Now recall the Euler sequence on $\mathbb{P}^{n+1}$, which says that there is an exact triangle $$\Omega(1)\rightarrow \O^{n+2}\rightarrow \O(1)$$
    Since the cohomology of $O(1)$ is $n+2$-dimensional, we can read off from the first exact triangle that $ (\O(1)\boxtimes \O) \circ K_1\simeq \O(1) \boxtimes \Omega(1)$. We can now plug this into the second object in the last triangle, as well as replace the last object in the third triangle by the second triangle to get: $$K_1 \circ K_1 \rightarrow \O(1)\boxtimes \Omega(1)\rightarrow \O(2)\boxtimes \O \rightarrow \O_\Delta(2)$$
    
    Now, assume the statement holds for $i$ (we have just shown it holds for $i=2$). Then we can apply $-\circ K_1$. We notice by the same argument that $\O_\Delta(i)\circ K_1= [\O(i+1)\boxtimes \O \rightarrow \O_\Delta(i+1)]$ and also using cohomology on the fibers and the Euler sequence that $\big( \O(i)\boxtimes O\big)\circ K_1=\O(i)\boxtimes \Omega^1(1)$. What we need to understand is the other parts $\big(\O(i-k)\boxtimes \Omega^{k}(k)\big) \circ K_1$. This can be done by applying $\big(\O(i-k)\boxtimes \Omega^k(k)\big)\circ -$ to the triangle defining $K_1$. What we get is the following: $$\big(\O(i-k)\boxtimes \Omega^k(k)\big)\circ K_1 \rightarrow \O(i-k)\boxtimes \O \otimes H^\bullet(\Omega^k(k+1))\rightarrow \O(i-k)\boxtimes \Omega^k(k+1)$$
    
    Now we need to use a variant of the Euler sequence, \cite[Corollary 17.1.3]{arapura_algebraic_2012}: $$\Omega^{k+1}(k+1)\rightarrow \O^{\binom{n+2}{k+1}}\rightarrow \Omega^k(k+1)$$
    
    This, together with a computation of the cohomology of $\Omega^k(k+1)$ having dimension $\binom{n+2}{k+1}$ (it has only $H^0$ using Bott vanishing etc.), tells us that $\big(\O(i-k)\boxtimes \Omega^{k}(k)\big) \circ K_1=\O(i-k)\boxtimes \Omega^{k+1}(k+1)$. This completes the induction, which is illustrated in the picture below, where the squiggly arrows denote convolution with $K_1$:
        \[\begin{tikzcd}
            {K_1^i} & {\O(1)\boxtimes \Omega^{i-1}(i-1)} & \dots & {\O(i)\boxtimes \O} & {\O_\Delta(i)} \\
            {K_1^{i+1}} & {\O(1)\boxtimes \Omega^{i}(i)} & \dots & {\O(1)\boxtimes \Omega(1)} & {[\O(i+1)\boxtimes \O \rightarrow \O_\Delta(i+1)]} \\
            & {}
            \arrow[from=1-1, to=1-2]
            \arrow[from=1-2, to=1-3]
            \arrow[from=1-3, to=1-4]
            \arrow[from=1-4, to=1-5]
            \arrow[from=2-1, to=2-2]
            \arrow[squiggly, from=1-1, to=2-1]
            \arrow[squiggly, from=1-2, to=2-2]
            \arrow[squiggly, from=1-3, to=2-3]
            \arrow[from=2-2, to=2-3]
            \arrow[squiggly, from=1-4, to=2-4]
            \arrow[squiggly, from=1-5, to=2-5]
            \arrow[from=2-4, to=2-5]
            \arrow[from=2-3, to=2-4]
        \end{tikzcd}\]
\end{proof}
\paragraph*{The main result}

\begin{theorem}{Shift functor}{}
    The functor $T$ has $T^d=[2]$.
\end{theorem}

\begin{proof}
    In the course of the proof, we write $\P=\mathbb{P}^{n+1}$. Firstly, let's recall the Koszul resolution of $X$ in $\mathbb{P}^{n+1}$ and $X\times X\subset \P\times \P$. Since $X$ is given by a section of $\O(d)$ and $X\times X$ by a section of $\O(d) \boxtimes O \oplus \O \boxtimes \O(d)$, we have the following resolutions: \begin{gather*}
        0\rightarrow \O(-d)\rightarrow \O_\P \rightarrow \iota_* \O_X\rightarrow 0\\
        0\rightarrow \O(-d,-d)\rightarrow \O(-d,0)\oplus \O(0,-d)\rightarrow \O_{\P\times \P}\rightarrow (\iota\times \iota)_* \O_{X\times X}\rightarrow 0
    \end{gather*}

    We wish to understand the derived pullback $(\iota\times \iota)^* \O_{\Delta_\P} $. Instead, let us first consider $(\iota\times \iota)^*(\iota\times \iota)^* \O_{\Delta_\P} \simeq (\iota\times \iota)_* \O_{X\times X} \otimes \O_{\Delta_\P}$, by the projection formula. This derived tensor product is given by tensoring the Koszul resolution above with the diagonal. The resulting complex is $$\O_{\Delta_\P}(-2d)\rightarrow \O_{\Delta_\P}(-d)^{\oplus 2}\rightarrow \O_{\Delta_\P}$$
    The first map is injective, and this is just the sum of the two resolutions $\O_{\Delta_\P}(-d)\rightarrow \O_{\Delta_\P}\rightarrow \O_{\Delta_X}$ and $\O_{\Delta_\P}(-2d)\rightarrow \O_{\Delta_\P}(-d)\rightarrow \O_{\Delta_X}(-d)$. We conclude that $L_{-1}=\calH^{-1}=\O_{\Delta_X}(-d), L_0=\calH^{0}=\O_{\Delta_X}$. Now we note that $\iota$ is a closed embedding, hence exact and conservative, so we can ignore it.

    This now allows us to write down an exact triangle $$\O_{\Delta_X}(-d)[1]\simeq \calH^{-1}[1]\rightarrow (\iota\times \iota)^*\O_{\Delta_\P}\rightarrow \calH^0\simeq \O_{\Delta_X}$$
    We can rotate and twist by $d$ to get the exact triangle 
    \begin{equation}\label{eqn:derivedtriangle}
        (\iota\times \iota)^*\O_{\Delta_\P}\rightarrow \O_{\Delta_X}(d)\rightarrow \O_{\Delta_X}[2]
    \end{equation}

    Now recall the Beilinson resolution of the diagonal, which we restrict to $X\times X$:\begin{gather*}
        0 \rightarrow \O (d-n-1)\boxtimes \Omega^{n+1}(n+1)\rightarrow \dots \rightarrow \O \boxtimes \Omega^{d}(d)\\
        \rightarrow \O(1)\boxtimes \Omega^{d-1}(d-1)\rightarrow \dots \rightarrow \O(d)\boxtimes \O \rightarrow \O_\Delta
    \end{gather*}

    We splice it into the top bit, where all the $\O$'s are non-positive, and the positive bit below, which is just the kernel $K_d$, by \ref{prop:Kernel of Serre functor of Kuznetsov component}. Let us call $K'_d$ the complex which is $K_d$ without the diagonal bit at the end. Then we have an exact triangle $$K'_d \rightarrow \O_{\Delta_X}(d)\rightarrow K_d$$

    We now compare this with triangle \ref{eqn:derivedtriangle}, by using the natural map $K'_d \rightarrow (\iota\times \iota)^*\O_{\Delta_\P}$ coming from Beilinson's resolution, and the identity map in the middle, which can be extended to a map $K_d \rightarrow \O_{\Delta_X}[2]$ by the axioms of triangulated categories: 

    \begin{center}\begin{tikzcd}
        K'_d \arrow[d] \arrow[r]                      & \O_{\Delta_X}(d) \arrow[d] \arrow[r] & K_d \arrow[d, dashed] \\
        (\iota\times \iota)^*\O_{\Delta_\P} \arrow[r] & \O_{\Delta_X}(d) \arrow[r]           & {\O_{\Delta_X}[2]}   
        \end{tikzcd}\end{center}

    Finally, we compare the effect of taking these as Fourier-Mukai kernels. We know that $\Phi_{K_d}=T^d$ and $\Phi_{\O_\Delta [2]}=[2]$ and we claim that they are isomorphic by the dotted arrow. To see this, we show that the other vertical arrows induce isomorphisms on Fourier-Mukai transforms.

    The middle bit is obvious, however to compare $\Phi_{K'_d}$ and $\Phi_{(\iota\times \iota)^*\O_{\Delta_\P}}$ we simply look again into the Beilinson resolution: we have that \begin{gather*}
        [ \O (d-n-1)\boxtimes \Omega^{n+1}(n+1)\rightarrow \dots \rightarrow \O \boxtimes \Omega^{d}(d)\rightarrow K'_d]=(\iota\times \iota)^*\O_{\Delta_\P}
    \end{gather*}

    However, for any $\calE_i=\O(d-i)\boxtimes \Omega^i(i)$ with $i=d,d+1,\dots,n+1,$ its Fourier-Mukai transform on $A\in \calA_X$ vanishes, by using the projection formula, base change and the fact that $\calA_X$ is orthogonal to $\O,\dots, \O(n+1-d)$:$$\Phi_{\calE_i}(A)=q_*(p^*A\otimes p^* \O(d-i)\otimes q^* \Omega^i(i))=\Omega^i(i)\otimes q_*p^* A(d-i)=\Omega^i(i) \otimes \RHom(\O(i-d),A)=0$$ We conclude that $\Phi_{K'_d}\simeq \Phi_{(\iota\times \iota)^*\O_{\Delta_\P}}$ and hence $T^d=[2]$.
\end{proof}

As an immediate corollary we get:
\begin{corollary}{Kuznetsov}{}
    If $2d>n+1$, then $\calA_X$ is a fractional Calabi-Yau category whose Serre functor obeys $\calS_{\calA_X}^{d/c}=[(n+2)(d-2)/c]$, where $c=\gcd(d,n+2)$.

    ~

    In the special case that $X$ is a cubic fourfold, $n=4,d=3$ hence $\calS=T^{-3}\circ [4]=[2]$. More generally, whenever $d|n+2$, the component is Calabi-Yau.
\end{corollary}

\subsection{Fundamental example: cubics containing a plane}
We now study the case of cubic fourfolds containing a plane $P\simeq \mathbb{P}^2$. These form a divisor $\calC_8$ in the moduli space of all cubic fourfolds of special importance, as we will see in the review of Addington-Thomas. They also play a central role in Voisin's proof of the global Torelli theorem.

Note that they do not belong on Hassett's list of cubic fourfolds with associated K3 surfaces: in fact, we will see that their Kuznetsov component is equivalent to $\calD(S,\alpha)$, the twisted derived category of a K3 surface $S$ relative to a Brauer class $\alpha\in H^2(S, \O_S^*)$. When the Brauer class vanishes, the cubic fourfold is rational and there is a different surface $T\neq P$ inside of $X$ which places it in $\calC_8\cap \calC_d$ with $d$ on Hassett's list, which is consistent with the predictions about associated K3 surfaces and rationality.

\subsubsection{Geometric constructions}

Consider $P\subset X \subset \P^5$ a cubic fourfold containing a plane $P$. The variety parametrizing $3$-planes containing $P$ is also a plane, and $X$ intersects such a $3$ plane in $P$ combined with a quadric $Q$, since generically the intersection of $X$ with a 3-plane should be degree $3$.

More precisely, we can blowup $P$ in $\P^5$ which resolves the rational map defined by the linear system $\calI_P \otimes \O(1)$ to a map defined by the complete linear system $\calI_E \otimes \tau^*\O(1)$. 

\begin{center}
\begin{tikzcd}
    \P \calN_{P/ \P^5} \simeq E \arrow[d] \arrow[r] & \mathrm{Bl}_P \P^5 \arrow[d, "\tau"] \arrow[rd] &      \\
    P \arrow[r]                             & \P^5 \arrow[r, dotted, "\phi"]                          & \P^2
    \end{tikzcd}
\end{center}

By definition, this means that $\phi^* \O(1)=\calI_E \otimes \tau^*\O(1)$.

If $P=\P V$, $V\oplus W =\C^6$ and we choose coordinates so that $V=\mathbb{V}(z_0,z_1,z_2)$, then the map $\P^5\rightarrow \P W\simeq \P^2$ is given by $z \mapsto [z_0:z_1:z_2]$ since these are the linear forms vanishing on $P$. This is then resolved by blowing up $P$ and essentially projecting down from the $\P^2$ fiber in the projectivized normal bundle, which is trivial:  $$E \simeq P \times \P^2 \xrightarrow{\phi = \pi_2} \P^2$$

We can think of the blowup in this simplified setting as the set $$\mathrm{Bl}_P \P^5 = \{p\in \P^5, q\in \P^2 |\, p_iq_j=p_jq_i, 0\leq i,j \leq 2\} \subset \P^5 \times \P^2$$
which comes equipped with two projection maps to $\P^5$ and $\P^2$ which are the blowup and resolution maps, respectively. We can clearly see that over $\P^2$, this is a projective bundle by looking at the affine cone over $\P^5$: $$\{ a\in \A^6, q\in \P^2|\,a_iq_j=a_jq_i, 0\leq i,j \leq 2\}\rightarrow \P^2$$

We can see that the coordinates $a_3, a_4, a_5$ are unconstrained, so give us a copy of the trivial bundle $\O^{\oplus 3}$. The other three coordinates give us precisely the tautological bundle: $$\O(-1)=\{(a_0,a_1,a_2)\in \A^3, q\in \P^2|\, (a_0, a_1, a_1)\in q\}$$

Hence, we conclude that $$\mathrm{Bl}_P \P^5 = \P (\O(-1)\oplus \O^{\oplus 3})$$

However, this isomorphism is slightly non-canonical: a coordinate-free alternative is to put $\calF = \phi_* \tau^* \O(1)$ which gives $\mathrm{Bl}_P \P^5 = \P \calF^*$.

Now, we restrict to $X$: the blowup $\tilde{X}:=\mathrm{Bl}_P X$ is the strict transform of $X$ in $\mathrm{Bl}_P \P^5$. If $F$ is the defining function of $X$, which is a cubic, we expect generically the fiber of $\tilde{X}\rightarrow \P^2$ to be a quadric, since the blowup only gives a linear condition. The reason for this is that $\tau^*F \in H^0(\tau^* \O(3) \otimes \calI_E)$ since $F$ vanishes on $P$ and hence $\tilde{X}$ is defined as the zero locus of a section of the line bundle $\calL := \tau^* \O(3) \otimes \calI_E$. To understand the fibers of $\tilde{X}$ we push this line bundle down to $\P^2$: $$\phi_*(\tau^* \O(3) \otimes \calI_E)=\phi_* \tau^* \O(2) \otimes \O(1)$$

Pushing down means taking cohomology on the fibers, hence $\phi_* \tau^* \O(2)=\phi_* \O_\phi(2)=\mathrm{Sym}^2 \calF$ essentially since $H^0(\O(2))=\mathrm{Sym}^2 H^0(\O(1))$. 

\begin{remark}{Relative line bundles on projective bundles}{Relative line bundles on projective bundles}

    More generally \cite[Remark~13.36]{Wedhorn}, given a projective bundle $\pi:\P\calE^\lor \rightarrow B$ which can be defined as a relative Proj of a sheaf of algebras $\underline{\Proj}\, \mathrm{Sym}^\bullet \calE$, we have the following fact: $$\pi_* \O_{\P\calE}(d)=\mathrm{Sym}^d \calE$$ 

    In particular, $\pi_*\O(1)=\calE$. Recall the relative Euler sequence: $$0\rightarrow \O\rightarrow \pi^* \calE \otimes \O_{\pi}(1)\rightarrow \calT_{\P\calE/B}\rightarrow 0$$
    %since $$\calT_\pi:=\calT_{\P\calE/B}\simeq \lHom(\O_{\pi}(-1),\mathcal{Q})\simeq \frac{\lHom(\O_{\pi}(-1),\mathcal{Q}\oplus \O_{\pi}(-1))}{\lHom (\O_{\pi}(-1), \O_{\pi}(-1))}\simeq \frac{\lHom(\O_{\pi}(-1), \pi^* \calE)}{ \O}$$  

    By twisting by $-1$ and taking determinants, we see that $$\pi^* \det \calE \simeq \O_\pi(-1) \otimes \det(\calT_\pi\otimes \O_\pi(-1))\simeq \O_\pi(-k-1) \otimes \det(\calT_\pi)$$and so $$\omega_\pi\simeq \pi^* \det \calE^* \otimes \O_{\pi}(-k-1)$$

    Moreover, the derived pushforward is given by $\pi_* \omega_\pi\simeq \O_B[-1]$, and so we conclude that $\pi_* \O_\pi(-k-1) \simeq \det \calE^*[-1]$. 
 \end{remark}

All in all, we see that $$\phi_* \calL = \mathrm{Sym}^2 \calF \otimes \O(1)\subset\lHom (\calF^* , \calF \otimes \O(1))$$

We see that the fiber over $q \in \P^2$ is the residual quadric defined by the quadratic form $q$ corresponding to $F$ which is a section of $\mathrm{Sym}^2\calF$. This quadric is smooth, unless we are in the discriminant locus where $\det(q)$=0. This is a section of the bundle $\Hom(\det(\calF^*), \det (\calF\otimes \O(1))\simeq \O(6) $ since $\det \calF=1$ and $\det \calF\otimes \O(1)= \det (\O(2) \oplus \O(1)^{\oplus 3})=5$. So the discriminant locus is a sextic curve $C$!

Let us summarize what we have done so far:

\begin{proposition}{Blowup quadric fibration}{Blowup quadric fibration}
    The blowup of a cubic fourfold $\tilde{X}$ at a plane projects to $\P^2$ with quadric fibers, namely the residual quadrics complementary to $P$ in the intersection between the 3-plane spanned by $P$ and $q$ with $X$. They degenerate to singular ones over a sextic curve $C$.
\end{proposition}


Now, a generic quadric is isomorphic to $\P^1 \times \P^1$ and has exactly two rulings, parametrized by the connected components of the Fano variety of lines $$\scrF(\P^1 \times \P^1)=\P^1 \coprod \P^1$$
A singular quadric is a cone, so has exactly one ruling: $$\scrF(Q)=\P^1$$

If we define $S$ to be the space of rulings on the fibers, we see that it is a double cover of $\P^2$ branched along the sextic curve. Moreover, the relative Fano variety parametrizing lines in the fibers is a $\P^1$ bundle over this: $$\tscrF \xrightarrow{\pi_S} S \xrightarrow{\pi} \P^2$$
In other words, for every $y\in \P^2$, the fiber of $\tscrF$ is equal to $\scrF(Q_y)$, the Fano variety of lines in the residual quadric corresponding to $y$.

In fact, this $S$ is a K3 surface: since it is a double cover of $\P^2$, it has $h^{0,1}(S)=h^{0,1}(\P^2)=0$ and its canonical bundle is given by $$K_S = \pi^*(K_{\P^2}+\frac{1}{2}C)=\pi^*(-3H +\frac{1}{2}6H)=0$$

The Fano varieties carry a universal $\P^1$-bundle $$\scrP = \{ (x,L)\in X\times \scrF | \, x\in L\} \subset X \times \scrF$$
and similarly a pullback to $\tscrF$ given by $\tscrP = \{ x,L | \, x\in L\} \subset X \times \tscrF,$
fitting into a diagram: 
\begin{center}
    \begin{tikzcd}
        & \tscrP \arrow[ld] \arrow[rd] \arrow[d] &                     \\
\tscrF \arrow[d] & \ \arrow[ld] \arrow[rd]   \scrP                   & \tilde{X} \arrow[d] \\
\scrF                   &                                                   & X                  
\end{tikzcd}
\end{center}

\begin{remark}{Brauer-Severi varieties}{}
    We have seen that $\tscrF\rightarrow S$ is a $\P^1$ bundle. There is a cohomological obstruction $\alpha \in H^2(S, \O_S^\times)$ to this being a projectivized vector bundle, which can be seen by locally trivializing and trying to glue over an open cover. However, there is an $\alpha$-twisted vector bundle $\calE$ such that $\tscrF=\P \calE$ and a sheaf of Azumaya algebras $\calB_0 = \mathrm{End}\,\calE$ whose Brauer class is $\alpha$.  Even more is true: the pushforward to $\P^2$ defined as $\pi_* \calB_0 = \calC_0$ is the sheaf of even Clifford algebras corresponding to $(\calF^*, q)$: $$\calC_0=\O \oplus \bigwedge\nolimits^2 \calF^*(-1)\oplus \bigwedge\nolimits^4\calF^*(-2)$$ whose center $\mathcal{Z}$ realizes $S=\calS pec\, \mathcal{Z}$ as a relative Spec of a sheaf of algebras. For more on this, consult \cite{auel_fibrations_2014}. In fact, Kuznetsov \cite{KuznetsovDerivedCubic} shows that $$\calD(\P^2, \calB_0)\simeq \calD(S, \alpha)$$
    
\end{remark}

In the next section, we will see how the above correspondence allows us to define a Fourier-Mukai transform $\Phi_{\calI(1)}: \calD (\tscrF)\rightarrow \calD (X)$ which sends a twisted subcategory $\calD(S,\alpha)$ isomorphically to the Kuznetsov component.

\subsubsection{Twisted sheaves on the associated K3 surface}

We have seen that to every cubic fourfold containing a plane, we can associate some K3 surface by blowing up the plane and realizing $S$ as the variety of rulings on the fibers of the quadric fibration $\tilde{X}\rightarrow \P^2$. Given that the Kuznetsov component looks like the derived category of a K3 surface, it is natural to conjecture that these two are related. In this section, we review two approaches to proving the following: 
\begin{proposition}{Kuznetsov}{Kuznetsov twisted K3}
    There is an equvalence between the derived category of twisted sheaves on $S$ and the Kuznetsov component: $$\calD(S,\alpha)\simeq \calA_X$$
\end{proposition}

\begin{proof}[Proof 1]
    The original approach is due to Kuznetsov (Derived category of cubic fourfolds). The point is that on the one hand, the blowup formula \ref{th:Bondal-Orlov blowup formula} shows that $$\calD(\tilde{X})\simeq \langle \calD(P), \calD(X)  \rangle $$
    This has $6$ exceptional objects and the category $\calA_X$.
    On the other hand, Kuznetsov's work on quadric fibrations shows that there is another semiorthogonal decomposition $$\calD(\tilde{X})=\langle \calD(S,\alpha), \calD(P), \calD(P) \otimes \O(1)\rangle$$
    This also has $6$ exceptional objects, together with the category $\calD(S,\alpha)$.

    A sequence of mutations then identifies the two categories in question.
\end{proof}

\begin{proof}[Proof 2]
A second approach is found in \cite[\S7]{huybrechts_geometry_2023} by using a Fourier-Mukai kernel. 

\textbf{Step 1}: We first choose a kernal and define a functor witnessing the desired equivalence.

The universal line $\tscrP\subset X \times \tscrF$ has an ideal sheaf $\calI$ and we consider the Fourier-Mukai transform associated to the correspondence: 
\[\begin{tikzcd}
    & X\times \tscrF \arrow[ld, "\pi_X"'] \arrow[rd, "\pi_{\tscrF}"] &        \\
  X &                                                                & \tscrF
  \end{tikzcd}\]
$$\Phi_{\calI(1)}: \calD(X)\rightarrow \calD(\tscrF)$$

Now, a theorem of Bernardara \cite{bernardara_semiorthogonal_2005} extends the projective bundle formula to Brauer-Severi varieties and we can apply it to the $\P^1$-bundle $\tscrF\xrightarrow{\pi_S} S$ hence have a semiorthogonal decomposition $$\calD(\tscrF)=\langle \calD(S,\alpha), \calD(S)\rangle$$

The $\calD(S)$ is included via the projection $\pi_S^*$ to $S$ and the twisted $\calD(S,\alpha)$ is included via $\pi_S^*\otimes \O_{\pi_S}(-1)$: we have to tensor by a relative twisted $\O_{\pi_S}(-1)$ so as to untwist and land in the honest derived category. We can consider the adjoint $\Psi$ of $\Phi$ and compose it with the inclusion  $\calD(S,\alpha)\hookrightarrow \calD(\tscrF)$ to get $$\Psi':\calD(S,\alpha)\rightarrow \calD(X)$$

The claim is that this lands in the Kuznetsov component and is a full and faithful functor. The fact that both are indecomposable Calabi-Yau 2 categories then completes the proof.

\textbf{Step 2}: We show that $\im \Psi'\subset \calA_X$.

We need that $\Psi'(E)\in \calA_X= \langle \O, \O(1), \O(2)\rangle ^\perp = \lperp ( \langle \O, \O(1), \O(2)\rangle\otimes \O(-3))$ by Serre duality. In other words, by adjointness we want $$\Hom(\Psi(E), \O(-i))=0 \iff \Hom (E, \Phi \O(-i))=0 \iff \Phi \O(-i) \in \calD(S,\alpha)^\perp, i=1,2,3$$

The ideal sheaf sequence $$0\rightarrow \calI \rightarrow \O_{X\times \tscrF}\rightarrow \O_{\tscrP}\rightarrow 0$$ implies that we have a triangle of FM functors $$\Phi_{\cal I}\rightarrow \Phi_{\O_{X\times \tscrF}}\rightarrow \Phi_{\O_{\tscrP}}$$

Applying this to $\O_X(-1)$ we see that $\Phi_{\O_{X\times \tscrF}(1)}\O(-1)=\Phi_{\O_{X\times \tscrF}}(\O_X)=\O$ by base change on the product since $H^\bullet(X, \O_X)=\C$ and similarly $\Phi_{\O_{\tscrP}(1)}(\O_X(-1))=\Phi_{\O_{\tscrP}}(\O_X)=\O$ since $\tscrP\rightarrow \tscrF$ is a $\P^1$ bundle. The two are isomorphic, and hence $\Phi(\O(-1))=0$ as the cone of an isomorphism.

Similarly, applying it to $\O_X(-2)$, we first see that $\Phi_{\O_{X\times \tscrF}}(\O_X(-1))=0$ since this is doing cohomology on the fibers $H^\bullet(X, \O_X(-1))=H^\bullet(X, \omega_X \otimes \O_X(2))=0$ by Bott vanishing. Moreover, $\Phi_{\O_{\tscrP}}(\O_X(-1))=0$ since this is cohomology on the $\P^1$ fibers, and $H^\bullet(\P^1, \O(-1))=0$.

Finally, we need to consider $\O_X(-3)$. We still have Bott vanishing for $X$ i.e. $\Phi_{\O_{X\times \tscrF}}(\O_X(-2))=0$ since $H^\bullet(X, \O_X(-2))=H^\bullet(X, \omega_X \otimes \O_X(1))=0$. However, the cohomology of $\O(-2)$ on the $\P^1$ fibers is nonzero, so we only have that $\Phi \O(-3)\simeq \Phi_{\O_{\tscrP}}(\O(-2))[-1]$. Explicitly, by Remark \ref{rem:Relative line bundles on projective bundles}, we have 
$$\Phi_{\O_{\tscrP}}(\O(-2))={\pi_{\tscrF}}_* \O_{\pi_{\tscrF}}(-2)=\O_{\tscrF} (-1)[-1]$$
The desired vanishing is thus equivalent via Serre duality to $$0 = \Hom(E, \Phi \O(-3))=\Hom(E, \O_{\tscrF} (-1)[-2])\iff 0=\Hom(\O_{\tscrF} (-1), E\otimes \omega_{\tscrF})$$

We only need to show this for skyscrapers. But they are embedded via $\pi_S^* \otimes \O_{\pi_S}(-1)$ so a skyscraper $\O_x \in \calD(S,\alpha)$ corresponds to $\O_{F_x}(-1)$, and so we are reduced to showing $$\Hom(\O_{\tscrF}(-1), \O_{F_x}(-1)\otimes\omega_{\tscrF})\simeq H^\bullet(F_x, \O_{F_x}(-1) \otimes \O_{F_x}(-2)\otimes \O_{\tscrF}(-1)|_{F_x})=H^\bullet(\P^1, \O(-1))=0$$since the embedding $\P^1 \simeq F_x \subset \tscrF$ is of degree $2$, i.e. it pulls back the Plucker polarization coming from the tautological line $\O(1)$ on $\P \bigwedge^2 \mathbb{C}^6$ to $\O(2)$.

\textbf{Step 3}: The final step in the proof consists of showing that $\Psi'$ is fully faithful, which requires the use of the criterion by Bondal-Orlov \ref{prop:Generation criterion}. 

Firstly, we identify the kernel for $\Psi'$. By \ref{prop:Adjoints of FM transform}, we know that is must be $\calI(1)^\vee \otimes \pi_X^* \omega_X [3]$. The generation criterion then requires to check what happens on skyscrapers $\O_s, s\in S$. Explicitly, $$\Psi'\O_s={\pi_X}_* ( \calI(1)^\vee \otimes \omega_X \boxtimes \O \otimes \pi_{\tscrF}^*\O_s)[3]$$

By the way we've set things up, the pullup is twisted by the way $\calD(S,\alpha)$ includes into $\calD(\tscrF)$, namely: $\pi^*_{\tscrF}\O_s=\O_{F_s}(-1)$ where $F_s \simeq \P^1$ is the fiber over $s$ in the $\P^1$ bundle $\tscrF\rightarrow S$. Hence, the object in question is $$\Psi'\O_s={\pi_X}_*(\calI_s(1)^\vee \otimes (\omega_{\tscrF}\otimes \O_{F_s}(-1))\boxtimes \O_X)[3]$$

Here, $\calI_s$ is the ideal sheaf of the subvariety $\tscrP_s \subset X \times F_s$, since tensoring with the skyscraper has made everything supported on a fiber. Under the projection $\pi_s: F_s\times X \rightarrow X$, the relative canonical sheaf is just $\omega_{\tscrF}\boxtimes \O_X$. Hence, we can apply Grothendieck-Verdier duality \ref{ex:Shriek functor}:

\begin{gather*}
    {\pi_s}_*(\calI_s(1)^\vee \otimes \omega_{\pi}\otimes \O_{F_s}(-1)\boxtimes \O_X)[3]\simeq 
    {\pi_s}_*\lHom(\calI_s\otimes \O\boxtimes \O_X(1), \omega_{\pi}\otimes \O_{F_s}(-1)\boxtimes \O_X)[3]\simeq \\
    {\pi_s}_*\lHom(\calI_s\otimes \O_{F_s}(1)\boxtimes \O_X(1), \omega_{\pi})[3]\simeq
    \lHom(({\pi_s}_*)\calI_s\otimes \O_{F_s}(1)\boxtimes \O_X(1), \O[2])\simeq\\
    ({\pi_s}_*(\calI_s\otimes \O_{F_s}(1)\boxtimes \O))^\vee\otimes \O_X(-1) [2]
\end{gather*}

We thus need to understand ${\pi_s}_*(\calI_s\otimes \O_{F_s}(1)\boxtimes \O)$. This is really the Fourier-Mukai transform of $\O_{F_s}(1)$ with kernel $\calI_s$, which fits into the triangle $$\calI_s\rightarrow \O_{X\times F_s}\rightarrow \O_{\tscrP_s}$$and hence we have a triangle $$\Phi_{\calI_s}(\O_{F_s}(1))\rightarrow \Phi_{\O_{X\times F_s}}(\O_{F_s}(1))\rightarrow \Phi_{\O_{\tscrP_s}}(\O_{F_s}(1))$$

Hence, we see that $$(\Psi'\O_s)^\vee\otimes \O_X(-1)[2]\simeq {\pi_s}_*(\calI_s\otimes \O_{F_s}(1)\boxtimes \O)\simeq \ker[H^0(X,\O_{F_s}(1))\otimes \O_X \rightarrow \O_Q(1)]:=K_s$$
where $Q$ is the residual quadric in $X$ corresponding to $s$.

The only thing left is to check the Bondal-Orlov criterion on $K_s$, since $$\Ext^i(\Psi'\O_s, \Psi'\O_t)\simeq \Ext^i(K_s,K_t)$$

The fact that $\Ext^0(K_s,K_s)=\C$ follows from $\dots$

$\Ext^i(K_s,K_s)$ vanishes for $i<0$ since this is a sheaf and the case $i>2$ follows by Serre duality.

When $s\neq t$, we need to check vanishing of $\Ext^0(K_s, K_t)$ which then implies by Serre duality vanishing of $\Ext^2(K_s, K_t)$. The final step, to check $\Ext^1(K_s, K_t)=0$, is given by a computation of the Euler characteristic: we know that since $H^0(X,\O_{F_s}(1))=\C^2$, that $K_s=2\O_X-\O_Q(1)$ in the Grothendieck ring. Therefore,

$$\chi(K_s,K_t)=4\chi(\O_X,\O_X)-2\chi(\O_X,\O_Q(1))+2\chi(\O_Q(1),\O_X)-\chi(\O_Q,\O_Q)=0$$






\end{proof}
\todo{Finish the proof}

\paragraph*{Family version}

This is an aside remark, which explains that one can use Homological Projective Duality and study the problem above in a $\P^1$-family. This was done in \cite{calabrese_derived_2016} and bears the following structure: 

Firstly, suppose we have a pencil of cubic fourfolds. This produces universal cubic hypersurface $$\mathcal{H}\rightarrow \P^1$$whose fiber over $t$ is the associated cubic fourfold to $t$. This also has a Kuznetsov component and an SOD $$\calD(\mathcal{H})=\langle \calA_{\mathcal{H}}, \pi^* \calD\P^1(i,0); i=3,4,5 \rangle $$

Homological Projective Duality prescribes an equivalence $$\calA_{\mathcal{H}} \simeq \calD(\mathcal{X})$$where $\mathcal{X}$ is the base locus of the linear system.

Consider, in particular, a pencil of cubics containing our fixed plane $P$. This is equivalent to choosing two sections $s_0,s_\infty$ of the linear system defining $\tilde{X}$ which is $\calI_E\otimes \tau^* \O(3)$. The base locus $\mathcal{X}=\{s_0=s_\infty=0\}$ of this sub-linear system is a $(3,3)$ complete intersection in $\P^5$, which is a Calabi-Yau threefold.

Then, in fact, the universal hypersurface is given by $$\mathcal{H}=\mathrm{Bl}_\mathcal{X}(\mathrm{Bl}_P\P^5)\subset {Bl}_P\P^5\times \P^1 \rightarrow \P^2 \times \P^1$$
which we can think of as a quadric fibration over $\P W\times \P^1$. Work of Kuznetsov then shows that $$\calD(\mathcal{X}) \simeq \calD(\P^2 \times \P^1, \calC_0)$$ where $\calC_0$ is the associated Azumaya algebra to the quadric fibration. 

Before, we had that $$\calD(\P^2, \calB_0)\simeq \calD(S, \alpha)$$ for the K3 surface $S$, but now we can study it in families to produce a $\P^1$-family of K3 surfaces $Y$, which is also a Calabi-Yau threefold, obtained similarly as a Stein factorization (modulo some details on the singularities). Then we have that $$\calD(\P^2 \times \P^1, \calC_0)\simeq \calD(Y, \alpha)$$ by work of Kuznetsov, which allows us to conclude that $$\calD(\mathcal{X})\simeq \calD(Y,\alpha)$$


\subsection{Connecting the Hodge theoretic and derived viewpoints: the paper of Addington-Thomas}

So far we have seen a countable union of divisors in the moduli space of cubic fourfolds with a distinguished divisor $\calC_8$. Addington-Thomas show that all the other $\calC_i$ meet this nontrivially. The cubic fourfolds in these divisors are the conjecturally rational ones. 

- Reformulation of Hassett condition
- Deformation theory away from $\calC_8$









