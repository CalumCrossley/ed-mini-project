\subsection{A quick review of K3 surfaces}
\begin{definition}{K3 surface}{K3 surface}
A K3 surface is a compact complex surface with $\omega_{X}$ trivial and $H^1(X,\mathcal{O}_{X})=0$.
\end{definition}

Firstly, spaces derived equivalent to K3 surfaces are also K3 surfaces: their dimensions have to be the same because the Serre functors are isomorphic and hence the orders of the canonical bundles are the same. Similarly, by passage to cohomology, we know that $$\bigoplus_{p-q=i}H^{p,q}(X)\simeq \bigoplus_{p-q=i}H^{p,q}(Y)$$
Hence, since $h^{0,1}=h^{1,2}$ by Hodge symmetry and Serre duality, we see that both are zero.

Computations using Riemann-Roch show that $c_2=e(X)=24$. Then Poincare duality shows $b_2=22$. 

The Hodge decomposition tells us that $$H^2(X,\mathbb{C})=H^{2,0}\oplus H^{1,1}\oplus H^{0,2}$$But both $h^{2,0}=h^{0,2}=1$ and hence $h^{1,1}=20$. 

\begin{theorem}{Global Torelli}{Global Torelli}
Two K3 surfaces are isomorphic if and only if there is a Hodge isometry $H^2(X,\mathbb{C})\simeq H^2(Y,\mathbb{C})$.
\end{theorem}

A Hodge isometry respects the intersection pairing and the Hodge decomposition. 

%Because the canonical bundle of $\mathbb{P}^1$ is $\mathcal{O}(-2)$ and is also isomorphic to the normal bundle by adjunction, we see that $$C\cdot C=\int _{C}c_{1}(\mathcal{O}(-2)) =-2$$
%Such curves define a reflection on the cohomology: $$\begin{gathered}
 
%s_{\delta}:H^2(X,\mathbb{Z})\xrightarrow{}H^2(X,\mathbb{Z})\\
%\alpha\mapsto \alpha + \langle \alpha ,\delta \rangle \delta
%\end{gathered}$$
%Is this induced by a spherical twist? Yes, I think the one given by $\mathcal{O}_{C}$. Let's see why this is true:

%Firstly, $\mathcal{O}_{C}$ is spherical because $$\mathrm{Ext^1}(\mathcal{O}_{C},\mathcal{O}_{C})=H^0(\mathcal{Ext}^1)=H^0(\mathcal{N}_{C/X})=H^0(\mathcal{O}(-2))=0$$The other Ext groups are one-dimensional, by Serre duality. 


%Now, the Todd class of a K3 surface $X$ is given as $(1,0,2)$ so its square root is $(1,0,1)$. Hence, for a vector bundle, we see that $$v(\mathcal{E})=(\mathrm{rk}\,\mathcal{E}, c_{1}\mathcal{E}, \mathrm{rk}(\mathcal{E})+c_{1}(\mathcal{E})^2/2-c_{2}(\mathcal{E}))$$
%So for a rational sphere given by the zero locus of a line bundle $\mathcal{L}$, we see by the resolution $$\mathcal{L}^\lor\xrightarrow{}\mathcal{O}_{X}\xrightarrow{}\mathcal{O}_{C}$$ that $\mathrm{ch}(\mathcal{O}_{C})=\mathrm{ch}(\mathcal{L}^\lor)-\mathrm{ch(\mathcal{O}_{X})}=-c_{1}(\mathcal{L})$. Moreover, $v(\mathcal{L}^\lor)=(1,-\delta,0), v(\mathcal{O}_{X})=(1,0,0)$ and hence $$v(\mathcal{O}_{C})=(0,\delta,0)$$ and this recovers the reflection on $H^2(X,\mathbb{Z})$. 

%Wait it's apparently $\mathcal{O}_{C}(-1)$?

%Next: derived equivalence of K3 surfaces iff Hodge isometry of the Mukai Hodge structure.

%Also: moduli of sheaves

\subsection{The cubic fourfold}

Take a cubic fourfold $X$ in $\mathbb{P}^5$. This has canonical bundle $K_{X}=-3H$ so is a Fano of index 3. By using Kodaira vanishing, one can show that $$\mathcal{O}_{X}(-2H), \mathcal{O}_{X}(-H), \mathcal{O}_{X}$$form an exceptional collection. The orthogonal complement is given by $$\mathcal{A}_{X}=\{\mathcal{F}\in \mathbf{D}^b(X)|\,\mathrm{Ext}^\bullet(\mathcal{O}_{X}(-iH), \mathcal{F})=0\}$$This component, named after Kuznetsov, is very interesting, as it looks like the derived category of a K3 surface.

\begin{proposition}{Kuznetsov component is CY2}{Kuznetsov component is CY2}
The category $\mathcal{A}_{X}$ is a CY2 category with Hochschild homology the same as that of a K3 surface. Therefore, it is indecomposable and the semiorthogonal decomposition is maximal.
\end{proposition}

The first statement will be proved in the subsequent section. To prove the second statement about the Hochschild cohomology, we first need to understand the Hodge diamond.

\todo{HKR isomorphism}
\begin{theorem}{Hochschild-Kostant-Rosenberg}{HKR}
    $$HH^k(\mathcal{D}(X))=\bigoplus_{p+q=k} H^q(X, \Lambda^p T_X), HH_k(\mathcal{D}(X))=\bigoplus_{q-p=k} H^q(X, \Omega^p_X)=\bigoplus_{q-p=k} H^{p,q}(X)$$
    
\end{theorem}

\subsubsection{Hodge diamond}

Let us first think about the Hodge diamond of $X$. Embed $\mathbb{P}^5$ via the Veronese embedding in $\mathbb{P}V_{3,5+1}$, the space of degree 3 polynomials in 6 variables. Then, a hyperplane section of the image of $\mathbb{P}^5$ consists of a linear relation between the basis of monomials of this vector space, and hence is precisely a cubic fourfold $X$. The Lefschetz hyperplane theorem states that then $$H^\bullet(\mathbb{P}^5)\xrightarrow{}H^\bullet (X) \text{ iso for }\bullet<4$$
We thus see, by Poincare duality, that $b_{1}=b_{3}=b_{5}=b_{7}=0$. Similarly, $b_{2}=b_{6}=1$ and must be given by $h^{1,1}$ by Hodge theory. We have that $H^2(X;\mathbb{Z})=\mathbb{Z}$ and the LES from the exponential sequence tells us that $$H^1(X,\mathcal{O}_{X})\xrightarrow{}H^1(X,\mathcal{O}_{X}^*)\xrightarrow{}H^2(X,\mathbb{Z})\xrightarrow{}H^2(X,\mathcal{O}_{X})$$
But the first term is 0, since it is a summand of $H^1(X;\mathbb{C})=0$. The second term is $\mathrm{Pic}(X)$ and the map is $c_1$ which lands in $H^{1,1}$. Since $c_1(K_X)=-3$ it is nonzero, so since $\mathbb{C}=H^2(X;\mathbb{C})=H^{2,0}\oplus H^{1,1}\oplus H^{0,2}$ the only option is for $h^{2,0}=h^{0,2}=0$ and $\mathrm{Pic}(X)=\mathbb{Z}$ generated by a hyperplane. 

Now, the interesring bit is the fourth row, the middle dimensional cohomology. By using the normal bundle, we have that $$c(T_{X})(1+3x)=(1+x)^6=\iota^* c(T_{\mathbb{P}^5})$$We compute $c_1=3x, c_2=6x^2, c_3=2x^3, c_4=9x^4$. Then by Gauss-Bonnet, $$b_{4}+4=\chi(X)=\langle 9x^4, [X] \rangle= 27 $$since $pd(X)=3\omega$. We conclude that $b_4=23$. 

We also have that $H^0(X, \Omega^4)=H^0(X, \mathcal{O}_{X}(-3))=0$ so $h^{4,0}=h^{0,4}=0$. The only bit left is to determine $h^{1,3}=h^{3,1}$. 

Can try Hirzebruch-Riemann-Roch for the cotangent bundle of $X$, by computing $$\chi(\Omega_{X})=-h^{1,1}-h^{1,3}=\int _{X}\mathrm{ch(\Omega_{X})}\mathrm{td}(T_{X})=3\int _{\mathbb{P}^5} \omega \,\mathrm{ch}(\Omega_{\mathbb{P}^5}-\mathcal{O}(-3)) \mathrm{td}(T_{\mathbb{P}^5}-\mathcal{O}(3))  $$
Because I am lazy, I used the Macaulay 2 code: 

\begin{lstlisting}[language=Python]
loadPackage "Schubert2"
P5  = flagBundle({1,5}, VariableNames=>{s,q})
T = tangentBundle(P5)
coT = cotangentBundle(P5)
O1 = dual(P5.Bundles#0)
w = chern(1,O1)
NX = O1^**3
coNX = dual(NX)
TX = T - NX
coTX = coT - coNX
Q = 3 *  w*ch(coTX)*todd(TX)
print  integral  Q
\end{lstlisting}
This gives the answer -2, which implies $h^{1,3}=1$.

Can try to compute $h^{2,2}$ similarly: again by Hirzebruch-Riemann-Roch
$$h^{2,2}=\chi(\Omega_{X}^2)=\int_{X} \mathrm{ch}(\Omega^2_{X})\mathrm{td}(TX)  $$
The Macaulay code is: 

\begin{lstlisting}[language=Python] 
coT2X = exteriorPower_2 coTX
A = 3*w*ch(coT2X)* todd(TX)
print integral A
\end{lstlisting}
%can add caption by declaring caption = ...
This gives the correct answer $21$! All in all, the Hodge diamond looks like: 

\begin{center}
\begin{tikzcd}[row sep=tiny, column sep=tiny]
    &   &   &   & 1  &   &   &   &   \\
    &   &   & 0 &    & 0 &   &   &   \\
    &   & 0 &   & 1  &   & 0 &   &   \\
    & 0 &   & 0 &    & 0 &   & 0 &   \\
  0 &   & 1 &   & 21 &   & 1 &   & 0 \\
    & 0 &   & 0 &    & 0 &   & 0 &   \\
    &   & 0 &   & 1  &   & 0 &   &   \\
    &   &   & 0 &    & 0 &   &   &   \\
    &   &   &   & 1  &   &   &   &  
  \end{tikzcd}
\end{center}

We see that, by \ref{th:HKR}, the Hochschild homology is $HH_\bullet(X)=\mathbb{C}[-2]\oplus \mathbb{C}^{25}\oplus \mathbb{C}[2]$. On the other hand, Hochschild homology is additive with respect to semiorthogonal decomposition, and the three exceptional objects contribute to three copies of $\mathbb{C}$ in degree zero, so we see that $HH_\bullet(\mathcal{A}_X)=\mathbb{C}[-2]\oplus \mathbb{C}^{22}\oplus \mathbb{C}[2]$, which is exactly the Hochschild homology of a K3 surface.

These properties of the category $\mathcal{A}_X$ made people suspect that there is an honest, geometric K3 surface inside of $X$. All known cases of cubic fourfolds containing a K3 surface are birational to $\mathbb{P}^4$, which led Kuznetsov to conjecture:

\begin{conjecture}{}{Kuznetsov's conjecture}
A cubic fourfold is rational if and only if it contains a smooth projective K3 surface $S$ such that $\mathcal{D}(S)\simeq \mathcal{A}_X$.
\end{conjecture}

In a subsequent section, we will explore different known examples where this is true, and connect the derived categories viewpoint of Kuznetsov with the Hodge-theoretic perspective of Hasset through the paper of Addington-Thomas.

\subsection{The Kuznetsov component is a CY2 category}

We now embark on the proof that the Kuznetsov component is what is called a Calab-Yau 2 category. What this means is that it has a Serre functor which is just given by shifting by $2$, as would be the case if it were the derived category of a K3 surface.

Let us consider a general degree $d$ hypersurface $X \subset \mathbb{P}^{n+1}$. By adjunction, its canonical bundle is given by $\O(d-n-2)$ and hence its Serre functor is $\otimes \,\O(d-n-2)[n] $. Moreover, the line bundles $\O,...,\O(n+1-d)$ are all exceptional and have an orthogonal complement which describes $$\calD(X)=\langle \calA_X,\O,...,\O(n+1-d) \rangle $$

\subsubsection{The kernel of the Serre functor}

The left mutation with respect to the admissible subcategory spanned by the line bundles is $$\mL_{\langle \O, \dots, \O(n+1-d) \rangle}\simeq \mL_\O \circ \dots \circ \mL_{\O(n+1-d)}$$

Now we notice the following: the mutation $\mL_{\O(i)}$ fits into an exact triangle $$\Hom(\O(i), F)\otimes \O(i) \xrightarrow{ev} F \rightarrow \mL_{\O(i)} F$$
\todo{Give explanation why ev is the unit and why mutations compose}

On the other hand, we can think of the middle as $\Phi_{\O_\Delta}F$ and the left object as $\Phi_{\O(-i)\boxtimes \O(i)}F$. More precisely, by using the projection and base change formulas:
\begin{gather*}
    \Phi_{\O(-i)\boxtimes \O(i)}F=p_* (p^*\O(i)\otimes q^* \O(-i)q^*F)\simeq p_* q^* F(-i)\otimes \O(i)\simeq\\
    \simeq \O_X \otimes \RG (F(-i))\otimes \O(i)\simeq \RG \RlHom(\O(i),F) \otimes O(i)=\RHom(\O(i),F) \otimes \O(i)
\end{gather*}

Hence, we can conclude that $\mL_{\O(i)}$ is given by a Fourier-Mukai transform with kernel $$[\O(-i)\boxtimes \O(i)\rightarrow \O_\Delta]$$

Now, define $$\mathbf{O}:=\mL_\O \circ (-\otimes \O(1))$$
We see that by \ref{cor:Autoequivalences and mutations}, $$\mathbf{O}^{n+2-d}\simeq \mL_\O \circ \mL_{\O(1)}\circ ... \circ \mL_{\O(n+1-d)}\circ (-\otimes \O(n+2-d))=\mL_{\langle \O, \dots, \O(n+1-d) \rangle} \circ (-\otimes \omega ^{-1})$$

But by \ref{lemma:Serre functors of admissible subcategories}, we know that the inverse of the Serre functor of $\calA_X$ is given by $$\calS_{\calA_X}^{-1} =  \mL_{\langle \O, \dots, \O(n+1-d) \rangle} \circ \calS_X ^{-1}=\mL_{\langle \O, \dots, \O(n+1-d) \rangle} \circ (-\otimes \omega^{-1})[-n]$$

We can thus conclude that $$\calS_{\calA_X}^{-1} = \mathbf{O}^{n+2-d}\circ [-n]$$

If we put $T:=\mathbf{O}|_{\calA_X}$, then we can conclude that $$\calS_{\calA_X}=T^{d-n-2}\circ [n]$$

Recall that $\mL_\O$ had kernel given by $[\O \boxtimes \O \rightarrow \O_\Delta]$. We can compose this with the kernel for $-\otimes \O(1)$ which is $\O_\Delta(1)$ to see that the kernel for $T$ is given by $K_1;=[\O(1)\boxtimes \O \rightarrow \O_\Delta(1)]$. 

Our ultimate aim is to show that a suitable power of the Serre functor is just a shift functor, so we need to understand the kernel for $T^i, i=1,2,\dots, $. So we need to convolute $K_1$ with itself multiple times. 

\todo{Do we have compositions of FM kernels? Also, need to talk about preserving exactness}

Firstly, composing with a Fourier-Mukai transform preserves exact triangles, so the same holds for the kernels. We can compose the triangle for $K_1$ with $\O(1)\boxtimes \O, \O_\Delta(1)$ on the left and $K_1$ on the right respectively to get three exact triangles \begin{gather*}
    (\O(1)\boxtimes \O) \circ K_1 \rightarrow (\O(1)\boxtimes \O)  \circ (\O(1)\boxtimes \O) \rightarrow (\O(1)\boxtimes \O) \circ \O_\Delta(1)\\
    \O_\Delta(1)\circ K_1 \rightarrow \O_\Delta(1) \circ (\O(1)\boxtimes \O)  \rightarrow \O_\Delta(1) \circ \O_\Delta(1)\\
    K_1 \circ K_1  \rightarrow (\O(1)\boxtimes \O)\circ K_1 \rightarrow \O_\Delta(1)\circ K_1
\end{gather*}

We can compute some of these: for example, the middle convolution on the first row is given by $$(\O(1)\boxtimes \O)  \circ (\O(1)\boxtimes \O) ={\pi_{13}}_*\big(\O(1)\boxtimes \O \boxtimes \O \otimes \O \boxtimes \O(1) \boxtimes O\big)=\big(\O(1)\boxtimes \O\big) \otimes H^\bullet(\O(1))$$
as pushing down is the same as cohomology on the fibers. This is the only computation that involves any cohomology: the others are given by tensoring with the diagonal, which turns the first two triangles into: \begin{gather*}
    (\O(1)\boxtimes \O) \circ K_1 \rightarrow \big(\O(1)\boxtimes \O\big) \otimes H^\bullet(\O(1)) \rightarrow \O(1)\boxtimes \O(1)\\
    \O_\Delta(1)\circ K_1 \rightarrow \O(2)\boxtimes \O \rightarrow \O_\Delta(2)
\end{gather*}

Now recall the Euler sequence on $\mathbb{P}^{n+1}$, which says that there is an exact triangle $$\Omega(1)\rightarrow \O^{n+2}\rightarrow \O(1)$$
Since the cohomology of $O(1)$ is $n+2$-dimensional, we can read off from the first exact triangle that $ (\O(1)\boxtimes \O) \circ K_1\simeq \O(1) \otimes \Omega(1)$. We can now plug this into the second object in the last triangle, as well as replace the last object in the third triangle by the second triangle to get: $$K_1 \circ K_1 \rightarrow \O(1)\boxtimes \Omega(1)\rightarrow \O(2)\boxtimes \O \rightarrow \O_\Delta(2)$$

We show by induction that the kernel $K_1^i$ fits into a sequence $$K_1^i\rightarrow \O(1)\boxtimes \Omega^{i-1}(i-1)\rightarrow \dots \rightarrow \O(i-1)\boxtimes \Omega(1)\rightarrow \O(i)\boxtimes \O \rightarrow \O_\Delta(i)$$

Assume it holds for $i$. Then we can apply $-\circ K_1$. We notice by the same argument that $\O_\Delta(i)\circ K_1= [\O(i+1)\boxtimes \O \rightarrow \O_\Delta(i+1)]$ and also using cohomology on the fibers and the Euler sequence tha $\big( \O(i)\boxtimes O\big)\circ K_1=\O(i)\boxtimes \Omega^1(1)$. What we need to understand is the other parts $\big(\O(i-k)\boxtimes \Omega^{k}(k)\big) \circ K_1$. This can be done by applying $\big(\O(i-k)\boxtimes \Omega^k(k)\big)\circ -$ to the triangle defining $K_1$. What we get is the following: $$\big(\O(i-k)\boxtimes \Omega^k(k)\big)\circ K_1 \rightarrow \O(i-k)\boxtimes \O \otimes H^\bullet(\Omega^k(k+1))\rightarrow \O(i-k)\boxtimes \Omega^k(k+1)$$

Now we need to use a variant of the Euler sequence: $$\Omega^{k+1}(k+1)\rightarrow \O^{\binom{n+2}{k+1}}\rightarrow \Omega^k(k+1)$$
\todo{This I found in Arupura's book, Corollary 17.1.3}

This, together with a computation of the cohomology of $\Omega^k(k+1)$ having dimension $\binom{n+2}{k+1}$ (it has only $H^0$ using Bott vanishing etc.), tells us that $\big(\O(i-k)\boxtimes \Omega^{k}(k)\big) \circ K_1=\O(i-k)\boxtimes \Omega^{k+1}(k+1)$. This completes the induction.

\subsubsection{The main theorem}
Now we are ready to prove the main theorem:

\begin{theorem}{Shift functor}{}
    The functor $T$ has $T^d=[2]$.
\end{theorem}

\begin{proof}
    In the course of the proof, we write $\P=\mathbb{P}^{n+1}$. Firstly, let's recall the Koszul resolution of $X$ in $\mathbb{P}^{n+1}$ and $X\times X\subset \P\times \P$. Since $X$ is given by a section of $\O(d)$ and $X\times X$ by a section of $\O(d) \boxtimes O \oplus \O \boxtimes \O(d)$, we have the following resolutions: \begin{gather*}
        0\rightarrow \O(-d)\rightarrow \O_\P \rightarrow \iota_* \O_X\rightarrow 0\\
        0\rightarrow \O(-d,-d)\rightarrow \O(-d,0)\oplus \O(0,-d)\rightarrow \O_{\P\times \P}\rightarrow (\iota\times \iota)_* \O_{X\times X}\rightarrow 0
    \end{gather*}

    We wish to understand the derived pullback $(\iota\times \iota)^* \O_{\Delta_\P} $. Instead, let us first consider $(\iota\times \iota)^*(\iota\times \iota)^* \O_{\Delta_\P} \simeq (\iota\times \iota)_* \O_{X\times X} \otimes \O_{\Delta_\P}$, by the projection formula. This derived tensor product is given by tensoring the Koszul resolution above with the diagonal. The resulting complex is $$\O_{\Delta_\P}(-2d)\rightarrow \O_{\Delta_\P}(-d)^{\oplus 2}\rightarrow \O_{\Delta_\P}$$
    The first map is injective, and this is just the sum of the two resolutions $\O_{\Delta_\P}(-d)\rightarrow \O_{\Delta_\P}\rightarrow \O_{\Delta_X}$ and $\O_{\Delta_\P}(-2d)\rightarrow \O_{\Delta_\P}(-d)\rightarrow \O_{\Delta_X}(-d)$. We conclude that $L_{-1}=\calH^{-1}=\O_{\Delta_X}(-d), L_0=\calH^{0}=\O_{\Delta_X}$. Now we note that $\iota$ is a closed embedding, hence exact and conservative, so we can ignore it.

    This now allows us to write down an exact triangle $$\O_{\Delta_X}(-d)[1]\simeq \calH^{-1}[1]\rightarrow (\iota\times \iota)^*\O_{\Delta_\P}\rightarrow \calH^0\simeq \O_{\Delta_X}$$
    We can rotate and twist by $d$ to get the exact triangle 
    \begin{equation}\label{eqn:derivedtriangle}
        (\iota\times \iota)^*\O_{\Delta_\P}\rightarrow \O_{\Delta_X}(d)\rightarrow \O_{\Delta_X}[2]
    \end{equation}

    Now recall the Beilinson resolution of the diagonal, which we restrict to $X\times X$:\begin{gather*}
        0 \rightarrow \O (d-n-1)\boxtimes \Omega^{n+1}(n+1)\rightarrow \dots \rightarrow \O \boxtimes \Omega^{d}(d)\\
        \rightarrow \O(1)\boxtimes \Omega^{d-1}(d-1)\rightarrow \dots \rightarrow \O(d)\boxtimes \O \rightarrow \O_\Delta
    \end{gather*}

    We splice it into the top bit, where all the $\O$'s are non-positive, and the positive bit below, which is just the kernel $K_d$. Let us call $K'_d$ the complex which is $K_d$ without the diagonal. Then we have an exact triangle $$K'_d \rightarrow \O_{\Delta_X}(d)\rightarrow K_d$$

    We now compare this with triangle \ref{eqn:derivedtriangle}, by using the natural map $K'_d \rightarrow (\iota\times \iota)^*\O_{\Delta_\P}$ coming from Beilinson's resolution, and the identity map in the middle, which can be extended to a map $K_d \rightarrow \O_{\Delta_X}[2]$ by the axioms of triangulated categories: 

    \begin{center}\begin{tikzcd}
        K'_d \arrow[d] \arrow[r]                      & \O_{\Delta_X}(d) \arrow[d] \arrow[r] & K_d \arrow[d, dashed] \\
        (\iota\times \iota)^*\O_{\Delta_\P} \arrow[r] & \O_{\Delta_X}(d) \arrow[r]           & {\O_{\Delta_X}[2]}   
        \end{tikzcd}\end{center}

    Finally, we compare the effect of taking these as Fourier-Mukai kernels. We know that $\Phi_{K_d}=T^d$ and $\Phi_{\O_\Delta [2]}=[2]$ and we claim that they are isomorphic by the dotted arrow. To see this, we show that the other vertical arrows induce isomorphisms on Fourier-Mukai transforms.

    The middle bit is obvious, however to compare $\Phi_{K'_d}$ and $\Phi_{(\iota\times \iota)^*\O_{\Delta_\P}}$ we simply look again into the Beilinson resolution: we have that \begin{gather*}
        [ \O (d-n-1)\boxtimes \Omega^{n+1}(n+1)\rightarrow \dots \rightarrow \O \boxtimes \Omega^{d}(d)\rightarrow K'_d]=(\iota\times \iota)^*\O_{\Delta_\P}
    \end{gather*}

    However, for any $\calE_i=\O(d-i)\boxtimes \Omega^i(i)$ with $i=d,d+1,\dots,n+1,$ its Fourier-Mukai transform on $A\in \calA_X$ vanishes, by using the projection formula, base change and the fact that $\calA_X$ is orthogonal to $\O,\dots, \O(n+1-d)$:$$\Phi_{\calE_i}(A)=q_*(p^*A\otimes p^* \O(d-i)\otimes q^* \Omega^i(i))=\Omega^i(i)\otimes q_*p^* A(d-i)=\Omega^i(i) \otimes \RHom(\O(i-d),A)=0$$ We conclude that $\Phi_{K'_d}\simeq \Phi_{(\iota\times \iota)^*\O_{\Delta_\P}}$ and hence $T^d=[2]$.
\end{proof}

\begin{corollary}{Kuznetsov}{}
    If $2d>n+1$, then $\calA_X$ is a fractional Calabi-Yau category whose Serre functor obeys $\calS_{\calA_X}^{d/c}=[(n+2)(d-2)/c]$, where $c=\gcd(d,n+2)$.\newline


    In the special case that $X$ is a cubic fourfold, $n=4,d=3$ hence $\calS=T^{-3}\circ [4]=[2]$. More generally, whenever $d|n+2$, the component is Calabi-Yau.
\end{corollary}





\subsection{Examples: Pfaffian cubics, cubics containing a plane, nodal cubics}

\subsection{Addington-Thomas}









