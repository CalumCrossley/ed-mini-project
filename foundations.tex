When one first encounters homological algebra, for example in algebraic topology, a big emphasis is put on the homology groups of a given space. However, it turns out that these groups are not a complete invariant: there are spaces $X,Y$ such that $H_\bullet(X) \simeq H_\bullet(Y)$, but $X$ is not homotopically equivalent to $Y$.

To remedy this, one needs a refinement of structure: instead of only considering the homology groups themselves, we can remember the whole singular chain complex $C_\bullet(X)$. Then, a theorem of Whitehead reassures us that two CW complexes with quasiisomorphic chain complexes are in fact homotopy equivalent! By enhancing homology groups to a chain level structure, homotopy equivalence becomes the same as quasiisomorphism of chain complexes. If we want to make quasiisomorphisms into an equivalence relation, however, the right object of study is the derived category.

In this section, we cover the basic construction of the derived category and in particular many important properties of $D^b(X)$, the derived category of a variety $X$.

\subsection{The derived category of an abelian category}

Suppose we have an abelian category $\calA$: the main example we will be considering is the coherent sheaves on a projective variety $\mathrm{Coh}(X)$. We would like to study $\mathrm{Ch}(\calA)$, which is also an abelian category, but invert the class of quasiisomorphisms. The process of inverting is called \emph{localization by a localizing class.}
\subsubsection{Localization}

Suppose we have a class $S$ of morphisms in an abelian category $\calC$ that we would like to invert. We have to be able to cancel them on the left and right for this construction to even make sense: in other words, we should always be able to replace $f:X\to Y, s:Z\to Y$ by another morphism $s':T\to X$ such that the following diagram commutes:

\begin{equation*}
    \begin{CD}
        T @>g>> Z \\
          @Vs'VV @VVsV \\
        X @>>f> Y
    \end{CD}
\end{equation*}

A class of morphisms is called a \emph{localizing class} if it satisfies this property.

\begin{definition}{Localization}{Localization}
    Suppose $\calC$ has a localizing class $S$. Then $\calC[S^{-1}]$ is the category with objects the same as that of $\calC$ and morphisms given by equivalence classes of roofs which we think of as $s^{-1}f$. Two roofs are the same if there is a bigger roof restricting to them (one can think of this as a larger numerator, or lcm)
\end{definition}

Composition is defined by choosing some roof as follows:

\[\begin{tikzcd}
    &                                    & F \arrow[ld, "s''"'] \arrow[rd, "h"] &                                     &   \\
    & D \arrow[ld, "s"'] \arrow[rd, "f"] &                                      & E \arrow[ld, "s'"'] \arrow[rd, "g"] &   \\
A \arrow[rr, "f/s", dotted] \arrow[rrrr, "g/s' \circ f/s=g\circ h/ s'\circ s''"', dotted, bend right] &                                    & B \arrow[rr, "g/s'", dotted]         &                                     & C
\end{tikzcd}\]
Under these constructions, morphisms in $S$ become invertible.

We run into the following problem: the quasiisomorphisms that we want to invert are not a localizing class in $\mathrm{Ch}(\calA)$! The way to get around this issue is to introduce the homotopy category, where they do indeed localize:

\begin{theorem}{Quasiisomorphisms localize in the homotopy category}{Quis form a localizing class}

    Quasiisomosphisms form a localizing class in the homotopy category of an abelian category $K(\calA)$\footnote[1]{This is the category where we mod out by chain homotopy equivalence, see e.g. \cite{Weibel}}. We denote the derived category by
    \begin{equation*}
        D(\calA)\coloneqq K(\calA)[Q^{-1}]
    \end{equation*}
\end{theorem}

%This is an additive category. Moreover, the cohomology functor factors through this.

Before embarking on the proof of this, let us recall the following construction from homological algebra:

\begin{definition}{Mapping cone}{Mapping cone}
    The mapping cone of $f:A^\bullet\to B^\bullet$ is the complex $A^\bullet[1]\oplus B^\bullet$ with differential $\begin{pmatrix} -d_A & 0 \\ f & d_B \end{pmatrix}$.
\end{definition}

The three objects $A^\bullet, B^\bullet, \mathrm{cone}(f)$ fit into a long exact sequence of cohomology groups. On the level of the homotopy category, we would like to put the boundary map on the same footing as the other ones, and form what is called a \emph{distinguished triangle}: \[\begin{tikzcd}
    & B^\bullet \arrow[rd] &                                       \\
A^\bullet \arrow[ru, "f"] &                      & \mathrm{cone}(f) \arrow[ll, "{[-1]}"]
\end{tikzcd}\]

The $[-1]$ operator shifts a chain complex by $1$. We see that $f$ is a quasiisomorphism iff $\cone(f)=0$.

We will say that a triangle is \emph{distinguished} if it is equivalent to one as above. We need to show that these triangles are well-defined and can be rotated, essentially showing that the homotopy category is a triangulated category.

\begin{proposition}{}{cones}
    $\cone(B\xrightarrow\tau\cone(f))\simeq A$ in the homotopy category.
\end{proposition}

\begin{proof}
    One needs to show that the two morphisms described below are homotopy inverses, which can be done by finding explicit chain homotopies. For more details, consult \cite[\S4]{gelfand_methods_2003}
\end{proof}

%$$\begin{CD}A_{i+1}@> \begin{pmatrix}
%-f \\ 1 \\ 0
%\end{pmatrix}> > B_{{i+1}}\oplus A_{{i+1}}\oplus B_{i} @> > > A_{{i+1}} \\ @V -d_{A} VV @VV\begin{pmatrix}
%-d_{B}& 0 & 0 \\ 0 & -d_{A} & 0 \\1 & f & d_{B}
%\end{pmatrix}V @VV-d_{A}V \\ A_{i+2} @> > > B_{i+2}\oplus A_{i+2}\oplus B_{i+1}@> > > A_{i+2}\end{CD}$$

From this proposition, we see that the following diagram commutes:
\begin{equation*}
    \begin{CD}
        B @>>> \cone(f) @>>> A[-1] \\
          @V=VV @V=VV @VVV \\
        B @>>> \cone(f) @>>> \cone(\tau)
    \end{CD}
\end{equation*}

Now we can complete the proof that quasiisomorphisms form a localizing class in the homotopy category. The benefit of this is that we can replace $A\to B\to\cone(f)$ with $B\to\cone(f)\to\cone(\tau)$ and compose $\tau$ with another morphism.
\begin{proof}[Proof of Theorem 1.2]
    Recall that we wanted to show quasiisomorphisms form a localizing class, so we'd like to be able to factorize in two ways: suppose $f$ is a quasiisomorphism and $g$ is any morphism. We'd like to find $?$ such that $?\to C$ is a quasiisomorphism and the diagram commutes.
    \begin{equation*}
        \begin{CD}
            ? @> \qi>> C \\
            @VVV @VVgV \\
            A @>>f> B
        \end{CD}
    \end{equation*}
    This comes down to the following diagram:
    \begin{equation*}
        \begin{CD}
            \cone(\tau g)[-1] @>\qi>> C @>>> \cone(f) @>>> \cone(\tau g) \\
              @VVV @VgVV @V=VV @VVV \\
            \cone(\tau)[-1] @>>> B @>\tau>> \cone(f) @>>> \cone(\tau) \\
              @V\qi VV @V=VV @V=VV @VVV \\
            A @>>f> B @>>> \cone(f) @>>> A[1]
        \end{CD}
    \end{equation*}
    Since $f$ is a quasiisomorphism, we see that $\cone(f)=0$ and hence $\cone(\tau g)$ must be quasiisomorphic to $C[1]$. So we put $?=\cone(\tau g)[-1]$.
\end{proof}

\subsubsection{The quotient viewpoint}

It is natural to ask which objects in $K(\calA)$ become zero in $D(\calA)$. Such
a complex must be acyclic, since cohomology lifts to $D(\calA)$, and then the
map from the zero complex is a quasi-isomorphism so this suffices. In fact one
can view $D(\calA)$ as the quotient of $K(\calA)$ by the subcategory of acyclic
complexes.

\begin{definition}[label=defn:verdierquotient]{}{}
    If $\calC$ is a triangulated category, and $\calD$ is a triangulated
    subcategory, then the morphisms in $\calC$ whose cones can be taken to be
    objects of $\calD$ form a localizing class $\Sigma$. We then define the
    quotient $\calC/\calD\coloneqq\calC[\Sigma^{-1}]$.
\end{definition}

\begin{remark}{}{}
    In particular $0\to d$ becomes an isomorphism for any $d\in\calD$, since the
    cone is $d$, and so $\calD$ maps to zero in $\calC/\calD$. In fact
    $\calC/\calD$ is the universal triangulated category with this property.
\end{remark}

\subsubsection{Derived functors}

When we have either enough projectives or injectives, then we can restrict to the full subcategory of such objects (see \cite[Proposition~2.40]{Huybrechts}):

\begin{theorem}[label=thm:DfromK]{}{}
    Assume $\calA$ has enough projectives. Then
    \begin{equation*}
        K^-(\calP) \simeq D^-(\calA).
    \end{equation*}
    Dually, if there are enough injectives,
    \begin{equation*}
        K^+(\calI) \simeq D^+(\calA).
    \end{equation*}
\end{theorem}

In fact, this shows that the injective complexes provide a \emph{dg-enhancement} of the derived category. Classically, to compute sheaf cohomology one replaces the sheaf by an injective resolution and then takes homology. In the derived category, we don't take homology but use the above equivalence to remember the whole resolution. In fact, any two resolutions are quasiisomorphic, so this object is well-defined.

%Suppose $F$ is an additive functor between abelian categories $\calA$, $\calB$. Then it induces a functor on the homotopy categories, by applying it termwise. If it is exact, then it sends quasiisomorphissms to quasiisomorphisms and commutes with kernels, cokernels etc.

\begin{definition}{Right derived functor}{Right derived functors}
    Suppose that $F:\calA \rightarrow \calB$ is a left exact functor between abelian categories, e.g. the pushforward functor. Then we can define $\dR F$, the right derived functor of $F$, by choosing an inverse to the equivalence on the left:
\begin{equation*}
    \begin{CD}
        D^+(\calA) @>>> D^+(\calB) \\
          @AAA @AAA \\
        K^+(\calI_A) @>F>> K^+(\calB)
    \end{CD}
\end{equation*}
Explicitly, we choose an injective resolution $A^\bullet\to I^\bullet$ and
define $\dR F(A^\bullet)\coloneqq F(I^\bullet)$. The $n$-th homology of this complex is defined to be $R^nF(A^\bullet)$.
\end{definition}

Dually, one can also define left derived functors for right exact functors by taking projective resolutions.

\paragraph{Derived functors in geometry}

Now we specialize to $D^b(X)=D^b(\Coh(X))$. There are a plethora of geometric functors:

\begin{itemize}
    \item $\Gamma:\QCoh(X)\to\Ab$
    \item $\Hom:\QCoh(X)\times\QCoh(X)\to\Ab$
    \item Given $f:X\to Y$, have $f_*:\QCoh(X)\to\QCoh(Y)$. For $Y=\pt=\Spec\C$, pushforward is the same as taking global sections of the sheaf.
    \item $\lHom:\QCoh(X)\times\QCoh(X)\to\QCoh(X)$. Notice $\Gamma\circ\lHom=\Hom$. These are left exact.
    \item $\otimes$, $f^*$. These are right exact. Note that $f^{-1}$ is exact, but to get $f^*$ one tensors with the structure sheaf.
\end{itemize}

$\QCoh(X)$ has enough injectives, but not a lot of projectives. So we can define right derived functors. We would like to say that, since $g_*\circ f_*=(g\circ f)_*$ then the same holds for their derived versions. We run into the following issue: the pushforward does not respect injectives! We will introduce another class of sheaves, the flasque ones, to make this work. Now, pushforwards respect flasque sheaves! This is an example of what is called an adapted class:

\begin{definition}{Acyclic objects and adapted classes}{Adapted classes}
     An object is called $F$-acyclic if $R^iF(A)=0$ for $i>0$. A class of objects is adapted to $F$ if it is closed under $\oplus$, all objects in the class are $F$-acyclic and every object in $\calA$ can be embedded in an object of the class.
\end{definition}

Notice that one can throw in all $F$-acyclic objects in the adapted class, but that would cause a problem if we want the inclusion in the following theorem to hold.

\begin{theorem}{Composition of derived functors}{Composition of derived functors}
    Suppose $F:\calA\to\calB$, $G:\calB\to\calC$ are two functors such that $F(R_\calA)\subset R_\calB$, where these are adapted classes for $F$ and $G$ respectively. Then $\dR G\circ\dR F = \dR(G\circ F)$.
\end{theorem}

Since pushforwards send flasques to flasques, and flasques are an adapted class, we get what we originally wanted, i.e. $$\dR g_* \circ \dR f_* = \dR (g\circ f)_*$$

%Similarly, we can ask what happens to the pullback functor. Locally frees i.e. vector bundles should be an adapted class for $f^*$. But we don't have enough projectives, so the last condition does not make sense. We need an equivalent one: if $E^\bullet$ is acyclic complex of locally frees, we require that $f^*E^\bullet$ is still acyclic.

\begin{remark}{}{}
    Dually, one can define adapted classes for left-derived functors. The last condition is then that every object admits a surjection from an object of the class.
\end{remark}

\subsubsection{Morphisms in the derived category and derived Hom}

The morphisms between two chain complexes in $\mathrm{Ch}(\calA)$ tend to be quite big: in fact, there is an enhanced Hom-chain complex:

\begin{gather*}
    \Hom(A^\bullet, -):\mathrm{Ch}(\calA)\rightarrow \mathrm{Ch}(\mathrm{Ab})\\
    \Hom^n(A^\bullet,B^\bullet) = \prod\Hom(A^i, B^{i+n})
\end{gather*}
with differential $df=d_Bf-(-1)^nfd_A$. We see that $Z^0$ is just chain maps, whereas $B^0$ is chain homotopies, so $H^0$ is $\Hom_{K(\calA)}(A,B)$, i.e. morphisms up to homotopy. More generally,
$$\begin{gathered}
    Z^i \Hom^\bullet(A^\bullet, B^\bullet)= \Hom_{\Ch(\calA)}(A^\bullet, B^\bullet[i])\\
    B^i \Hom^\bullet(A^\bullet, B^\bullet) = \text{morphisms homotopic to zero}\\
    H^i \Hom^\bullet(A^\bullet, B^\bullet)=\Hom_{K(\calA)}(A^\bullet, B^\bullet[i])
\end{gathered}$$

\begin{remark}{}{}
    This chain complex structure on $\Hom$'s makes $\Ch(\calA)$ into a
    \emph{differential graded category}, or \emph{dg-category}, giving a
    \emph{dg-enhancement} of $K(\calA)$.
\end{remark}

If we want to compute $\RHom(A^\bullet,B^\bullet)$, we replace $B$ by a complex of injectives $I^\bullet$ (which can always be done by a method called the Cartan-Eilenberg resolution) and then compute $\Hom^\bullet(A,I)$.

\begin{proposition}{Morphisms in the derived category}{}
    $$\Ext^i(A^\bullet, B^\bullet):=H^i(\RHom^\bullet(A^\bullet,B^\bullet))\simeq \Hom_{D(\calA)}(A^\bullet, B^\bullet[i])$$
\end{proposition}

\begin{proof}
    Essentially, have the sequence of isomorphisms $$\Hom_{D(\calA)}(A^\bullet, I^\bullet[i])\simeq \Hom_{K(\calA)}(A^\bullet, I^\bullet[i])\simeq H^i \Hom^\bullet(A^\bullet, I^\bullet)\simeq \Ext^i(A^\bullet, B^\bullet)$$
    We demonstrate this explicitly for $A^\bullet$ a single object concentrated in degree zero. Take an injective resolution $B^\bullet \to I^\bullet$. We see that
\begin{equation*}
    \Hom_{D(\calA)}(A,B^\bullet[i]) \simeq \Hom_{K(\calA)}(A,I^\bullet[i]).
\end{equation*}
But this is given by a morphism $f:A\to I^i$ which should be a chain map:
\begin{equation*}
    \begin{CD}
        0 @>>> A @>>> 0 \\
          @VVV @VfVV @VVV \\
        I^{i-1} @>>> I^i @>>> I^{i+1}
    \end{CD}
\end{equation*}
Hence, we can think of $f$ as a map in $\Hom^i(A, I^\bullet)$ which is closed. Moreover, it is exact precisely when it factors through $I^{i-1}$ which is the same as saying it is nullhomotopic. This shows that indeed $\Ext^i(A,B)\simeq\Hom_{D(\calA)}(A,B[i])$.
\end{proof}


%This construction is consistent with shifting by $n$. So
%\begin{gather*}
%    \Hom(A^\bullet, -):\mathrm{Ch}(\calA)\rightarrow \mathrm{Ch}(\mathrm{Ab})\\
%    \calH^n(\Hom^\bullet(A,B))
%        = \calH^0(\Hom^\bullet(A,B[n])
%        = \Hom_{K(\calA)}(A,B[n])
%\end{gather*}
%
%As a corollary, we see that
%\begin{equation*}
%    \calH^n\RHom(A,B) \simeq \Hom_{D(\calA)}(A,B[n])
%\end{equation*}
%just as we had for objects before, but now for arbitrary complexes of e.g. sheaves.



%In contrast, we show that if we only look up to homotopy, we get finite-dimensional vector spaces, the Ext groups:
%\begin{equation*}
%    \Ext_\calA^i(A,B) = \Hom_{D(\calA)}(A,B[i]).
%\end{equation*}
%To compute Ext, we need to replace $A$ by a projective resolution $P^\bullet\to A$, then Hom into $B$ and take cohomology. Equivalently, can resolve $B$ by injectives $B\to I^\bullet$, Hom from $A$ and take cohomology.

Now we focus on a geometric example. Take $E$ a quasicoherent sheaf. Then local Homs form a left exact functor $\lHom(E,-)$ and we can form its right derived functor
\begin{equation*}
    \RlHom(E,-):D^+(\QCoh(X))\to D^+(\QCoh(X))
\end{equation*}

Crucially, projective varieties have enough vector bundles to resolve any sheaf:

\begin{proposition}{}{}
    If $X$ is projective, then $\Coh(X)$ has enough locally frees.
\end{proposition}

The reason is that given $F$, there is $n\gg0$ such that $F(n)$ is generated by global sections. These are given by maps $\O_X\to F(n)$, and since $F$ is finite type we get a surjection $\O_X^N\to F(n)$. Can then untwist. In this case, every bounded above complex of coherent sheaves is quasiisomorphic to a bounded above complex of vector bundles. If $X$ is regular, can just take bounded. Need Hilbert syzygy theorem.

This means we have an adapted class of locally frees, allowing us to similarly
define $\RlHom(-,F)$.

Now, $\lHom(E,-)$ takes injectives to flasques. From this, using \ref{th:Composition of derived functors} we see that
\begin{equation*}
    \RHom(E,-)=\RG(\RlHom(E,-))
\end{equation*}

%What about $\RlHom(E^\bullet, F^\bullet)$? Note that $\lHom(E^\bullet,-)$ is not a functor between $\QCoh$, but between homotopy categories. Need lemma:

%\begin{lemma}{}{}
%    If $E^\bullet$ or $F^\bullet$ is acyclic and $F^\bullet$ is a complex of injectives, then $\lHom(E^\bullet, F^\bullet)$ is acyclic.
%\end{lemma}

\subsubsection{Compatibilities between derived functors}

Recall: to derive $\lHom(\calF, -)$ we can resolve the second term by injectives, whereas to derive $\lHom(-,\calG)$ we can resolve the first term by vector bundles. Similarly for complexes of sheaves. For the derived tensor product $\calF^\bullet\Lotimes\calG^\bullet$ we have the same story, except need only use locally frees on either side. There are nice properties which can be checked on locally frees, and then taking complexes of locally frees e.g. associativity: $$(E^\bullet\Lotimes F^\bullet)\Lotimes G^\bullet \simeq E^\bullet\Lotimes(F^\bullet\Lotimes G)$$

 Now consider the left derived functor $\dL f^*:D^-(\Coh Y)\to D^-(\QCoh X)$. Since we have an adapted class of locally frees and pullbacks of vector bundles are vector bundles, we see that
\begin{equation*}
    \dL g^*\circ\dL f^* \simeq \dL(g\circ f)^*
\end{equation*}
%*Exercise:* blow up origin at plane, derive pull back structure sheaf at origin.

We have all sorts of other compatibilities:
\begin{gather*}
    \dL f^*(\calE^\bullet\Lotimes\calF^\bullet)
        \simeq \dL f^*\calE^\bullet\Lotimes\dL f^*\calF^\bullet \\
    \RHom(\calE^\bullet,\RlHom(\calF^\bullet,\calG^\bullet))
        \simeq \RlHom(\calE^\bullet\Lotimes\calF,\calG^\bullet) \\
    \RlHom(\calE^\bullet,\calF^\bullet)\Lotimes\calG^\bullet
        \simeq \RlHom(\calE^\bullet,\calF\Lotimes\calG^\bullet)\\
    \calE^\vee\Lotimes\calG^\bullet
        \simeq \RlHom(\calE^\bullet,\calG^\bullet), \quad
    \calE^\vee
        \simeq \RlHom(\calE,\O_X)
\end{gather*}

Also have an adjunction projection formula as usual:
\begin{gather*}
    \dR f_*\RlHom_X(\dL f^*\calE,\calF)
        \simeq \RlHom_Y(\calE,\dR f_*\calF) \\
    \RHom_X(\dL f^*\calE,\calF)
        \simeq \RHom_Y(\calE,\dR f_*\calF)
\end{gather*}
Works if $\calE$ is injective, $\calF$ locally free. The second version comes from the first one by taking global sections. Note that $\Gamma(f_*\calG)=\Gamma(\calG)$ and the same holds upon applying derived functors.

The adjunction formula is a version of the projection formula.
\begin{equation*}
    \dR f_*(\calE\Lotimes\dL f^*\calF)
        \simeq (\dR f_*\calE)\Lotimes\calF
\end{equation*}
The projection formula, as well as the other ones, usually requires that $\calF$ be locally free, but in the derived setting it works for everything, since we resolve by locally frees!

We finish off by citing the following incredibly useful fact:

\begin{theorem}{Cohomology and base change}{Cohomology and base change}
    Suppose we have a flat morphism $g:Y\to Z$ and a pullback square
    \begin{equation*}
        \begin{CD}
            W @>\tilde f>> Y \\
              @V\tilde gVV @VVgV \\
            X @>>f> Z
        \end{CD}
    \end{equation*}
    Then
    \begin{equation*}
        \dL g^*\circ\dR f_*=\dR\tilde f_*\circ\tilde \dL g^*.
    \end{equation*}
\end{theorem}
%Note that $g$ and hence $\tilde g$ being flat means the pullbacks are exact and
%don't need to be derived.
%The reason it is true is because it works in the usual case for injective sheaves and flat morphisms.

\begin{example}{}{}
When $Z=\Spec\C$, $W=X\times Y$ then we must have that
\begin{equation*}
    \dR p_*q^*\calE \simeq g^*\dR f_*\calE
\end{equation*}
where both $f$ and $g$ map to a point. But $\dR f_*=\RG$ in this case. The derived category of $\Spec\C$ is the category of graded vector spaces, where every object can be replaced by its cohomology with zero differential. Hence, the thing on the right is just
\begin{equation*}
    \O_Y\otimes_\C\RG(\calE)
\end{equation*}
where $\RG(\calE)$ is a graded vector space.

\end{example}

We can apply this to a classical example, the Poincare bundle on the elliptic curve:

\begin{example}{Poincare bundle}{}
    When $X$ is an elliptic curve, then if $\calE=\O_X$ concentrated in degree zero, we have
\begin{equation*}
    \RG(\O_X)=H^\bullet(X,\O_X) = \C\xrightarrow{0}\C
\end{equation*}
Then $g^*\dR f_*\calE = \O_Y\xrightarrow{0}\O_Y = \O_Y\oplus\O_Y[1]$.

Now, let $E$ be an elliptic curve and fix $P_0\in E$. Then every degree zero line bundle is given by $\O_E(P-P_0)$. Hence
\begin{equation*}
    \Pic^0(E)\simeq E.
\end{equation*}
Can make universal bundle over $\calL\to\Pic^0(E)\times E$ given by $\calL\coloneqq\O_{E\times E}(\Delta-E\times P_0)$. We see that
\begin{equation*}
    \calL|_{P\times E}\simeq \O_E(P-P_0).
\end{equation*}
Now, note that
\begin{equation*}
    H^\bullet(E,\O_E(P-P_0))
        = \begin{cases}
            0, & P\neq P_{0} \\
            \C\oplus\C[1] & P=P_{0}
        \end{cases}
\end{equation*}
which is the skyscraper sheaf computation we did before.
Have the following diagram, to which we want to apply cohomology and base change:
\begin{equation*}
    \begin{CD}
        P\times E @>>> E\times E \\
          @VVV @VV\pi V \\
        P @>>> E
    \end{CD}
\end{equation*}
We have just seen, since pushforward to a point is global sections, that
\begin{equation*}
    \dR\pi_*\iota^*\calL = H^\bullet(E,\O_E(P-P_0))
\end{equation*}
From the cohomology and base change formula, this must be equal to $\dL\iota^*\dR\pi_*\calL$. We appeal to the fact that $\dR\pi_*\calL=\O_{P_0}[-1]$. Moreover, the ideal sheaf sequence is in particular a Koszul resolution:
\begin{equation*}
    0\to\O_E(-P_0)\to\O_E\to\O_{P_0}\to 0
\end{equation*}
This is a resolution of locally free sheaves allowing us to compute the derived restriction pullback of $\O_{P_0}$ which is exactly $\C\xrightarrow{0}\C$ if $P=P_0$ and $0$ if not equal, as expected.

% Question: why is $\dR\pi_{*}\calL=\O_{P_{0}}[-1]$? Need to ask Ed.

\end{example}

\begin{remark}{Fourier-Mukai transforms}{}
    What we have just seen is an example of a Fourier-Mukai kernel: the universal line bundle $\calL$ on $E\times E$, which defines an endomorphism $\Phi_\calL:D^b(E)\to D^b(E)$, $\calF\mapsto\dR p_*(q^*\calF\Lotimes\calL)$. In fact, its adjoint is given by the Fourier-Mukai transform with kernel $\calL^\vee [1]$, as we will see later in the section on Fourier-Mukai transforms. Now, we know that $\dR \pi_* \calL = \Phi_\calL \O$. Hence, by adjointness we see that $$\Hom(\dR \pi_* \calL, \O_{P}[-1])\simeq \Hom(\O, \Phi_{\calL^\vee}\O_{P})\simeq H^\bullet(E, \O(P-P_0))$$This allows us to conclude that $\dR \pi_* \calL \simeq \O_{P_0}[-1]$ as was mentioned before.
\end{remark}

From now on we will drop the $\dR$'s and $\dL$'s unless needed for clarity.

\subsubsection{Computing derived functors: the Grothendieck spectral sequence}

Suppose we know the approximations $R^iF(A^p)$ or $R^iF(\calH^pA)$ for a complex $A^\bullet$. How can we compute $R^iF(A)$ from this information?

We first need the Cartan-Eilenberg complex $I^{\bullet, \bullet}$, which is a special bicomplex resolution of $A^\bullet$.

One can check that $\Tot I \to A$ is a quasi-isomorphism. So $\dR F(A)=F(\Tot I)=\Tot FI$. When we take cohomology, we are interested in the $n$-th cohomology of the total complex, for which we have two spectral sequences associated to the bicomplex: one starts with horizontal differential, and the other with the vertical.

If we start with vertical, then the first page is
\begin{equation*}
    E_1^{p,q} = R^qF(A^p) \Rightarrow R^nF(A)
\end{equation*}
If we start with horizontal, then
\begin{equation*}
    E_2^{p,q} = R^qF(\calH^pA) \Rightarrow R^nF(A)
\end{equation*}
More generally, there is a computational tool called the \emph{Grothendieck spectral sequence} which computes the derived functors of a composition: $$E_2^{pq}=R^pG \circ R^q F\implies R^{p+q}G\circ F$$

Now we describe two very important instance of the Grothendieck spectral sequence, the Leray spectral sequence and the local to global spectral sequence.

\begin{example}{Leray spectral sequence}{} Consider a composition $X\xrightarrow{g}Y\xrightarrow{f}Z$. The Leray spectral sequence is as follows:
\begin{equation*}
    R^n(f\circ g)_*\calF
        = \calH^n(\dR(f\circ g)_*\calF)
        = \calH^n(\dR f_*\dR g_*\calF)
        = R^nf_*(\dR g_*\calF)
\end{equation*}
We are in the situation of trying to compute the n-th derived functor of $F=f_*$ of some complex $A=\dR g_*\calF$. We can apply the second spectral sequence to see that
\begin{equation*}
    E_2^{p,q}
        = R^qf_*(R^pg_*\calF)
        \Rightarrow R^n(f\circ g)_*\calF
\end{equation*}
When we apply this to $Z=\Spec\C$, then $f_*$ is global sections and $\dR f_*$ is sheaf cohomology, as is $\dR(f\circ g)_*$ which gives us the Leray spectral sequence:
\begin{equation*}
    H^q(Y,R^pg_*\calF) \Rightarrow H^{p+q}(X,\calF)
\end{equation*}
\end{example}

\begin{example}{Local to global spectral sequence}{}
We know from before that
\begin{equation*}
    \RG\circ\RlHom = \RHom
\end{equation*}
Upon taking cohomology, on the right hand side we get $\Ext^n(\calF,\calG)$. On the other hand, taking $F=\Gamma$, $A=\RlHom(\calF,\calG)$ then the spectral sequence gives us
\begin{equation*}
    H^q(X,\lExt^p(\calF,\calG)) \Rightarrow \Ext^n(\calF,\calG)
\end{equation*}
\end{example}

As an application, we compute the Ext groups of a point, as well as the self local Exts of a subvariety.

More precisely, let $Y\subset X$ be a subvariety given by the zero locus of a section of $\calE$, which we can Koszul resolve:
\begin{equation*}
    \cdots\to\wedge^2\calE^\vee\to\calE^\vee\to\O_X\to\O_{Y}\to0
\end{equation*}
i.e. we replace $\O_Y$ by $\bigwedge^\bullet\calE^\vee$. Then $\lExt^\bullet(\O_Y,\O_Y)$ is computed by applying $\lHom(-,\O_Y)$ to the complex and taking cohomology. But once we restrict to $Y$, the maps become zero since they are given by the defining section $s$ for $Y$. Hence, the complex is formal and we get
\begin{equation*}
    \lExt^i(\O_Y,\O_Y)
        = \lHom(\bigwedge\nolimits^i\calE^\vee,\O_Y)
        = \bigwedge\nolimits^i\calE\otimes\O_Y
        = \bigwedge\nolimits^i\calE|_Y
        = \bigwedge\nolimits^i\calN_{Y/X}
\end{equation*}
This allows us to compute the self Ext groups of a point, by using the local-to-global spectral sequence:
\begin{equation*}
    H^i(X,\lExt^j(\O_p,\O_p)) \Rightarrow \Ext^{i+j}(\O_p,\O_p)
\end{equation*}
The local homs will be supported at $p$. So it has no higher cohomology i.e. the cases $i>0$ give nothing. But when $i=0$, we have no differential. So in fact
\begin{equation*}
    \Ext^\bullet(\O_p,\O_p) \simeq H^0(X,\lExt^\bullet(\O_p,\O_p))
\end{equation*}
From the Koszul resolution, we see that the latter is just an exterior algebra!

\begin{remark}{Mirror symmetry}{}
    In mirror symmetry, varieties come in pairs such that the Floer cohomology groups on one side correspond to Ext groups on the other. If we think of $X$ as the moduli space of its skyscraper sheaves, then since Ext groups should match up with self Floer groups, the mirrors to points should be Lagrangians with self Floer cohomology equal to an exterior algebra. But self Floer cohomology is equal to usual cohomology for unobstructed Lagrangians, which motivates the fact that mirror to points we should have (special) Lagrangian tori. This is the basis of the SYZ philosophy.
\end{remark}

\begin{remark}{Deformations}{}
    We see that the global sections of the normal bundle of $Y$, which describe its deformations inside $X$, can be described by $\Ext^1(\O_Y,\O_Y)$.
\end{remark}
We mention briefly the Hochschild (co)homology $$HH^\bullet(X):=\Ext^\bullet(\O_{\Delta_X}, \O_{\Delta_X}), HH_\bullet(X):= \Ext^{n-\bullet}(\Delta_* \omega_X ^{-1}, \O_{\Delta})$$ in the form of the following important theorem due to \cite{kontsevich_deformation_2003}, which boils down to the local-to-global spectral sequence degenerating:
\begin{theorem}{Hochschild-Kostant-Rosenberg}{HKR}
    $$HH^k(X)=\bigoplus_{p+q=k} H^q(X, \Lambda^p T_X), HH_k(X)=\bigoplus_{q-p=k} H^q(X, \Omega^p_X)=\bigoplus_{q-p=k} H^{p,q}(X)$$
\end{theorem}

Note that Hochschild cohomology is not functorial, but Hochschild homology is (though both are under fully faithful functors). 

\begin{remark}{More on deformations}{}
    The second Hochschild cohomology group parametrizes the deformations of the derived category $\calD(X)$ and by the HKR theorem we have that $$HH^2(X)=H^2(X,\O_X)\oplus H^1(X,T_X)\oplus H^0(X, \bigwedge\nolimits^2 T_X) $$
The first piece can be thought of deforming the sheaves on $X$: if we take $\alpha \in H^2(X,\O_X)$, its exponential $\exp(\alpha)\in H^2(X,\O_X^*)$ is a \emph{Brauer class}, i.e. a cocycle that allows for the definition of twisted vector bundles (see \ref{Twisted projective bundle formula} for more on this).
\newline

The second piece determines the deformations of $X$ as a variety, whereas the final, more mysterious piece, determines the non-commutative deformations.
\end{remark}


\subsection{Serre duality}

Serre duality is a classical fact about projective varieties that can be lifted to a structure on the derived category. We explain how.

\begin{proposition}{Classical Serre duality}{Classical Serre duality}
    If $X$ is smooth projective over $\C$, and $\calE$ is a vector bundle on $X$, then
    \begin{equation*}
        H^i(X,\calE)^* \simeq H^{n-i}(X,\calE^\vee\otimes\omega_X)
    \end{equation*}
\end{proposition}

The duality is given by a pairing. Namely, we have a perfect pairing
\begin{equation*}
    H^i(X,\calE)\otimes H^{n-i}(X,\calE^\vee\otimes\omega_X)
        \to H^n(X,\calE\otimes\calE^\vee\otimes\omega_X)
        \xrightarrow{\mathrm{tr}} H^n(X,\omega_X)
        \xrightarrow{\simeq} \C
\end{equation*}
An alternative way to prove this is via the Hodge star operator on vector bundles.

There is also a hypercohomology version. If we replace $\calE$ by a complex of vector bundles, we define the dual complex as
\begin{equation*}
    \lHom(\calE^\bullet,\O_X)^n = \lHom(\calE^{-n},\O_X)
\end{equation*}
Similarly, there is a trace map $\calE^\vee\otimes\calE\to\O_X$, so in the same way we can construct a pairing.

%*Example:* $\calE=V^0\to W^1$. The dual complex is $W^\vee\to V^\vee$, where now $W^\vee$ lives in degree $-1$. The tensor complex is $$W^\vee \otimes V \to  V^\vee \otimes V \oplus W^\vee \otimes W \to  V^\vee \otimes W$$The map is the natural one in degree zero.

\begin{proposition}{Serre duality for complexes}{Serre duality for complexes}
    Define the hypercohomology of a complex as $\bbH^i(X,\calE)\coloneqq\calH^i\RG(\calE)$. Then
    \begin{equation*}
        \bbH^i(X,\calE)\otimes\bbH^{n-i}(X,\calE^\vee\otimes\omega_X)\to\C
    \end{equation*}
    This pairing is still perfect.
\end{proposition}

One can use the LES on hypercohomology via truncating complexes and reduce this to the case of vector bundles (this method of proof is called devissage).

Finally, we come to the categorical version:

\begin{proposition}{Categorical form of Serre duality}{Categorical form of Serre duality}
    If $A,B\in D^b(X)$, and $n=\dim X$, then
    \begin{equation*}
        \Hom(A,B)^\vee \simeq \Hom(B,A\otimes\omega_X[n])
    \end{equation*}
\end{proposition}

The idea is as follows: we can compute the Homs of the derived category as
\begin{align*}
    \Hom_{D^b(X)}(B,A\otimes\omega_X[n])
        &\simeq \calH^n\RHom(B,A\otimes\omega_X) \\
        &\simeq \calH^n\RG(\RlHom(B,A\otimes\omega_X)) \\
        &\simeq \bbH^n(X,B^\vee\Lotimes A\otimes\omega_X).
\end{align*}

Similarly,
\begin{align*}
    \Hom_{D^b(X)}(A,B)
        &\simeq \calH^0\RHom(A,B) \\
        &\simeq \calH^0\RG(X,\RlHom(A,B)) \\
        &\simeq \bbH^0(X,A^\vee\Lotimes B).
\end{align*}
These are dual by the previous version of Serre duality, applying it to $\calE=A^\vee\Lotimes B$.

\begin{definition}{Serre functor}{Serre functor}
    Define $S:D^b(X)\to D^b(X)$, $F\mapsto F\otimes\omega_X[\dim X]$. Then
    \begin{equation*}
        \Hom(A,B)^*\simeq\Hom(B,S(A))
    \end{equation*}
     A Serre functor is an additive, $\C$-linear autoequivalence obeying the above property.
\end{definition}

Two such functors are canonically equivalent, since $S(A)$ represents the functor $B\mapsto\Hom(A,B)^*$.

% Check that $S$ commutes with any autoequivalence: $$\begin{gathered}
% \Hom(B,SFA)\simeq \Hom(FA,B)^*\\
% \Hom(B,FSA)\simeq \Hom(F^{-1}B,SA)\simeq \Hom(A, F^{-1}B)^*\simeq \Hom(FA,B)^*
% \end{gathered}$$So the two represent the same functor.

% The element $1_A$ corresponds via the isomorphism to some $\tau_A: \Hom(A, S(A))\to k$, which we call a trace.

% *Exercise:* given sequence $$A\xrightarrow{x}B\xrightarrow{y}S(A)\xrightarrow{S(x)}S(B)$$then $\mathrm{Tr}(yx)=\mathrm{Tr}(S(x)y)$.

The Serre functor allows us to switch between left and right adjoints. Suppose $F$ left adjoint to $G$. Then $SFS^{-1}$ is right adjoint to $G$.

% The point is that the duality pairing is the same as the composition and then the trace: $\mathrm{Tr}(y\circ x)=\mu(x,y)$, where $\mu$ is the duality pairing. Claim we have a commutative diagram, where on the left we send $(x,y)\mapsto(y, Sx)$.
% $$\begin{CD}
% \Hom(A,B)\otimes \Hom(B,SA)@> > > k\\ @VVV @VVV \\ \Hom(B,SA)\otimes \Hom(SA,SB)@> > > k
% \end{CD}$$
% Equivalently, can think about composite isomorphisms $$\Hom(B,SA)\simeq \Hom(A,B)^*\simeq \Hom(SA,SB)^*\simeq \Hom(B,SA)$$
% The first arrow sends $x\mapsto \mu(x,-)$. However, the leftmost arrow is given by $x\mapsto \mu(-,x)$. Hence, the two are related via $S$ and so $\mathrm{Tr}(yx)=\mu(x,y)=\mu(y,Sx)=\mathrm{Tr}(S(x)y)$.

\begin{definition}{Calabi-Yau category}{Calabi-Yau category}
    A category is called a Calabi-Yau $n$ category if it has a Serre functor given by just shifting by $n$, $S=[n]$.
\end{definition}

\begin{example}{Shriek functor and Grothendieck-Verdier duality}{Shriek functor}
Given $f:X\to Y$, we have that $\dL f^*$ is left adjoint to $\dR f_*$. In both categories there are Serre functors, so can obtain the right adjoint to the pushforward as
\begin{gather*}
    f^!(A) \coloneqq (\dL f^*A\otimes f^*\omega_Y^{-1}[-\dim Y])
            \otimes \omega_X[\dim X] \\
        = \dL f^*A\otimes(f^*\omega_Y^{-1}\otimes\omega_X)[\dim X -\dim Y]
\end{gather*}
We call $\omega_f\coloneqq f^*\omega_Y^{-1}\otimes\omega_X=f^!\O_Y$, the relative dualizing sheaf. The adjunction reads
\begin{equation*}
    \Hom(\dR f_*A,B)\simeq \Hom(A,f^!B)
\end{equation*}
In fact, there is a sheafified version:
\begin{equation*}
    \RlHom(\dR f_*A,B)=\dR f_*\RlHom(A,f^!B)
\end{equation*}
In particular, with $B=\O_Y$ get
\begin{equation*}
    (\dR f_*\calF)^\vee
        \simeq \dR f_*(\calF^\vee\otimes\omega_f[\dim X-\dim Y])
\end{equation*}
Putting $Y=\pt$, we recover usual Serre duality! So this can be seen as a relative version of Serre duality. But in general, cannot take the pieces $R^i$ as dual! Have to keep the whole chain complex.
\end{example}

%\begin{example}{Poincare bundle again}{}
%Take the universal line bundle on an elliptic curve. The canonical bundle is trivial, so by relative Serre duality we have
%\begin{equation*}
%    (\dR\pi_*\calL)^\vee \simeq \dR\pi_*\calL^{-1}[1]
%\end{equation*}
%Can use Cech resolutions to see that $f_*\calL$ is a torsion free sheaf supported at a point, so has to be zero. So $R^0=0$ on both sides. But $R^1\pi_*\calL=\O_{P_0}[1]$ as we saw before, so the duality does not hold termwise, but over the whole complex! This means that there is a group
%\begin{equation*}
%   \lExt^1(\O_{P_0},\O) = \O_{P_0}
%\end{equation*}

%There is a generalization of $f^!$ to general nice schemes, if we work with quasicoherent sheaves. Still have duality isomorphisms and the isomorphism $f^!\calF\simeq \dL f^*\calF\otimes f^!\O$.
%\end{example}

%In general, if $p:X\rightarrow pt$, we are interested in $p^!(\O_{pt})=\calD_{X}$. But we can compose with an embedding into projective space $i:X\rightarrow \P^n$ and since adjoints are functorial, see that $\calD_{X}=i^!\calD_{\P^n}=i^!\omega_{\P^n}[n]$. So need to understand $i^!$ for an embedding. But it can actually be defined as a derived functor!

%If we have a closed embedding $i:X\rightarrow Y$, there is an abelian version of $i^!$, namely a derived functor of $$\calF\mapsto \lHom(i_{*}\O_{X},\calF)=i_{*}(\mathcal{H}^0i^!\calF)$$More generally, $$\RlHom(i_{*}\O_{X},\calF)\simeq i_{*} i^!\calF$$If $X$ is Cohen-Macaulay, then we have a formula for the dualizing sheaf: $$i_{*}\omega_{X}\simeq \lExt^{codim X}(i_{*}\O_{X}, \omega_{\P^n})$$
%So we have that $\calD_{X}\simeq \omega_{X}[n]$ and Serre duality becomes $H^i(X,\calF)^*\simeq H^{n-i}(\calF, \omega_{X})$.

% example of a (2,3) K3 surface in P4?

\subsection{Fourier-Mukai transforms}
\subsubsection{Definition and Orlov's theorem}
We come to the subject of Fourier-Mukai transforms. A kernel in analysis is a function against which we integrate, and in a similar way a kernel in algebraic geometry is an object on the product which we tensor with and then we pushforward:

\begin{definition}{Fourier-Mukai functor}{Fourier-Mukai functor}
    Given $\calE\in D^b(X\times Y)$, its associated Fourier-Mukai functor is
    \begin{gather*}
        \Phi_\calE : D^b(X)\to D^b(Y) \\
        \calF \mapsto \dR p_{*}(\calE\otimes\dL q^*\calF)
    \end{gather*}
    where $q:X\times Y\to X$, $p:X\times Y\to Y$ are the projections.
    % p,q convention here follows Huybrechts, although it's unintuitive. could change to explicit \pi_X,\pi_Y but then that's confusing in X\times X examples
\end{definition}

We will sometimes write $\calG \boxtimes \calF := p^* \calG \otimes q^* \calF$ for the sheaf on the product coming from $\calG$ on $X$ and $\calF$ on $Y$.

\begin{remark}{}{}
    Since $X\times Y\simeq Y\times X$, an object $\calE\in D^b(X\times Y)$
    defines two FM transforms; mapping $D^b(X)\to D^b(Y)$ and
    $D^b(Y)\to D^b(X)$. We write $\Phi^{X\to Y}_\calE$ and $\Phi^{Y\to X}_\calE$
    to distinguish these if it is not clear from context.
\end{remark}

All of the functors we have so far encountered are FM transforms:
\begin{itemize}
    \item Given $f:X\to Y$ with $\Gamma_f\subset X\times Y$, the structure sheaf of the graph is the kernel of the pushforward functor and the pullback functor, depending on the way we go.

    Have $\O_\Gamma=(1\times f)_*\O_X$ and we use the projection formula:
    \begin{equation*}
        p_*(\O_\Gamma\otimes q^*\calF)
            = p_*(1\times f)_*(\O_X\otimes(1\times f)^*q^*\calF)
            = f_*\calF
    \end{equation*}

    \item Tensoring with $\calG$ is the Fourier-Mukai transform with respect to $\Delta_*\calG$ where $\Delta:X\to X\times X$ is the diagonal embedding. In particular the identity is induced by $\O_\Delta$.

    \item Shift functor is induced by $\O_\Delta[n]$.

    \item If $Y$ is a fine moduli space of sheaves on $X$ equipped with a universal sheaf $\calE\to X\times Y$ parametrizing sheaves by restricting to $X\times\{y\}$. We see that the Fourier-Mukai transform w.r.t. the universal sheaf of a skyscraper sheaf $\O_y$ is $\calE|_{X\times\{y\}}$, the sheaf that the point $y\in Y$ parametrizes. Have induced map
        \begin{equation*}
            T_yY
                = \Ext^1(\O_{y},\O_{y})
                \to \Ext^1(\calE|_{X\times\{y\}},\calE|_{X\times\{y\}})
        \end{equation*}
        This is the so-called Kodaira-Spencer map! Deformations of the point give first-order deformations of the sheaf.
\end{itemize}

Given kernels $\calE$, $\calE'$ on $X\times Y$ and $Y\times Z$ respectively, we consider the kernel $\calE''=\pi_{13}*(\pi_{12}^*\calE\otimes\pi{23}^*\calE')$ on $X\times Z$ by pulling to the triple product and then pushing down. We claim that
\begin{proposition}{Composition of kernels}{Composition of kernels}
\begin{equation*}
    \Phi_{\calE'}\circ\Phi_{\calE} = \Phi_{\calE''}
\end{equation*}
\end{proposition}
\begin{proof}
    This is done by using the projection formula a bunch of times and finally an instance of flat base change. Basically, one first transforms everything to a tensor product on the triple product $X\times Y \times Z$ by using the projection formula. Then, one one can use different ways to factorize the projection to $Z$, together with the projection formula and base change for the diamond involving $X\times Y \times Z$, $X\times Y$, $Y\times Z$, $Y$. For the complete proof and diagram, consult \cite[\S5][Proposition~5.10]{Huybrechts}.
\end{proof}

\begin{proposition}{Adjoints of FM transform}{Adjoints of FM transform}
    The adjoints of a Fourier-Mukai functor are given by the dual kernel together with a Serre functor: given $\Phi_\calE$ going from $X$ to $Y$, have two adjoints given by $\Phi_{\calE^\vee\otimes q^*\omega_{X}[\dim X]}$ and $\Phi_{\calE^\vee\otimes p^* \omega_{Y}[\dim Y]}$.
\end{proposition}
\begin{proof}
For this, we crucially need Grothendieck-Verdier duality!
\begin{gather*}
    \Hom_Y(\Phi_\calE(\calF),\calG)
        = \Hom_Y(p_*(\calE\otimes q^*\calF),\calG)
        \simeq \Hom_{X\times Y}(\calE\otimes q^*\calF,p^!\calG) \\
        \simeq \Hom_{X\times Y}(\calE\otimes q^*\calF,p^*\calG\otimes\omega_p)
        \simeq \Hom_{X\times Y}(\calE\otimes q^*\calF,p^*\calG\otimes q^*\omega_X[\dim X]) \\
        \simeq \Hom_{X\times Y}(q^*\calF,\calE^\vee\otimes p^*\calG\otimes q^* \omega_X[\dim X])
        \simeq \Hom_X(\calF,\Phi_{\calE^\vee\otimes q^*\omega_X[\dim X]}(\calG))
\end{gather*}
\end{proof}
So if the kernel induces an equivalence, the dimensions of the two varieties should be the same.

\begin{theorem}{Orlov}{Orlov's theorem on Fourier-Mukai transforms}
    Every fully faithful exact functor between is induced by a Fourier-Mukai transform. In particular every equivalence as well. Moreover the kernel is unique up to isomorphism.
\end{theorem}

We also cite the following useful criterion, due to Bondal-Orlov \cite{bondal_semiorthogonal_1995}:
\begin{proposition}{Generation criterion}{Generation criterion}
    A Fourier-Mukai transform with kernel $\calE$ is fully faithful if and only if \begin{gather*}
        \Hom (\Phi_\calE \O_x, \Phi_\calE \O_y[i])=\begin{cases*}
           \C, \text{ if }x=y, i=0\\
           0, \text{ if }x\neq y \text{ or } i<0 \text{ or } i> \dim(X)
        \end{cases*}
    \end{gather*}

\end{proposition}

The proof is slightly technical and can be found in \cite[\S7][Proposition~7.1]{Huybrechts}. The point is that skyscraper sheaves are a \emph{spanning class}.

%There is a generalization by Canonaco and Stellari which requires only that $\Ext^i(F,F)=0$ for $i<0$ implies $\Ext^i(\Phi F,\Phi F)=0$.

% We can consider a 2-category with objects smooth projective varieties, morphisms given by Fourier-Mukai transforms via kernels and 2-morphisms are maps of kernels. Orlov's theorem says the functor into the 2-category of triangulated categories is essentially injective.

\subsection{Exceptional objects, admissible subcategories and semiorthogonal decompositions}

\subsubsection{Exceptional collections and Beilinson's theorem}

\begin{definition}{Exceptional object}{Exceptional object}
    An exceptional object is one with the cohomology of a point:
    \begin{equation*}
        \Ext^i(\calE,\calE) = \Hom_{D^b(X)}(\calE,\calE[i]) = \begin{cases*}
            \C & if $i=0$, \\
            0 & otherwise.
        \end{cases*}
    \end{equation*}

    Given a morphism $X\rightarrow Y$, we say $\calE$ is relatively exceptional w.r.t. $Y$ if $$\dR f_* \lHom(\calE, \calE)=\O_Y$$
    This reduces to the previous definition in the case that $Y$ is a point.
\end{definition}

For example, line bundles on $\P^n$ and more generally on Fanos by Kodaira vanishing. Also, $\O_{E}(t)$ on a blowup at a point with exceptional divisor $E$. This is done by applying LES to the ideal sheaf sequence
\begin{equation*}
    0 \to \O_{\tilde X}(-E) \to \O_{\tilde X} \to \O_E \to 0
\end{equation*}
Then after homming into $\O_E$ we get
\begin{equation*}
    \Ext^\bullet(\O(-E),\O_E)
        = H^\bullet_{\tilde X}(\O_E(E))
        \simeq H^\bullet_E(\O_E(-1))=0
\end{equation*}
since $E=\P^{n-1}$. Moreover, since $p^*\O_X=\O_{\tilde X}$, by adjunction we get
\begin{equation*}
    \Ext^\bullet(\O_{\tilde X},\O_E)
        = \Ext^\bullet(\O_X,p_*\O_E)
        = \Ext^\bullet(\O_X,\O_p)
        = \C
\end{equation*}

Many homogenous vector bundles are exceptional, e.g. the tangent bundle of projective space found in
Example~\ref{ex:tangentbundle}.

Deformable objects are in general not exceptional (as they have an $\Ext^1$), e.g. skyscraper sheaves, line bundles on curves. 

\begin{remark}{}{}
    Calabi-Yau's don't admit exceptional objects by Serre duality. However, they do admit spherical objects, e.g. $\O_C$ for a rational curve $C$ in a K3 surface.
\end{remark}

\begin{definition}{Exceptional collections}{Exceptional collections}
    A collection $E_i$ is exceptional if each $E_i$ is exceptional and there are no homs from right to left: $\RHom(E_i,E_j)=0, i>j$.
\end{definition}

We will see in a moment that $\O,\dots,\O(n)$ is such a collection on projective space. Cannot have more, since then we will have the cohomology of the canonical bundle $\omega=\O(-n-1)$ which is nonzero.

More generally, for a Fano, $\omega_X=\O_X(-k)$ is antiample. Then $\O,\dots,\O(k-1)$ is exceptional. We will come back to this when we discuss the cubic fourfold.

We put $E \perp F$ whenever $\RHom(E,F)=0$. This is not a symmetric relation! However, it is on Calabi-Yau's.

\begin{definition}{Full exceptional collection}{Full exceptional collection}
    An exceptional collection is full if you can generate every object by sums, shifts and cones.
\end{definition}

\begin{theorem}{Beilinson}{Beilinson's theorem}
    On $\P^n$, the exceptional collection $\O,\dots,\O(n)$ is full.
\end{theorem}

\begin{proof}
    \textbf{Step 1}:
    First, we check that $\langle\O,\dots,\O(n)\rangle$ is an exceptional collection.

    Consider the morphisms between two elements, where $0\leq j\leq i \leq n$
    \begin{align*}
        \Hom_{D^b(\P^n)}(\O(i),\O(j)[l])
            &= R^l\Hom(\O(i),\O(j))
            = R^l\Gamma(\O(j-i)) \\
            &= H^l(\P^n,\O(j-i))
            = \begin{cases*}
                \C & if $l=j-i=0$, \\
                0 & otherwise
            \end{cases*}
    \end{align*}
    since $H^{l}(\P^n,\O(m))=0$ for any $m<0$ or $l \neq 0$. Hence the collection is strong and exceptional.

    \textbf{Step 2}:
    Now, it remains to show that the sequence is full. That is, it generates all of $D^b(\P^n)$.

    We do this by finding a Koszul resolution of the diagonal, so we need to find a bundle and a section of it which cuts it out.

    Consider the projection of the product to each component
    \begin{equation*}
        \begin{tikzcd}
            & \P^n\times\P^n \ar[dl,"q"'] \ar[dr,"p"] \\
            \P^n & & \P^n
        \end{tikzcd}
    \end{equation*}

    We use the notation $\O(-1)\boxtimes\Omega(1)=q^*\O(-1)\otimes p^*\Omega(1)$. We have from \cite{Hartshorne} the Euler sequence
    \begin{equation*}
        0 \to \Omega(1) \to \O^{n+1} \to \O(1) \to 0
    \end{equation*}
    Noting that $p^*\O_{\P^n}\simeq q^*\O_{\P^n}\simeq\O_{\P^n\times\P^n}$, we can form the composition of the exact sequence under $p^*$ and $q^*$ respectively, giving
    \begin{equation*}
        p^*\Omega(1) \to \O_{\P^n\times\P^n}^{n+1}\to q^*\O(1)
    \end{equation*}
    Tensoring with $q^*\O(-1)$, and the fact that $q^*\O(-1)\otimes q^*\O(1)\simeq q^*\O\simeq\O_{\P^n\times\P^n}$, we get the natural composition of morphisms
    \begin{equation}\label{eqn:natural section}
        \O(-1)\boxtimes\Omega(1) \to \O_{\P^n\times\P^n}
    \end{equation}
    The fibre of $\O(-1)$ above a point in $\P^n$ is the point considered as a one-dimensional subspace of $\C^{n+1}$, and the fibre of $\Omega(1)$ at a point $l \in \P^n$ is the space of maps $\phi : \C^{n+1}\to \C$ which are zero on the line $l$. Hence, an element of the fibre of  $\O(-1)\boxtimes \Omega (1)$ at $(l,l') \in \P^{n}\times \P^n$ is $(v,\phi)$, where $v \in l$ and $\phi$ vanishes on $l'$. Looking locally over the point $(l,l')$, we have the evaluation map
    \begin{equation*}
        \ev : \O(-1)\boxtimes \Omega (1) \to \O_{\P^{n}\times \P^{n}}
    \end{equation*}
    sending $(v,\phi)$ to $\phi(v)$. The image of this map cuts out the same locus as the ideal sheaf of the diagonal, so we can use the Koszul construction to get the locally free resolution for $\O_\Delta$
    \begin{equation*}
        0 \to \bigwedge\nolimits^{n}\O(-1)\boxtimes\Omega(1) \to \dots\to \O(-1)\boxtimes\Omega(1)\to \O_{\P^{n}\times \P^{n}} \to O_{\Delta}\to 0.
    \end{equation*}

    Essentially, the morphism \ref{eqn:natural section} can be thought as a morphism $$\calL_1 \rightarrow \mathcal{Q}_2$$where $\calL_1=p^*\O(-1)$ is the tautological line and $\mathcal{Q}_2=q^*\mathcal{Q}$ is the universal quotient. This morphism is a section of the Hom-bundle $\lHom(\calL_1, \mathcal{Q}_2)$ and is zero precisely at $(x,y)\in \P^{n}\times \P^n$ such that the composition $$l_x \hookrightarrow \C^{n+1}\twoheadrightarrow \C^{n+1}/l_y$$
    is zero, where $l_x, l_y$ are the lines corresponding to $x,y$. This happens exactly when $x=y$ i.e. on the diagonal.


    Let $\calE\coloneqq\O(-1)\boxtimes\Omega(1)$.
    We can split the above resolution into a chain of short exact sequences
    \begin{align*}
        0 \to \bigwedge\nolimits^{n}\calE \to &\bigwedge\nolimits^{n-1}\calE\to M_{n-1}\to 0  \\
        0\to M_{n-1}\to & \bigwedge\nolimits^{n-2}\calE\to M_{n-2}\to 0  \\
        &\quad\vdots \\
        0\to M_{1}\to &\O_{\P^{n}\times \P^{n}}\to\O_{\Delta}\to 0
    \end{align*}
    with $M_{k-1}\coloneqq\im(\bigwedge^{k}\calE \to \bigwedge^{k-1}\calE)$.

    Let $\calF\in D^b(\P^n)$. Given that $q^*$, $p^*$ and $\otimes$ are exact, $\Phi_{(-)}(\calF):\calG\mapsto q_{*}(p^{*}\calF\otimes\calG)$ is also exact for $\calG \in D^b(\P^n\times \P^n)$. This gives us exact triangles
    \begin{align*}
        \Phi_{\bigwedge^{n}\calE}(\calF) \to &\Phi_{\bigwedge^{n-1}\calE} (\calF)\to \Phi_{M_{n-1}}(\calF)  \\
        \Phi_{M_{n-1}}(\calF)\to & \Phi_{\bigwedge^{n-2}\calE} (\calF)\to \Phi_{M_{n-2}}(\calF)  \\
        &\vdots \\
        \Phi_{M_{1}}(\calF)\to &\Phi_{\O_{\P^n\times\P^n} }(\calF)\to \Phi_{\O_{\Delta}} (\calF)
    \end{align*}
    We can identify that $\bigwedge^{k}\O(-1)\boxtimes\Omega(1)$ with $\O(-k)\boxtimes\Omega^k(k)$. Then, due to the flatness of $p$ and $q$, and the projection formula, we have for each $i \leq n$,
    \begin{align*}
        \Phi_{\bigwedge^n\calE}(\calF)
            &= q_*(p^*\calF\otimes(\O(-i)\boxtimes\Omega^i(i))) \\
            &= q_*(p^*\calF\otimes(q^*\O(-i)\otimes p^*\Omega^i(i))) \\
            &= q_*(p^*(\calF\otimes\Omega^i(i))\otimes q^*\O(-i)) \\
            &= q_*p^*(\calF\otimes\Omega^i(i)\otimes\O(-i)
                \qquad \text{(projection formula)} \\
            &= \RG(\calF\otimes\Omega^i(i))\otimes\O(-i)
                \qquad \text{(base change)} \\
            &= H^\bullet(\P^n,\calF\otimes\Omega^i(i))\otimes\O(-i)
    \end{align*}
    Hence $\Phi_{\bigwedge^{n}\calE} (\calF)$ lies in $\langle\O(-i)\rangle$ as a tensor product. By closure under exact triangles $\Phi_{M_{n-1}}(\calF)$ also lies in $\langle\O(-n),\O(-n+1)\rangle$. By induction, $\Phi_{M_{n-i}}(\calF)$ lies in $\langle\O(-n),\O(-n+1),\dots,\O(-n+i)\rangle$. Hence $\Phi_{\O_\Delta}(\calF)$ is generated by $\O(-n),\dots,\O$. But $\Phi_{\O_\Delta}(\calF)\simeq\calF$. Tensoring with $\O(n)$, which is an exact equivalence, we have the desired result.
\end{proof}

After reviewing admissible subcategories and semiorthogonal decompositions, we will show how to extend this to projective bundles (essentially by using the same resolution of the diagonal).

% one method of proof: use Koszul resolution of skyscrapers to see they are contained in this subcategory. Then, find the orthogonal complement is empty, for which we need the collection to be exceptional. Suppose that $\mathrm{RHom}(\O_{p}, \calF)=0,\forall p \in X$. Use the Grothendieck spectral sequence $$\mathrm{R}^p\Hom(\O_{p}, \mathcal{H}^q \calF)\implies \mathrm{R}^{p+q}\Hom(\O_{p},\calF)=0 $$If we choose $q$ maximal with $\mathcal{H}^q\calF\neq 0$ and $p\in \mathrm{supp}\,\calF$, then we will have a nonzero term in the top right corner of the second page, which survives to infinity. This is a contradiction, and hence either $\calF=0$ or all the cohomologies are zero.

% Junk on blowup of P2
% Want to show that for a blowup $\mathrm{R}f_{*}\O_{\tilde{X}}=\O_{X}$. By Zariski's main theorem, have that it's true for $R^0$. Have projective bundle $\P^n\to E\to Z$ and a SES $$0\to \O_{\tilde{X}}(-E)\to \O_{\tilde{ X}}\to \O_{E}\xrightarrow{ }0$$Hence, $\mathrm{R}f_{*}\O_{E}=\O_{Z}$. When we push down to $X$ under the LES, the first level is the ideal sheaf sequence for $Z$. Can twist by $-E$, hence $\O_{E}(E)=\O_{E}(1)$. Want to show these have no higher pushforwards. In particular, $R^{>0}f_{*}(\O(-nE))=0, n\gg 0$. But $\O_{X}(-E)=\O_{f}(1)$ is relatively ample, so can use Serre vanishing? Not sure.

% > [!theoremred] Theorem (Kapranov)  Let $\mathcal{U}\to \mathrm{Gr}_{k}(n)$ be the tautological bundle over the Grassmanian. Then the Schur functors $\mathrm{Schur}^\lambda \mathcal{U}^\vee$ for $\lambda$ a Young diagram form a full exceptional collection.

%Blowup exceptional collection
%Let's go back to the example of the blowup of $\P^2$ at a point. For any $p\notin E$, we have that $\O_{p}=\mathrm{L}f^*\O_{f(p)}$ which we can Koszul resolve on the base then pull back to get $$\O_{\tilde{X}}(-2H)\to \O_{\tilde{ X}}(-H)^2\to \O_{\tilde{X}}\to \O_{p}$$So these skyscraper sheaves are in the span of $\O,\O(H),\O(2H)$, after twisting by $2H$. When $p\in E$, then $f(p)$ is in the center. By exercise, have $$\mathrm{L}_{i}f^*\O_{f(p)}=\begin{cases}
%\O_{E},i=0 \\
%\O_{E}(-1), i=1
%\end{cases}$$
%Hence, have a map $\O_{E}(-1)[1]\to \mathrm{L}f^*\O_{f(p)}$ since all complexes admit a map from their last cohomology and a map to their first cohomology. The cone of this map has to be $\O_{E}$ by looking at the LES in cohomology.

%Finally, have a SES $$0\rightarrow \O_{E}(-1)\to \O_{E}\to \O_{p}\to 0$$This shows that the exceptional collection is full.

%Quadric in P5 Have that $\omega_{Q}=\O(-4)$ so have exceptional collection of line bundles up to degree 4. We will show this is not full. This contains a plane $P\subset Q$. Consider $\O_{P}(1)$. Then $$\mathrm{RHom}(\O_{Q}(3), \O_{P}(1))=\mathrm{R\Gamma}(\O_{P}(-2))=0$$Similarly with $\O_{Q}(2)$. However, with $\O_{Q}(1)$ get $\mathrm{R}\Gamma (\O_{P})=k$. So it is not orthogonal! Consider the kernel of the restriction morphism $$0\to \mathcal{I}_{P/Q}(1)\to \O_{Q}(1)\to \O_{P}(1)\to 0$$We claim that the ideal sheaf is orthogonal to the line bundles. Up to a shift, this is a cone of two things which were already orthogonal to $\O(2), \O(3)$ but orthogonality is closed under taking cones. So we need only worry about $\O(1)$. Then use LES to conclude that $\Ext^\bullet(\O_{Q}(1), \mathcal{I}(1))=0$. What about $\O_{Q}$?

%Via the LES again, we can conclude that $\Ext^\bullet(\O_{Q}, \mathcal{I})=k^3$ concentrated in degree 3, given by the three linear forms cutting out $P$. So not orthogonal and play the same game again: $$0\to \mathcal{S}\to \O_{Q}^3\to \mathcal{I}\to 0$$Then $\mathcal{S}$ is orthogonal to $\O, \O(1),\dots,\O(3)$. It is of rank $2$ so is nonzero. It is a spinor bundle? Moreover, it is exceptional - see exercises.

%In Kapranov's paper, if we take planes $P_+, P_-$ from two families on $Q,$ then $S_+, S_-$ are exceptional and $$\calD^b(Q)=\langle S_{+},S_{-}, \O_{Q}, \O_{Q}(1), \O_{Q}(2), \O_{Q}(3)\rangle$$
%See: Bondal-Orlov on two quadric intersecting, Kuznetsov on cubic fourfold.

\subsubsection{Admissible subcategories and mutations}

Given a subcategory $\calA \hookrightarrow \calD$ we can study its orthogonal complements: $$\calA^\perp : = \{ X \in \calD\mid\Hom(A,X)=0, \forall A\in \calA\}, \quad \lperp\calA : = \{ X \in \calD\mid\Hom(X,A)=0, \forall A\in \calA\}$$

An exceptional collection $E$ might not be full, but we can study the leftover bit. In fact, we can think of a Gram-Schmidt sort of process that projects to the component
\begin{equation*}
    \calD \to \langle E\rangle^\perp \coloneqq \{F\mid\Ext^i(E,F)=0\}
\end{equation*}
Suppose we want to project out of a single exceptional object $E$. In analogy with vector spaces, the orthogonal component of a vector $v$ is given by $v-\langle e,v \rangle e$. Here, the $\langle e,v \rangle e$ is the bit of $v$ that lies in the subspace generated by $e$ and the difference is the orthogonal component.

We can do the same thing in derived categories, except that homs are not symmetric, so there are two ways to do this, corresponding to the left and right orthogonals.

\begin{definition}{Orthogonal projections from single exceptional object}{Orthogonal projections from single exceptional object}
    There are two projection functors $\mL_E:\calD\to\langle E\rangle ^\perp$, $\mR_E:\calD\to\lperp\langle E\rangle$
    \begin{equation*}
        \mL_E(F) \coloneqq \cone(\RHom(E,F)\otimes E \to F), \quad
        \mR_E(F) \coloneqq \cone(F\to\RHom(F,E)^\lor\otimes E )[-1]
    \end{equation*}
\end{definition}

The main point is that there are two projection functors to the subcategory $\langle E \rangle$ given by $F\mapsto\RHom(E,F)\otimes E$ and $F\mapsto\RHom(F,E)^\lor\otimes E$, of which we take the cone. These two projections are universal, in the sense that they are the two adjoints of the inclusion functor $\iota:\langle E \rangle \to \calD$. Another way to say this is that $\langle E \rangle$ is an admissible subcategory, which roughly means a subcategory equipped with projection functors.

\begin{definition}{Admissible subcategory}{Admissible subcategory}
    A full subcategory $\calA$ closed under shifts and cones is admissible if the inclusion $\iota_*$ admits both a left $\iota^*$ and a right adjoint $\iota^!$.

    We call $\calA$ left-admissible if only a left adjoint exists and $\iota^*\circ \iota=1$, and similarly right-admissible of only a right adjoint exists and $\iota^! \circ \iota=1$.
\end{definition}

For example, the image of a fully faithful functor $D^b(Y)\to D^b(X)$ which is induced by a FM kernel, which has adjoints given by Serre functors.

Nonexample: the subcategory generated by skyscrapers. This has empty orthogonal complement, but everything in it has finite support. But if one has an admissible subcategory with empty orthogonal, it actually generates the whole category. So it is enough for an admissible subcategory to span all skyscrapers for it to generate the derived category.

Now, we can proceed in the exact analogous way to define orthogonal projection functors, by taking the cones of the unit and counit morphisms:

\begin{definition}{Mutations}{Mutations}
    Given an admissible subcategory $\calA$, there are two functors
    \begin{equation*}
        \mL_\calA(F) \coloneqq \cone(\iota_*\iota^!\to F)\in \calA^\perp, \quad
        \mR_\calA(F) \coloneqq \cone(F\to \iota_*\iota^* F )[-1]\in \lperp\calA
    \end{equation*}
    which are in fact the left and right adjoints of the inclusions $\calA^\perp \subset\calD$ and $\lperp\calA\subset\calD$. Moreover, the two define inverse equivalences $\mathcal{A}^\perp \simeq \lperp{}\mathcal{A}$
\end{definition}

\begin{proof}
    \textbf{Step 1}: We first show the mutation functors land in the orthogonal categories

    Note that if $F\in \calA$, then both $F\simeq \iota_*\iota^* F \simeq \iota_*\iota^! F$ so the cone is actually zero. Hence, both of these functors vanish on $\calA$ and in fact $\mL_\calA$ lands in $\calA^\perp$ whereas $\mR_\calA$ lands in $\lperp\calA$, by considering the long exact sequence associated to $\cone(\iota_*\iota^!F\to F)$. For any $A\in \calA$ we have
    \begin{gather*}
        \Hom_\calD(A, \iota_*\iota^!F)
            \simeq \Hom_\calD(\iota_* A, \iota_*\iota^!F)\simeq \\
            \simeq \Hom_\calA(A, \iota^!F)\simeq \Hom_\calD(\iota_* A, F)
            \simeq \Hom_\calD(A, F)
    \end{gather*}
    This implies by the LES that $\Hom(A,\cone(\iota_*\iota^!F\to F))=0$.

    \textbf{Step 2}: Now we show the mutation functors are adjoint to inclusions.

    Recall that the mutation functors fit into exact triangles $$k_*\mR_\calA (E)\rightarrow E \rightarrow \iota_* \iota^* E,\, \iota_* \iota^! E \rightarrow E \rightarrow j_*\mL_\calA(E)$$
    where $j_*: \calA^\perp \rightarrow\calD, k_*: \lperp{}\calA\rightarrow\calD$ are the inclusions.

    %By the third axiom of triangulated categories, if we have a morphism $E\rightarrow j_*F$, we can fill in
    %\begin{center}
    %\begin{tikzcd}
    %    \iota_* \iota^! E \arrow[r] \arrow[d] & E \arrow[r] \arrow[d] & j_*\mL_\calA E \arrow[d, dotted] \\
    %    0 \arrow[r]                           & j_* F \arrow[r]       & j_* F
    %    \end{tikzcd}
    %\end{center}

    %This gives us a map $$\Hom_\calD(E, j_*F)\rightarrow \Hom_\calD(j_*\mL_\calA E, j_*F)\simeq \Hom_{\calA^\perp}(\mL_\calA E, F)$$

    %Similarly, we can get a map in the opposite direction, which is inverse to this one, showing that $\mL_\calA \dashv j_*$. In a similar way we can show $k_* \dashv \mR_\calA$. 
    
    Since cones commute with Homs, we have that
    \begin{gather*}
        \Hom_{\calA^\perp}(\mL_\calA E, F)\simeq \Hom_\calD(j_*\mL_\calA E, j_*F) \simeq \\
        \simeq \Hom_\calD (\mathrm{cone}(\iota_* \iota^! E \rightarrow E), j_*F)\simeq \mathrm{cone}\big(\Hom_\calD(\iota_* \iota^! E, j_* F)\rightarrow \Hom_\calD(E, j_*F)\big)\simeq \\
        \simeq \mathrm{cone}(0\rightarrow \Hom_\calD(E, j_*F))\simeq \Hom_\calD(E, j_*F)
    \end{gather*}
    The reason being that $\Hom(\iota_* \iota^! E, j_* F)=0$ since $\iota_* \iota^! E\in \calA, j_*F \in \calA^\perp$. This shows that the left mutation is left adjoint to $j_*$. The right mutation can be shown analogously.

    \textbf{Step 3}: Now we show they induce equivalences between the left and right orthogonals.

    Firstly, we see that $\mL_\calA j_* F = \mathrm{cone}(\iota_* \iota^! j_* F \rightarrow j_* F)$. But this is just $j_* F$ since $\iota_* \iota^! j_* F=0$, which is true since $\ker \iota^! = \calA^\perp $. So $\mL_\calA$ is the identity on $\calA^\perp$ and zero on $\calA$, and similarly for $\mR_\calA$.

    Let $F\in \lperp{}\calA$. We observe that since $\mL$ and $\mR$ are adjoint to inclusions, which are exact, then they themselves must be exact. So we can turn the exact triangle $$\iota_* \iota^! F \rightarrow F \rightarrow j_*\mL_\calA(F)$$
    into $$\mR_\calA(\iota_* \iota^! F) \rightarrow \mR_\calA F \rightarrow \mR_\calA (j_*\mL_\calA(F))$$

    But $\mR_\calA$ vanishes on $\calA$, so we get that the first object is zero and hence the restrictions $\mR \mL F \simeq \mR F\simeq F$, since we picked $F\in \lperp{}\calA$ where $R$ acts as the identity. We can do the same in reverse to show that $\mL|_{\lperp{}\calA}: \lperp{}\calA \rightarrow \calA^\perp, \mR_{\calA^\perp}: \calA^\perp \rightarrow \lperp{}\calA$ are inverse equivalences.


    %Recall that a general pair of adjoint functors defines an equivalence on the subcategories where the unit, resp. counit, are isomorphisms. In our case for $\mL_\calA$, the left hand side will just be $\calA^\perp$. On the other hand, the subcategory where the the counit is an isomorphism consists of all $F\in \calD, F\simeq j_* \mL_\calA F$. But from the exact triangle we see that this is equivalent to $\iota_* \iota^! F=0$ i.e. $\iota^! F=0$. We thus see that the induced equivalence is between $\calA^\perp$ and $\ker \iota^!$.
\end{proof}

\begin{corollary}{Mutations of exceptional sequences}{}
    Given an exceptional sequence $\langle\calE,\calF\rangle$, we get mutated
    exceptional sequences
    \begin{equation*}
        \langle\mL_\calE\calF,\calE\rangle
            = \langle\calE,\calF\rangle
            = \langle\calF,\mR_\calF\calE\rangle.
    \end{equation*}
    In particular, a full exceptional sequence
    $\langle\calE_1,\ldots,\calE_n\rangle$ gives rise to mutated full
    exceptional sequences
    \begin{equation*}
        \langle\calE_1,\ldots,
            \calE_{i-1},\mL_{\calE_i}\calE_{i+1},\calE_i,\calE_{i+2},\ldots,
            \calE_n\rangle
    \end{equation*}
    and
    \begin{equation*}
        \langle\calE_1,\ldots,
            \calE_{i-2},\calE_i,\mR_{\calE_i}\calE_{i-1},\calE_{i+1},\ldots,
            \calE_n\rangle.
    \end{equation*}
\end{corollary}

\begin{remark}{}{}
    This gives us a braid group action on full exceptional sequences. One
    interesting question is when this action is transitive; see
    \cite{chang2023braid}.
\end{remark}

\begin{proof}
    We prove this for the first sequence; the other one follows by a dual
    argument. Restricting to the category $\langle\calE,\calF\rangle$, we have
    that
    \begin{equation*}
        \mL_\calE:
            \lperp\langle\calE\rangle=\langle\calF\rangle
            \to\langle\calE\rangle^\perp
    \end{equation*}
    is an equivalence, so $\mL_\calE\calF$ generates
    $\langle\calE\rangle^\perp$. To see that it is exceptional, note that
    \begin{equation*}
        \RHom(\mL_\calE\calF,\mL_\calE\calF)
            = \RHom(\calF,\mL_\calE\calF)
            = \RHom(\calF,\calF)
    \end{equation*}
    since the term in the cone defining $\mL_\calE\calF$ is
    $\RHom(\calE,\calF)\otimes\calE
    \in\lperp\langle\mL_\calE\calF\rangle
    =\langle\calF\rangle^\perp$.
\end{proof}

\begin{example}[label=ex:tangentbundle]{}{}
    Consider the Beilinson exceptional sequence
    $D^b(\P^n)=\langle\O,\O(1),\ldots,\O(n)\rangle$. We have
    \begin{align*}
        \mL_\O\,\O(1)
            &= \cone\bigl(\RG(\O(1))\otimes\O\to\O(1)\bigr) \\
            &= \cone(\O^{n+1}\to\O(1))
            = \Omega(1)[1]
    \end{align*}
    and
    \begin{align*}
        \mR_{\O(1)}\O
            &= \cone\bigl(\O\to\RG(\O(1))^\vee\otimes\O(1)\bigr)[-1] \\
            &= \cone(\O\to\O(1)^{n+1})[-1]
            = \calT[-1]
    \end{align*}
    by the Euler sequence
    \begin{gather*}
        0\to\Omega(1)\to\O^{n+1}\to\O(1)\to0, \\
        0\to\O\to\O(1)^{n+1}\to\calT\to0.
    \end{gather*}
    Hence we get full exceptional sequences
    \begin{equation*}
        D^b(\P^n)
            = \langle\Omega(1),\O,\O(2),\ldots,\O(n)\rangle
            = \langle\O(1),\calT,\O(2),\ldots,\O(n)\rangle.
    \end{equation*}
\end{example}

The mutation functors also allow us to view admissible categories in the following way: any object $E$ fits into a triangle surrounded by an object of $\calA^\perp$ on the right and an object of $\calA$ on the left, and similarly for $\lperp\calA$.

\begin{corollary}{}{}
    If $\calA$ is admissible and $\calA^\perp=0$, then $\calA=\calD$.
\end{corollary}

\begin{proof}
    If $\calA^\perp=0$, then $\mL_\calA(F)=0$ and hence $F\simeq \iota_* \iota^! F\in \calA$.
\end{proof}

\begin{corollary}{Autoequivalences and mutations}{Autoequivalences and mutations}
    If $\Phi$ is any autoequivalence of $\calD$, then $\Phi \circ \mL_\calA = \mL_{\Phi\calA} \circ \Phi$, and similarly for $\mR_\calA$.
\end{corollary}
\begin{proof}
    We first note that this obviously holds on $\calA \subset \calD$, where both sides vanish. On $\lperp{}\calA$, we have the following diagram:
    \begin{center}
    \begin{tikzcd}
        \lperp{}\calA \arrow[d, "\Phi"'] \arrow[r, "\mL_\calA"] & \calA ^\perp \arrow[d, "\Phi"]        \\
        \Phi(\lperp{}A) \arrow[d, "\simeq"']                    & \Phi(\calA^\perp) \arrow[d, "\simeq"] \\
        \lperp{} \Phi\calA \arrow[r, "\mL_{\Phi\calA}"']         & (\Phi\calA)^\perp
        \end{tikzcd}
    \end{center}

    So again everything commutes as it should (since autoequivalences commute with taking orthogonal complements).

    Finally, any $X\in \calD$ fits into an exact triangle $A\rightarrow X \rightarrow B$, with $A\in\calA, B\in \lperp{}\calA$. Then, we can apply $\Phi \circ \mL_\calA $ respectively $\mL_{\Phi\calA} \circ \Phi$ to this triangle. Since both of these vanish on $\calA$ we then see that $\Phi\mL_\calA B \simeq \Phi \mL_\calA X $ while also $\mL_{\Phi\calA} \Phi B \simeq \mL_{\Phi\calA} \Phi X$. We know the result holds on $B\in \lperp{}\calA$ so it holds for $X$ as well.

\end{proof}

\begin{lemma}{Serre functors of admissible subcategories}{Serre functors of admissible subcategories}
    An admissible subcategory $\calA\subset \calD$ admits a Serre functor if $\calD$ does, and it is given by $$\calS_\calA = \iota^! \circ \calS_\calD \circ \iota_*$$
    Moreover, $\calA^\perp$ is also admissible and its Serre functor can be described as $$\calS_{\calA^\perp}=\calS_\calD \circ \mR_\calA|_{\calA^\perp}$$
  \end{lemma}
  \begin{proof}
    The first of these follows by a straightforward chase around adjunctions. The second one follows since given $X,Y \in \calA^\perp,$ then $$\Hom(X, \calS_\calD \mR_\calA Y)\simeq \Hom(\mR_\calA Y, X)^*\simeq \Hom(\mL_\calA \mR_\calA Y, X)^*\simeq \Hom(Y,X)^*$$where we use the fact that $\mL \mR =id$ on $\calA^\perp$ and furthermore the exact triangle $\iota_* \iota^! \mR_\calA Y \rightarrow \mR_\calA Y \rightarrow \mL_\calA \mR_\calA Y$, combined with the fact that $\Hom(\iota_* \iota^! \mR_\calA Y,X)=0, X\in \calA^\perp$ tells us that $\Hom(\mR_\calA Y, X)\simeq \Hom(\mL_\calA \mR_\calA Y, X)$.
 \end{proof}


%\begin{remark}{}{}
     %Given an exceptional $E$, can think of it as an object on $D^b(\pt\times X)$ and its induced map $FM_{E}:D^b(\pt)\to D^b(X)$ is fully faithful, sending a vector space $V\mapsto V\otimes E$. The endomorphisms on both sides agree: $\Ext^\bullet(k,k)=k=\Ext^\bullet(E,E)$. The right adjoint is given by the kernel $E^\vee$ so it sends $F\mapsto \pi_* (E^\vee \otimes F)=\Ext^\bullet(E,F)$, since derived projection is taking cohomology. The counit provides us with an evaluation map $$E\otimes \Ext^\bullet(E,F)\xrightarrow{\ev}F$$In general, objects of $D^b(X)$ correspond to functors from $D^b(\pt)$ and exceptionals correspond to fully faithful functors. We will see that spherical twists come from taking cones on the (co)evaluation maps.
%\end{remark}

\begin{definition}{Semi-orthogonal decomposition}{Semi-orthogonal decomposition}
    A semiorthogonal decomposition of $\calD$ is a sequence of admissible subcategories $\calA_{i}$ such that $\calA_{j}\perp \calA_{i}, i<j$ and they generate $\calD$.
\end{definition}

We finish by mentioning a few facts about semiorthogonal decompositions coming from left resp. right admissible subcategories:

\begin{proposition}{Associated semiorthogonal decompositions}{Associated semiorthogonal decompositions}
    Suppose $\alpha: \calA \rightarrow \calD$ is left-admissible with left adjoint $\alpha^*$. Then there is a semiorthogonal decomposition $$\calD = \langle \alpha(\calA), \ker \alpha^*\rangle $$
    Similarly, if $\beta:\mathcal{B}\rightarrow \calD$ is right-admissible with right adjoint $\beta^!$, we have a semiorthogonal decomposition $$\calD=\langle \ker \beta^!, \beta(\mathcal{B})\rangle$$
\end{proposition}

\begin{proof}
    The proof is similar to the one about admissible subcategories, except we only have one adjoint. We only cover the right admissible case: again fit an object $E$ into a triangle $$\beta \beta^! E \rightarrow E \rightarrow \mathrm{cone}$$

    We show that the cone lies in $\ker \beta!$ as follows: applying $\beta^!$ to the triangle, we need to consider $$\beta^! \beta \beta^!E \rightarrow \beta^! E\rightarrow \beta^! \mathrm{cone}$$

    We would like to show the first map is an isomorphism, from which the result follows and hence we can generate any object as the cone of a morphism between objects in the s.o.d.

    This essentially follows since the composition $$\beta^!E \rightarrow \beta^! \beta \beta^! E \rightarrow \beta^!E$$

    is the identity, by the fact that $\beta^!$ is right adjoint to $\beta$, as well as the condition that $\beta^!\beta=1$.
\end{proof}

\begin{example}{Relative exceptional SOD}{Relative exceptional SOD}
Let $f:X\rightarrow Y$ be a map and $\calE$ be a relative exceptional object. Then we have an adjunction \[\begin{tikzcd}
    {D^b(Y)} & {D^b(X)}
	\arrow[""{name=0, anchor=center, inner sep=0}, "{\calE \otimes f^*-}", curve={height=-12pt}, from=1-1, to=1-2]
	\arrow[""{name=1, anchor=center, inner sep=0}, "{f_* \lHom (\calE ,-)}", curve={height=-12pt}, from=1-2, to=1-1]
	\arrow["\dashv"{marking, allow upside down}, draw=none, from=0, to=1]
\end{tikzcd}\]
and moreover the composition is the identity, by assumption that $\calE$ is relative exceptional: $$f_* \lHom(\calE, \calE \otimes f_* \calF)=f_*\lHom(\calE, \calE)\otimes \calF = \calF$$

Hence, we can apply \ref{prop:Associated semiorthogonal decompositions} and get the following s.o.d.:

$$D^b(Y)=\langle \ker f_*\lHom(\calE,-), \calE\otimes f^*D^b(X)\rangle$$
\end{example}

\subsubsection{The projective bundle and blowup formulas}
We now focus on two important semiorthogonal decompositions associated to projective bundles and blowups.

\paragraph*{Projective bundle formula}
\begin{proposition}[label=prop:projbundleformula]{Projective bundle formula for derived categories}{Projective bundle formula for derived categories}
    Given the projectivisation $p:\P\calE\to B$ of a vector bundle of rank $r+1$ we have a semiorthogonal decomposition
    \begin{equation*}
        D^b(\P(\calE)) = \langle p^*D^b(B),p^*D^b(B)\otimes\O_p(1),\ldots,p^*D^b(B)\otimes\O_p(r)\rangle
    \end{equation*}
\end{proposition}

\begin{proof}
    \textbf{Step 1}: orthogonality.

    We see that $p^*$ is fully faithful:
    \begin{equation*}
        \RHom(p^*F,p^*G) = \RHom(F,p_*(p^*G\otimes\O_{\P(E)})) = \RHom(F,G)
    \end{equation*}
    since $p_*\O_{\P(E)}=\O_X$. So each of the bits is admissible.

    We also need to show there are no Homs from right to left, which follows more or less since $H^\bullet(\P^r, \O(-i))=0, i=-1,...,-r$.
    \begin{gather*}
        \RHom(p^*F\otimes\O(i),p^*G\otimes\O(j))
            = \RHom(F,p_*(p^*G\otimes \O(j-i))) = \\
            = \RHom(F,G\otimes p_*\O(j-i)) = 0, \text{ since } j-i\in \{-1,-2,...,-r\}
    \end{gather*}
for $i<j$. We could also have argued using \ref{ex:Relative exceptional SOD} by applying it many times to the relative exceptional line bundles $\O(i)$: the fact they are relative exceptional follows from the fact that $p_*\O_{\P\calE}=\O_B$ for a projective bundle.

\textbf{Step 2}: We now show generation. We do this by resolving the diagonal just as we did for $\P^n$, but now for a projective bundle: we have an Euler sequence as before $$0\rightarrow \O\rightarrow p^* \calE \otimes \O_{p}(1)\rightarrow \calT_{\P\calE/B}\rightarrow 0$$since $$\calT_p:=\calT_{\P\calE/B}\simeq \lHom(\O_{p}(-1),\mathcal{Q})\simeq \frac{\lHom(\O_{p}(-1),\mathcal{Q}\oplus \O_{p}(-1))}{\lHom (\O_{p}(-1), \O_{p}(-1))}\simeq \frac{\lHom(\O_{p}(-1), p^* \calE)}{ \O}$$

We can twist by $\O_p(-1)$ to get the sequence $$0\rightarrow \O_p(-1)\rightarrow p^* \calE \rightarrow \mathcal{Q}\rightarrow 0$$

As in the proof of Beilinson's theorem \ref{th:Beilinson's theorem}, this allows us to realize the diagonal on the projective bundle as the zero locus of a section of $$\lHom(\O_p(-1)\boxtimes \O, \O\boxtimes \mathcal{Q})\simeq \O_p(1) \boxtimes \mathcal{Q}$$
and hence produce a Koszul resolution $$0\rightarrow \bigwedge\nolimits^r \O_p(-1) \boxtimes \mathcal{Q}^*\dots\rightarrow \bigwedge\nolimits^i \O_p(-1) \boxtimes \mathcal{Q}^*\rightarrow \dots \rightarrow \O_p(-1) \boxtimes \mathcal{Q}^*\rightarrow \O_{\P\calE \times \P\calE}\rightarrow \O_{\Delta}\rightarrow 0$$

In the exact analogous way, this shows that the pieces generate the full category.
\end{proof}

\begin{remark}{Relative canonical bundle of projective bundle}{Relative canonical bundle of projective bundle}

        More generally \cite[Remark~13.36]{Wedhorn}, given a projective bundle $p:\P\calE \rightarrow B$ which can be defined as a relative Proj of a sheaf of algebras $\underline{\Proj}\, \mathrm{Sym}^\bullet \calE^\vee$, we have the following fact: $$p_* \O_{\P\calE}(d)=\mathrm{Sym}^d \calE^\vee$$ 
    
        In particular, $p_*\O(-1)=\calE$. Recall the relative Euler sequence: $$0\rightarrow \O\rightarrow p^* \calE \otimes \O_{\pi}(1)\rightarrow \calT_{\P\calE/B}\rightarrow 0$$
        %since $$\calT_\pi:=\calT_{\P\calE/B}\simeq \lHom(\O_{\pi}(-1),\mathcal{Q})\simeq \frac{\lHom(\O_{\pi}(-1),\mathcal{Q}\oplus \O_{\pi}(-1))}{\lHom (\O_{\pi}(-1), \O_{\pi}(-1))}\simeq \frac{\lHom(\O_{\pi}(-1), \pi^* \calE)}{ \O}$$  
    
        By twisting by $-1$ and taking determinants, we see that $$p^* \det \calE \simeq \O_p(-1) \otimes \det(\calT_p\otimes \O_p(-1))\simeq \O_p(-r) \otimes \det(\calT_p)$$and so $$\omega_p\simeq p^* \det \calE^\vee \otimes \O_{p}(-r)$$
    
        Now, to understand the derived pushforward of the relative canonical bundle, we can use Grothendieck-Verdier duality: $$\O_B=\lHom (p_* \O_{\P\calE}, \O_B)\simeq p_* \lHom(\O_{\P\calE}, p^!\O_B)\simeq p_* \omega_p[r]$$
        Hence,  $p_* \omega_p\simeq \O_B[-r]$, and so we conclude that $p_* \O_p(-r) \simeq \det \calE[-r]$. 
\end{remark}

This allows us to talk about blowups, since over the exceptional locus they are given by a projective bundle.

\paragraph*{Twisted projective bundle formula}\label{Twisted projective bundle formula}
We quickly mention a twisted version of the projective bundle formula due to Bernardara \cite{bernardara_semiorthogonal_2005}. In general, a projective space bundle $\P^r \rightarrow P\rightarrow B$ need not be a projectivized vector bundle - it is only locally so. Let's say that over an open cover $\mathcal{U}$, $P|_U\simeq \P \calE_U$. Then, on overlaps we have isomorphisms $\P\calE_U \simeq \P \calE_V$ but in order to lift this to an isomorphism of vector bundles $\calE_U \simeq \calE_V$, we need a choice of lift from $\P\mathrm{GL}(r,U\cap V)$ to $\mathrm{GL}(r,U\cap V)$. This choice is not canonical, and hence on triple overlaps we get an obstruction cocycle $\alpha \in \Gamma(\C^\times, U\cap V \cap W)$ i.e. an element of $H^2(B,\O_B^\times)$. This element is called the \emph{Brauer class of the Brauer-Severi variety }$P$.

What we get is a twisted vector bundle $(\calE, \alpha)$ such that the gluing cocycles with values in $\mathrm{GL}(r)$ satisfy $$\psi_{12} \circ \psi_{23}=\alpha \psi_{13}$$

This is an honest vector bundle precisely when $\alpha=1$.

We can rephrase this via the long exact sequence associated to $$0\rightarrow \C^\times \rightarrow \mathrm{GL}(r)\rightarrow \P \mathrm{GL}(r)\rightarrow 0$$

The Brauer-Severi variety $P$ gives us a class $[P]\in H^1(B, \mathrm{GL}(r))$ whose boundary map is the Brauer class $\alpha$.

Essentially all the nice compatibilities between derived functors carry over to the twisted setting, following Caldararu's PhD thesis \cite{CaldararuThesis}.

\begin{proposition}{Bernardara twisted decomposition \cite{bernardara_semiorthogonal_2005}}{Bernardara twisted decomposition}
    If $P\rightarrow B$ is a Brauer-Severi variety, then there is a semiorthogonal decomposition $$D^b(P)=\langle \calD_0,..., \calD_r\rangle$$
    where each $\calD_k\simeq D^b(B,\alpha^{-k})$.
\end{proposition}

The pieces of the semiorthogonal decomposition are essentially $p^*D^b(B)\otimes \O_p(k)$. The proof boils down to using the local-to-global spectral sequence to show there are no homs from right to left, as well as using a resolution of the diagonal as before to show they generate.
%Given a blowup $f:\tilde{X}\to X$, we still have that $f^*$ is fully faithful, so we have an admissible subcategory $f^*D^b(X)$. Suppose the centre of the blowup is $Z$, with exceptional locus $E\xrightarrow{j}\tilde X$. Then we also have a fully faithful functor $\varphi_t:D^b(Z)\to D^b(\tilde{X})$ for each $t\in\Z$, given by sending $F$ to $j_*(f^*F\otimes\O_E(tE))$. This is a Fourier-Mukai transform with kernel $\O_E(tE)$.

%*Exercise:* find the right adjoint of this and show it composes to identity.

%\begin{theorem}{Bondal-Orlov blowup formula}{Bondal-Orlov blowup formula}
%    \begin{equation*}
%        D^b(\tilde{X}) = \langle\varphi_{k-1}D^b(Z),\ldots,\varphi_1D^b(Z),f^*D^b(X)\rangle
%    \end{equation*}
%\end{theorem}
%\todo{Prove this, e.g. by looking into Kuznetsov's notes Derived categories view on rationality problems}

%Compare with the formula for cohomology!

\paragraph*{Blowup formula}

Let $Z$ be a smooth subvariety of $Y$ of codimension $c$. Consider the blow up $X = \mathrm{Bl}_{Z}(Y)$ which fits into the fibered square below, with $E$ the projectivisation $\mathbb{P} (\mathcal{N})$ of the normal bundle $\mathcal{N}_{Z/Y}$.

\[\begin{tikzcd}[column sep=2.25em]
	{E } & X \\
	Z & {Y}
	\arrow["p"', from=1-1, to=2-1]
	\arrow["\pi", from=1-2, to=2-2]
	\arrow["i", hook, from=1-1, to=1-2]
	\arrow["j"', hook, from=2-1, to=2-2]
\end{tikzcd}\]

Note that the blow-up has projective fibres, so $\pi_{*}\mathcal{O}_{X}= \mathcal{O}_Y$. Taking the fibre product $X \times Z$, for integer $k$ we can consider $\mathcal{O}_{E}(kE)$ as an element of the product, so can define the collection of Fourier Mukai transforms indexed by the integers, $$\Phi_{k}:= i_{*}\left( \mathcal{O}_{E}(kE)\otimes p^{*}(- ) \right): D^b(Z)\to D^b(X).$$

By properties of Fourier Mukai transforms, $\Phi_k$ has left and right adjoints. Moreover, $\Phi_k$ is fully faithful, which can be seen by verifying that the composition of $\Phi_k$ with its right adjoint $\Phi_{k}^{!}:= p_{*}(\mathcal{O}_{E}(-kE)\otimes i^{!}(-))$ is isomorphic to the identity on $D^b(Z)$. Moreover the right adjoint allows us to see the $D^b(Z)$ as an admissible subcategory of $D^b(X)$ under $\Phi_k$. Consider the fact from~\cite*{Huybrechts}.

\begin{proposition}[label=prop:ffpullback]{}{}
    Suppose $f: S \to T$ is a projective morphism of smooth projective varieties. If $f_{*}\mathcal{O}_{S} = \mathcal{O}_T$ , then the pullback $$f^{*}: D^b(T)\to D^b(S)$$is fully faithful and induces an equivalence of $D^b(T)$ with an admissible triangulated subcategory of $D^b(S)$.
\end{proposition}

Hence we can see $D^b(Y)$ as an admissible subcategory of $D^b(X)$ under $\pi^*$. Denote the images $\mathcal{D}_{-k}=\mathrm{Im}(\Phi_{k})$ and $\mathcal{D}_{0}= \pi^{*}D^b(Y)$ which are admissible in $D^b(X)$.

\begin{lemma}{}{}
    The sequence $$\mathcal{D}_{-c+1},\dots,\mathcal{D}_{-1},\mathcal{D}_0$$is a sequence of semiorthogonal admissible subcategories in $D^b(X)$.
\end{lemma}


\begin{proof}
We already know that each component is admissible. To see that they are orthogonal in the left direction, take integers $-c+1 \leq l <k <0$, and $\mathcal{E}, \mathcal{F}\in D^b(Z)$. Let $i^*$ be the left adjoint to $i_*$. Then we have $$\mathrm{Hom}(i_{*}\left( p^{*}\mathcal{F}\otimes \mathcal{O}_{E}(-kE) \right), i_{*}\left( p^{*}\mathcal{E}\otimes \mathcal{O}_{E}(-lE) \right)  ) \simeq \mathrm{Hom}(i^{*}i_{*}p^{*}\mathcal{F}, p^{*}\mathcal{E}\otimes \mathcal{O}_{E}((k-l)E))$$

From \cite[Corollary~11.4]{Huybrechts}, we have a distinguished triangle $$p^{*}\mathcal{F}\otimes  \mathcal{O}_{E}(-E)[1]\to i^{*}i_{*}p^{*}\mathcal{F}\to p^{*}\mathcal{F}\to p^{*}\mathcal{F}\otimes \mathcal{O}_E(-E)[2]$$
So using the projection formula the above is equal to
\begin{align*}
\mathrm{Hom}(p^{*}\mathcal{F},p^{*}\mathcal{E} \otimes  \mathcal{O}_{E}((k-l)E)) &\simeq \mathrm{Hom}(\mathcal{F},p_{*}(p^{*}\mathcal{E} \otimes  \mathcal{O}_{E}((k-l)E)))  \\
&\simeq \mathrm{Hom}(\mathcal{F},\mathcal{E} \otimes p_{*} \mathcal{O}_{E}((k-l)E))
\end{align*}

Since the fibres of $p$ are $\mathbb{P}^{c-1}$, then if $-c+1<l-k<0$, $p_{*}\mathcal{O}_{E}((k-l)E)=0$. Hence the above is equal to 0, so $\mathrm{Hom}(\mathcal{F},\mathcal{E} \otimes p_{*} \mathcal{O}_{E}((k-l)E)) =0$, and $\mathcal{D}_{l}\subset \mathcal{D}_k^\perp$.

Now take $\mathcal{E}\in D^b(Y)$, and $\mathcal{F}\in D^b(Z)$. Similarly, we have
\begin{align*}
\mathrm{Hom}(\pi^{*}\mathcal{E},i_{*}(p^{*}\mathcal{F}\otimes \mathcal{O}_{E}(-lE))) &\simeq \mathrm{Hom}(\mathcal{E},\pi_{*}i_{*}(p^{*}\mathcal{F}\otimes \mathcal{O}_{E}(-lE))) \\
&\simeq \mathrm{Hom}(\mathcal{E},j_{*}p_{*}(p^{*}\mathcal{F}\otimes \mathcal{O}_{E}(-lE)))  \\
&\simeq \mathrm{Hom}(\mathcal{E},j_{*}(\mathcal{F}\otimes p_{*}\mathcal{O}_{E}(-lE)))
\end{align*}
but as $-c+1\leq l<0$, $p_{*}\mathcal{O}_{E}(-lE)=0$. Hence $\mathcal{D}_{l}\subset D_{0}^\perp$. We have shown that the collection $\mathcal{D}_{-c+1},\dots,\mathcal{D}_0$ is semiorthogonal.
\end{proof}

\begin{theorem}[label=thm:blowupformula]{Bondal-Orlov blowup formula}{Bondal-Orlov blowup formula}
    Let $Z$ be a smooth subvariety of $Y$ of codimension $c$, and $X = \mathrm{Bl}_{Z}(Y)$. Then we have a semi orthogonal decomposition $$ D^b(X) = \left< \mathcal{D}_{-c+1},\dots,\mathcal{D}_{-1}, \mathcal{D}_0 \right> $$
\end{theorem}

The proof follows directly from the lemma, and the assertion that this collection does indeed generate $D^b(X)$, which is shown in \cite{orlov_projective_1993}.

\subsection{Spherical objects and spherical twists}

\begin{definition}{Spherical object}{}
    A \emph{spherical object} is one which
    \begin{enumerate}
        \item ``Is self-dual with a twist'': $\calE\otimes\omega_X\simeq\calE$, and
        \item ``Has the cohomology of a sphere'':
            \begin{equation*}
                \Ext^i(\calE,\calE)
                    = \Hom_{D^b(X)}(\calE,\calE[i])
                    = \begin{cases*}
                        \C & if $i=0,\dim X$ \\
                        0 & otherwise.
                    \end{cases*}
            \end{equation*}
    \end{enumerate}
\end{definition}

Note that 1. implies $\Hom_{D^b(X)}(\calE,\calE[i])\simeq\Hom_{D^b(X)}(\calE,\calE[\dim X-i])^*$
by Serre duality. It follows that $\calE^\vee$ is also spherical because
$\RlHom(\calE,\O_X)\otimes\omega_X\simeq\RlHom(\calE\otimes\omega_X,\O_X)$.

For any $\calE$ we have the trace map $\calE^\vee\otimes\calE\to\O_X$, defined
in the usual way for locally free sheaves and then extended to the derived
category by locally free resolutions. This induces a map
$q^*\calE^\vee\otimes p^*\calE\to\O_\Delta$ over $X\times X$, since
$q^*\calE^\vee\otimes p^*\calE$ restricts to $\calE^\vee\otimes\calE$ on the
diagonal.

\begin{definition}{Spherical twist}{}
    The \emph{spherical twist} along $\calE$ is the Fourier-Mukai transform
    $T_\calE:D^b(X)\to D^b(X)$ with kernel given by the cone of the map
    $q^*\calE^\vee\otimes p^*\calE\to\O_\Delta$.
\end{definition}

\begin{proposition}[label=prop:twistformula]{}{}
    The spherical twist of $\calF$ along $\calE$ can be computed as the cone on
    the evaluation map:
    \begin{equation}\label{eqn:twistformula}
        T_\calE(\calF)=\cone(\dR\Hom(\calE,\calF)\otimes_k\calE\to\calF).
    \end{equation}
\end{proposition}

\begin{remark}{}{}
    Taking cones is not functorial, so the spherical twist functor is only
    defined up to non-unique isomorphism, and the formula in
    Proposition~\ref{prop:twistformula} does not a priori define a functor. The
    spherical twist gives a functorial way of choosing the cone in this formula
    by moving the non-uniqueness into the kernel of a Fourier-Mukai transform.
\end{remark}

\begin{proof}
    Since the Fourier-Mukai transform is exact in the kernel the twist of
    $\calF$ is the cone on $\Phi_{q^*\calE^\vee\otimes p^*\calE}(\calF)
    \to\Phi_{\O_\Delta}(\calF)=\calF$, and
    \begin{align*}
        \Phi_{q^*\calE^\vee\otimes p^*\calE}(\calF)
            &\simeq p_*(q^*(\calE^\vee\otimes\calF)\otimes p^*\calE) \\
            &\simeq p_*q^*(\calE^\vee\otimes\calF)\otimes\calE \\
            &\simeq \dR\Hom(\calE,\calF)\otimes_\C\calE
    \end{align*}
    by the projection and base change formulas.
\end{proof}

\begin{remark}{}{}
    If $\calE$ is an exceptional object, then this is precisely the formula for
    the left mutation $\mL_\calE$. In this case we can view
    \eqref{eqn:twistformula} as a kind of Gram-Schmidt formula, projecting
    $\calF$ onto $\langle\calE\rangle^\perp$ with respect to the ``inner
    product'' $\RHom(-,-)$ by ``subtracting off''
    $\RHom(\calE,\calF)\otimes_\C\calE$.
\end{remark}

\begin{proposition}{Spherical twists are autoequivalences}{}
    If $\calE$ is a spherical object then
    $T_\calE:D^b(X)\xrightarrow\sim D^b(X)$ is an autoequivalence.
\end{proposition}

\begin{proof}
    The left and right adjoints are given by tensoring the kernel with
    $q^*\omega_X[\dim X]$, respectively $p^*\omega_X[\dim X]$, and so coincide
    because $\calE\otimes\omega_X\simeq\calE$ and
    $\calE^\vee\otimes\omega_X\simeq\calE^\vee$, recalling that $\calE^\vee$ is
    also spherical. So it suffices to show that $T_\calE$ is fully faithful.
    (Then taking cones on the counit decomposes $D^b(X)$ into the essential
    image and the objects sent to zero by the adjoint, but $D^b(X)$ is
    indecomposable for connected $X$.)

    Note that $\{\calE\}\cup\calE^\perp$ is a spanning class (tautological on
    one side, and follows from Serre duality on the other), so it suffices to
    check $\Hom(T_\calE(\calE),T_\calE(\calE))$ and
    $\Hom(T_\calE(\calF),T_\calE(\calG))$ for $\calF,\calG\in \calE^\perp$. But
    $T_\calE$ restricts to the identity on $\calE^\perp$ by
    Proposition~\ref{prop:twistformula}, and $T_\calE(\calE)=\calE[1-\dim X]$
    with $\id_\calE$ mapping to $\id_{\calE[1-\dim X]}$ from
    $\RHom(\calE,\calE)=\C\oplus\C[\dim X]$, again using
    Proposition~\ref{prop:twistformula}.
\end{proof}

%\begin{proposition}{Action of spherical twist on cohomology}{}
%    \begin{equation*}
%        T_{\calE}^H(v) = v-\langle v(\calE),v\rangle v(\calE)
%    \end{equation*}
%    where we have the Mukai pairing.
%\end{proposition}

%\begin{proof}
%The proof is as follows:
%\begin{equation*}
%    \text{kernel} = \cone(q^*\calE^\lor\otimes p^* \calE\to\O_{\Delta})
%\end{equation*}
%The Mukai vector of this is
%\begin{equation*}
%    \Theta = v(\O_{\Delta})-q^*v(\calE^\lor)p^*v(\calE)
%\end{equation*}
%but $v(\O_{\Delta})=[\Delta]$. So the action on cohomology is
%\begin{gather*}
%    T_{\mathcal{E}}^H(v)
%        = p_{*}(\Theta \cup q^*v)
%        = v-p_{*}(p^*v(\mathcal{E})q^*v(\mathcal{E}^\lor)q^* v) = \\
%        = v-v(\mathcal{E)}\int _{X} v.v(\mathcal{E}^\lor)
%\end{gather*}
%\end{proof}
