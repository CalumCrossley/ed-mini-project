\usepackage{graphicx}
\usepackage{tcolorbox}
%\usepackage[dvipsnames]{xcolor}
\tcbuselibrary{theorems}
\usepackage{mdframed}
\usepackage{amsmath, amsfonts, amsthm, amscd, bbm, thmtools, hyperref}
\usepackage{pgfplots}
\usepackage[all]{xy}
\usepackage{mathtools}
\pgfplotsset{width=\columnwidth,compat=1.13}
\usepackage{float}
\usepackage{titling}
\usepackage{caption}
\usepackage{subcaption}
\usepackage{tikz-cd}
\usepackage{euscript}
\usepackage{mathtools}
\usepackage{stmaryrd}
\usetikzlibrary{decorations.markings}
\usetikzlibrary{decorations.pathmorphing,shapes}
\usetikzlibrary{patterns, patterns.meta}
\usetikzlibrary {intersections,pgfplots.fillbetween} 
\usetikzlibrary{shapes.geometric}
\usetikzlibrary{fadings}
\usetikzlibrary{calc}
\usepackage{hyperref}

\DeclareMathOperator{\image}{im} 
\DeclareMathOperator{\ind}{ind}
\DeclareMathOperator{\cokernel}{coker} 
\DeclareMathOperator{\Hom}{Hom}
\DeclareMathOperator{\lExt}{\mathcal{E}\mathit{xt}} %local Ext
\DeclareMathOperator{\lHom}{\mathcal{H}\mathit{om}} %local Hom
\DeclareMathOperator{\RHom}{\mathbf{R}Hom} %derived usual Hom
\DeclareMathOperator{\RlHom}{\mathbf{R}\mathcal{H}\mathit{om}} %derived local Hom
\DeclareMathOperator{\RG}{\mathbf{R}\Gamma} %derived global sections

\DeclareMathOperator*{\ulalim}{\underleftarrow\lim}
\graphicspath{{images/}}
\setcounter{section}{-1}

\definecolor{maincolor}{RGB}{122, 17, 49}
\hypersetup{linkcolor = maincolor, linkbordercolor=maincolor, citebordercolor = maincolor} %colorlink removes boxes

\numberwithin{equation}{section}

%\newtheorem{theorem}{Theorem}[subsection]
%\numberwithin{theorem}{subsection}% Reset theorem counter with every section

%\theoremstyle{definition}
%\newtheorem{definition}[theorem]{Definition}
%\numberwithin{definition}{section}% Reset theorem counter with every section

%\mdfdefinestyle{theoremstyle}{linecolor=Emerald, linewidth=2pt, frametitlerule=true, frametitlebackgroundcolor=gray!20, innertopmargin=\topskip
%} \mdtheorem[style=theoremstyle]{definition}{Definition}

%\mdfdefinestyle{theoremstyle}{ linecolor=Aquamarine, linewidth=2pt, frametitlerule=true, frametitlebackgroundcolor=gray!20, innertopmargin=\topskip
%} \mdtheorem[style=theoremstyle]{theorem}{Theorem}

 
\newtcbtheorem[number within=section]{theorem}{Theorem} {theorem style = plain, colback=maincolor!30!white, coltitle=black, colframe=white, fonttitle = \upshape\bfseries, fontupper=\itshape}{th}

\newtcbtheorem[use counter from = theorem]{definition}{Definition} {theorem style = plain, colback=maincolor!30!white, coltitle=black, colframe=white, fonttitle = \upshape\bfseries,  fontupper=\itshape}{dfn}

\newtcbtheorem[use counter from = theorem]{proposition}{Proposition} {theorem style = plain, colback=maincolor!30!white, coltitle=black, colframe=white, fonttitle = \upshape\bfseries,  fontupper=\itshape}{prop}

\newtcbtheorem[use counter from = theorem]{cor}{Corollary} {theorem style = plain, colback=maincolor!30!white, coltitle=black, colframe=white, fonttitle = \upshape\bfseries,  fontupper=\itshape}{cor}

\newtcbtheorem[use counter from = theorem]{lemma}{Lemma} {theorem style = plain, colback=maincolor!30!white, coltitle=black, colframe=white, fonttitle = \upshape\bfseries,  fontupper=\itshape}{lemma}

\newtcbtheorem[no counter]{ob}{Observation} {theorem style = plain, colback=maincolor!30!white, coltitle=black, colframe=white, fonttitle = \upshape\bfseries,  fontupper=\itshape}{ob}

\newtcbtheorem[no counter]{example}{Example} {theorem style = plain, colback=maincolor!10!white, coltitle=black, colframe=white, fonttitle = \upshape\itshape}{ex}

\newtcbtheorem[no counter]{technicalities}{Technicalities} {theorem style = plain, colback=maincolor!10!white, coltitle=black, colframe=white, fonttitle = \upshape\itshape}{tech}

\newtcbtheorem[no counter]{remark}{Remark} {theorem style = plain, colback=maincolor!10!white, coltitle=black, colframe=white, fonttitle = \upshape\itshape}{rem}

%\newtheorem{lemma}[theorem]{Lemma}
%\numberwithin{lemma}{section}% Reset theorem counter with every section
%\newtheorem{prop}[theorem]{Proposition}
%\numberwithin{prop}{section}% Reset theorem counter with every section
%\newtheorem{cor}[theorem]{Corollary}
%\numberwithin{cor}{section}% Reset theorem counter with every section


\newcommand{\half}{\frac{1}{2}}
\newcommand{\dbyd}[2]{\frac{\partial #1}{\partial #2}}
\newcommand{\vsig}{\vec{\sigma}}
\newcommand{\R}{\mathbb{R}}
\newcommand{\C}{\mathbb{C}}
\newcommand{\norm}[1]{\left\lVert #1 \right\rVert}
\newcommand{\comp}{\circ}
\renewcommand{\O}{\mathcal{O}}
\renewcommand{\u}{{u_1}}
\renewcommand{\v}{{u_2}}
\newcommand{\laplace}{\Delta}
\renewcommand{\line}[2]{{\overrightarrow{#1,#2}}}

\newcommand{\todo}[1]{{\color{red}\textbf{TODO!: #1}}}

\setlength{\parindent}{0em}
\setlength{\parskip}{1em}

\renewcommand{\.}{\,.}