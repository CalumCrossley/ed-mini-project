
\subsection{Derived categories of singularities}

When considering smooth projective varieties, we made use of the existence of
bounded locally free resolutions (Hilbert's syszygy theorem), e.g. in defining
derived pullbacks and tensor products. If we want to look at singular varieties,
the failure of this property is an important point.

\begin{theorem}{Serre}{}
    A commutative local ring $A$ is regular iff it has finite global dimension,
    i.e. there is a global bound on the length of projective resolutions of
    $A$-modules.
\end{theorem}

Motivated by this, we make the following definition.

\begin{definition}{}{}
    Suppose $X$ is a quasi-projective variety. A complex of sheaves on $X$ is
    perfect if it is quasi-isomorphic to a bounded complex of finite type
    locally free sheaves. This defines a triangulated subcategory
    $\DPerf(X)\subseteq D^b(X)$ by the construction of cones in terms of direct
    sums. The derived category of singularities is the quotient
    $D_\Sg(X)\coloneqq D^b(X)/\DPerf(X)$.
\end{definition}

Recall from Definition~\ref{defn:verdierquotient} that this quotient is given by
inverting all morphisms in $D^b(X)$ whose cones are perfect complexes.

\begin{remark}{}{}
    As noted earlier, for a smooth variety all bounded complexes of coherent
    sheaves are perfect, so $\DPerf(X)=D^b(X)$ and $D_\Sg(X)=0$.
\end{remark}

\begin{remark}{}{}
    The \emph{underived} category of perfect complexes
    $\Perf(X)\subseteq K(\Coh X)$ inherits a differential graded structure from
    $K(\Coh X)$. The homotopy category $H^0\Perf(X)$ given by taking 0th
    cohomology of $\Hom^\bullet(-,-)$ for morphisms is equivalent to
    $\DPerf(X)$, since % TODO

    In the smooth case, where $\DPerf(X)=D^b(X)$, this gives a
    ``dg-enhancement'' of $D^b(X)$. The richer structure of a dg-category can be
    nicer to work with than the triangulated category $D^b(X)$, e.g. functorial
    cones.
\end{remark}

\begin{proposition}{}{}
    Any object in $D_\Sg(X)$ is isomorphic to the image of some shifted sheaf.
\end{proposition}

\begin{proof}
    We may take a bounded above projective resolution of an object in $D_\Sg(X)$
    \begin{equation*}
        \begin{tikzcd}
            \cdots \ar[r] & \calP^{a-1} \ar[r] \ar[d] &
            \calP^a \ar[r] \ar[d] & \cdots \ar[r] & \calP^b \ar[d] \\
            \cdots \ar[r] & 0 \ar[r] & \calA^a \ar[r] & \cdots \ar[r] & \calA^b,
        \end{tikzcd}
    \end{equation*}
    and then the truncation
    \begin{equation*}
        \begin{tikzcd}
            \cdots \ar[r] & 0 \ar[r] & \calP^{a-1}/\calP^{a-2} \ar[d] \ar[r] &
            \calP^a \ar[r] \ar[d] & \cdots \ar[r] & \calP^b \ar[d] \\
                & \cdots \ar[r] & 0 \ar[r] &
                \calA^a \ar[r] & \cdots \ar[r] & \calA^b
        \end{tikzcd}
    \end{equation*}
    is also a quasi-isomorphism. This truncated resolution projects to
    $\calP^{a-1}/\calP^{a-2}[1-a]$ with cone quasi-isomorphic to
    $\calP^a\to\cdots\to\calP^b$, so in $D_\Sg(X)$ it is isomorphic to
    $\calP^{a-1}/\calP^{a-2}[1-a]$.
\end{proof}

\begin{example}{}{}
    Consider a non-reduced point $X=\Spec\C[t]/t^n$. By the structure theorem
    for finite type modules over a PID, coherent sheaves on $X$ are direct sums
    of the modules $\C[t]/t^i$ for $i=0,\ldots,n$. Moreover, by taking Smith
    normal forms any complex is a direct sum of shifts of these modules. Only
    $i=n$ gives a projective module, and so $D_\Sg(X)$ is generated by
    $\C[t]/t,\C[t]/t^2,\ldots,\C[t]/t^{n-1}$. Note that these all have infinite
    periodic free resolutions as $\C[t]/t^n$-modules, corresponding to the
    factorizations $t^n=t^it^{n-i}$.
\end{example}

One might hope that $D_\Sg(X)$ only depends on the local geometry of $X$ near
its singular locus, and in fact this is true; $D_\Sg(U)\simeq D_\Sg(X)$ for a
formal neighbourhood $U$ of the singular locus. See
\cite[Prop~1.14]{OrlovSingularities} and \cite[\S6]{Shipman}.

\subsection{Matrix factorizations}
