
\subsection{Derived categories of singularities}

When considering smooth projective varieties, we made use of the existence of
bounded locally free resolutions (Hilbert's syszygy theorem), e.g. in defining
derived pullbacks and tensor products. If we want to look at singular varieties,
the failure of this property is an important point.

\begin{theorem}{Serre}{}
    A commutative local ring $A$ is regular iff it has finite global dimension,
    i.e. there is a global bound on the length of projective resolutions of
    $A$-modules.
\end{theorem}

Motivated by this, we are led to the following definition due to Orlov.

\begin{definition}{}{}
    Suppose $X$ is a quasi-projective variety. A complex of sheaves on $X$ is
    perfect if it is quasi-isomorphic to a bounded complex of finite type
    locally free sheaves. This gives a triangulated subcategory
    $\DPerf(X)\subseteq D^b(X)$, by the construction of cones in terms of direct
    sums. The derived category of singularities is the quotient
    $D_\Sg(X)\coloneqq D^b(X)/\DPerf(X)$.
\end{definition}

Recall from Definition~\ref{defn:verdierquotient} that this quotient is given by
inverting all morphisms in $D^b(X)$ whose cones are perfect complexes.

\begin{remark}{}{}
    As noted above, for a smooth variety all bounded complexes of coherent
    sheaves are perfect, so $\DPerf(X)=D^b(X)$ and $D_\Sg(X)=0$.
\end{remark}

\begin{remark}{}{}
    The \emph{underived} category of perfect complexes
    $\Perf(X)\subseteq K^b(\Coh X)$ inherits a differential graded structure
    from $K^b(\Coh X)$. The homotopy category $H^0\Perf(X)$ given by taking 0th
    cohomology of $\Hom^\bullet(-,-)$ for morphisms is equivalent to
    $\DPerf(X)$, as in Theorem~\ref{thm:DfromK}. In the smooth case, where
    $\DPerf(X)=D^b(X)$, this gives a ``dg-enhancement'' of $D^b(X)$. The richer
    structure of a dg-category can be nicer to work with than the triangulated
    category $D^b(X)$, e.g. giving functorial cones.
\end{remark}

\todo{say what dg categories are}

\begin{proposition}{}{}
    Any object in $D_\Sg(X)$ is isomorphic to the image of some shifted sheaf.
\end{proposition}

\begin{proof}
    We may take a bounded above projective resolution of an object in $D_\Sg(X)$
    \begin{equation*}
        \begin{tikzcd}
            \cdots \ar[r] & \calP^{a-1} \ar[r] \ar[d] &
            \calP^a \ar[r] \ar[d] & \cdots \ar[r] & \calP^b \ar[d] \\
            \cdots \ar[r] & 0 \ar[r] & \calA^a \ar[r] & \cdots \ar[r] & \calA^b,
        \end{tikzcd}
    \end{equation*}
    and then the truncation
    \begin{equation*}
        \begin{tikzcd}
            \cdots \ar[r] & 0 \ar[r] & \calP^{a-1}/\calP^{a-2} \ar[d] \ar[r] &
            \calP^a \ar[r] \ar[d] & \cdots \ar[r] & \calP^b \ar[d] \\
                & \cdots \ar[r] & 0 \ar[r] &
                \calA^a \ar[r] & \cdots \ar[r] & \calA^b
        \end{tikzcd}
    \end{equation*}
    is also a quasi-isomorphism. This truncated resolution is an extension of
    $\calP^{a-1}/\calP^{a-2}[1-a]$ by the perfect complex
    $\calP^a\to\cdots\to\calP^b$, and so in $D_\Sg(X)$ is isomorphic to
    $\calP^{a-1}/\calP^{a-2}[1-a]$.
\end{proof}

\begin{proposition}[label=prop:DSgHom]{}{}
    If $\calF,\calG\in\Coh(X)$ satisfy
    $\lExt^i(\calF,\O_X)=0=\lExt^i(\calG,\O_X)$ for $i>0$, then
    \begin{equation*}
        \Hom_{D_\Sg(X)}(\calF,\calG[i])
            \simeq \Hom_{D^b(X)}(\calF,\calG[i])/\calR
    \end{equation*}
    where $\calR$ is the subspace of maps factoring through a locally free
    sheaf.
\end{proposition}

% TODO: proof?

\begin{example}{}{}
    Consider a non-reduced point $X=\Spec\C[t]/t^n$. By the structure theorem
    for finite type modules over a PID, coherent sheaves on $X$ are direct sums
    of the modules $\C[t]/t^i$ for $i=0,\ldots,n$. Moreover, by taking Smith
    normal forms any complex is a direct sum of shifts of these modules. Only
    $i=n$ gives a projective module, and so $D_\Sg(X)$ is generated by
    $\C[t]/t,\C[t]/t^2,\ldots,\C[t]/t^{n-1}$. Note that these all have infinite
    periodic free resolutions as $\C[t]/t^n$-modules, corresponding to the
    factorizations $t^n=t^it^{n-i}$. % TODO: morphisms
\end{example}

One might hope that $D_\Sg(X)$ only depends on the local geometry of $X$ near
its singular locus, and in fact this is true; $D_\Sg(U)\simeq D_\Sg(X)$ for a
formal neighbourhood $U$ of the singular locus. See
\cite[Prop~1.14]{OrlovSingularities} and \cite[\S6]{Shipman}.

\subsection{Matrix factorizations}

Consider the cone $Z=\{xy=z^2\}\subset\A^3$. Looking at the structure sheaf of
the line $L=\{x=z=0\}\subset Z$, we are led to a periodic projective resolution
\begin{equation}\label{eqn:coneMF}
    \cdots \to
    \O_Z^2 \xrightarrow{\begin{pmatrix}
        y & -z \\ -z & x
    \end{pmatrix}}
    \O_Z^2 \xrightarrow{\begin{pmatrix}
        x & z \\ z & y
    \end{pmatrix}}
    \O_Z^2 \xrightarrow{\begin{pmatrix}
        y & -z \\ -z & x
    \end{pmatrix}}
    \O_Z^2 \xrightarrow{\begin{pmatrix}
        x & z
    \end{pmatrix}}
    \O_L\to0,
\end{equation}
arising from a factorization of $xy-z^2$ into matrices:
\begin{equation*}
    \begin{pmatrix}
        y & -z \\ -z & x
    \end{pmatrix}
    \begin{pmatrix}
        x & z \\ z & y
    \end{pmatrix}
        = (xy-z^2)I =
    \begin{pmatrix}
        x & z \\ z & y
    \end{pmatrix}
    \begin{pmatrix}
        y & -z \\ -z & x
    \end{pmatrix}.
\end{equation*}
Viewed over $\A^3$, this is a kind of twisted chain complex, satisfying
$d^2=xy-z^2$ instead of $d^2=0$.

\begin{definition}{}{}
    Suppose $X$ is a smooth quasi-projective variety, and $W:X\to\A^1$ is a
    regular function. A matrix factorization of $W$ on $X$ is a pair of vector
    bundles $\calE^0,\calE^1$ on $X$ with maps
    \begin{equation*}
        \begin{tikzcd}
            \calE^0 \ar[r,"d^0",shift left] & \calE^1 \ar[l,"d^1",shift left]
        \end{tikzcd}
    \end{equation*}
    satisfying $d^1d^0=W\cdot\id_{\calE^0}$, $d^0d^1=W\cdot\id_{\calE^1}$.
\end{definition}

In other words, these are $\Z_2$-graded twisted chain complexes with $d^2=W$.
Given two matrix factorizations $\calE^\bullet,\calF^\bullet$ we obtain a
genuine $\Z_2$-graded complex $\Hom^\bullet(\calE^\bullet,\calF^\bullet)$ with
differential $f\mapsto d\circ f-(-1)^{|f|}f\circ d$ squaring to $W-W=0$. This
gives a differential $\Z_2$-graded category, and as seen earlier we then have a
triangulated homotopy category.

\begin{definition}{}{}
    Taking the 0th cohomology of the complex
    $\Hom^\bullet(\calE^\bullet,\calF^\bullet)$ for morphisms gives a
    triangulated category $\MF(X,W)$ of matrix factorizations.
\end{definition}

\begin{remark}{}{}
    If $W=0$ we recover a weaker version of $\DPerf(X)=D^b(X)$ with only
    $\Z_2$-grading, having a natural forgetful functor $D^b(X)\to\MF(X,0)$.
\end{remark}

\begin{example}{}{knorrerbase}
     Suppose $X=\A^2$, with $W=xy$. The obvious factorization of $W$ gives an
     object
     \begin{equation*}
         K = \left(\begin{tikzcd}
             \O_X \ar[r,"x",shift left] &
             \O_X \ar[l,"y",shift left]
         \end{tikzcd}\right) \in\MF(X,W).
     \end{equation*}
     Note that
     \begin{equation*}
         \Hom^\bullet(K,K) =
         \left(\begin{tikzcd}[ampersand replacement=\&,column sep=huge]
                 \O_X^2 \ar[shift left]{r}{\begin{pmatrix}
                     x & -x \\ -y & y
                 \end{pmatrix}} \&
                 \O_X^2 \ar[shift left]{l}{\begin{pmatrix}
                     y & x \\ y & x
                 \end{pmatrix}}
         \end{tikzcd}\right),
     \end{equation*}
     with cohomology
     \begin{align*}
         H^0\Hom(K,K) &= \{(f,g):f=g\}/\langle(x,x),(y,y)\rangle
             = \C\cdot(1,1), \\
         H^1\Hom(K,K) &= \{(f,g):yf+xg=0\}/\langle(x,-y)\rangle
             = 0,
     \end{align*}
     so $K$ is an exceptional object. In fact $\langle K\rangle^\perp=0$, so
     $\MF(X,W)\simeq D^b(\pt)$ is generated by $K$. % TODO: why?
\end{example}

\begin{remark}{}{}
    Matrix factorizations first arose in commutative algebra, relating to
    maximal Cohen--Macaulay modules \cite{Eisenbud}. In the geometric setting of
    derived categories they are a natural object of study, e.g. via minimal
    resolutions, and motivated by Orlov's theorem. In the context of mirror
    symmetry, (graded) matrix factorizations arise as the conjectural (due to
    Kontsevich) category of B-branes on a Landau--Ginzburg model $(X,W)$,
    generalizing the case $W=0$ which should give $D^b(X)$. See
    \cite{OrlovTheorem}, \cite{Ed}.
\end{remark}

\todo{move this to graded MFs + Orlov's theorem section}

\begin{example}{}{cubicMF}
    Consider a cubic curve $C=\{f=0\}\subset\P^2$, and a point $p\in C$. If
    $p=[0:0:1]$ we can write $f=xP+yQ=\det\begin{pmatrix}
        x & y \\ -Q & P
    \end{pmatrix}$, from which we get a matrix factorization
    \begin{equation*}
        \begin{tikzcd}[ampersand replacement=\&,column sep=huge]
            \O_{\A^3}^2 \ar[shift left]{r}{\begin{pmatrix}
                    x & y \\ -Q & P
            \end{pmatrix}} \&
            \O_{\A^3}^2, \ar[shift left]{l}{\begin{pmatrix}
                    P & -y \\ Q & x
            \end{pmatrix}}
        \end{tikzcd}
    \end{equation*}
    on the cone $\A^3$ over $\P^2$. (We could view this as a matrix
    factorization on $\P^2$, but that requires twisting by $\O(1)$, leading to
    the notion of a \emph{graded} matrix factorization.) This cone $Z$ over $C$
    has a line $L$ corresponding to $p$, and this gives a resolution
    \begin{equation*}
        \cdots \to
        \O_Z^2 \xrightarrow{\begin{pmatrix}
                x & y \\ -Q & P
        \end{pmatrix}}
        \O_Z^2 \xrightarrow{\begin{pmatrix}
                P & -y \\ Q & x
        \end{pmatrix}}
        \O_Z^2 \xrightarrow{\begin{pmatrix}
            x & y
        \end{pmatrix}} \O_Z \to \O_L \to 0.
    \end{equation*}
    % TODO: independent of P,Q choice up to homotopy
    As we vary $p$, and correspondingly $L$, the ideal sheaf deforms
    non-trivially, so we expect to get matrix factorizations of distinct
    homotopy types. Note that this factorization of a matrix determinant
    generalizes via the adjugate construction: $\adj(A)A=A\adj(A)=\det(A)I$.
\end{example}

\subsection{Comparing categories of singularities and matrix factorizations}

Now we describe how to obtain a periodic resolution from a matrix factorization,
as in \eqref{eqn:coneMF}. We will resolve the cokernel
$\calE^0\to\calE^1\to K\to0$. Writing $Y=\{W=0\}\subset X$, we have that $K$ is
supported on $Y$, since it is annhilated by $W$:
\begin{equation*}
    \begin{tikzcd}
        K_0 \ar[r,"W"] & K_0 \\
        \calE^1 \ar[r,"W"] \ar[dr,"d^1"'] \ar[u,two heads] &
        \calE^1 \ar[u,two heads] \\
        & \calE^0 \ar[u,"d^0"'] \ar[uu,bend right=60,"0"']
    \end{tikzcd}
\end{equation*}
Suppose that $Y=\{W=0\}\subset X$ is a Cartier divisor, i.e. $W$ is not a
zero-divisor. Restricting the exact sequence to $Y$, we then have
\begin{equation*}
    0 \to d^1\calE^1/W\calE^1
        \to \calE^0/W\calE^0 \xrightarrow{d^0|_Y} \calE^1/W\calE^1 \to K \to 0.
\end{equation*}
Here the kernel of $d^0|_Y$ is the image of $d^1$ since $W=d^1d^0$ and we can
cancel $W$. Repeating the same argument for $d^1|_Y$, we then get a locally free
resolution of $K$ in $\Coh(Y)$:
\begin{equation*}
    \cdots \to \calE^0/W\calE^0 \to \calE^1/W\calE^1
        \to \calE^0/W\calE^0 \to \calE^1/W\calE^1 \to K \to 0.
\end{equation*}

\begin{proposition}{}{}
    Associating $K\in\Coh(Y)$ to the matrix factorization $\calE^\bullet$
    defines a functor $\MF(X,W)\to D_\Sg(Y)$ which is an exact equivalence.
\end{proposition}

\begin{remark}{}{}
    We could define a similar functor by taking the cokernel of $d^1$ instead of
    $d^0$. These give distinct objects of $D^b(Y)$ (consider
    Example~\ref{ex:knorrerbase}), but both are given by the same periodic
    locally free resolution up to a shift and an appended perfect complex, so
    the results in $D_\Sg(Y)$ only differ by a shift.
\end{remark}

\begin{proof}
    A map of matrix factorizations $\calE^\bullet\to\calF^\bullet$ is
    null-homotopic iff it factors through
    \begin{equation*}
        \begin{tikzcd}[ampersand replacement=\&,column sep=huge]
            \calF^0\oplus\calF^1 \ar[shift left]{r}{\begin{pmatrix}
                    d^0 & 0 \\ 1 & -d^1
            \end{pmatrix}} \&
            \calF^1\oplus\calF^0, \ar[shift left]{l}{\begin{pmatrix}
                    d^1 & 0 \\ 1 & -d^0
            \end{pmatrix}}
        \end{tikzcd}
    \end{equation*}
    which has cokernel
    \begin{equation*}
        \frac{\calF^1\oplus\calF^0/d^1\calF^1}
            {\langle d^0x\oplus x : x\in\calF^0\rangle}
            \simeq \calF^1/W\calF^1
    \end{equation*}
    since the injection $d^0:\calF^0\to\calF^1$ induces an injection
    $d^0:\calF^0/d^1\calF^1\to\calF^1/d^0d^1\calF^1=\calF^1/W\calF^1$. This is
    the restriction of the locally free sheaf $\calF^1$ to $Y$, and hence gives
    zero in $D_\Sg(Y)$, so $F_0$ is a well-defined functor, which naturally
    respects the triangulated structures.

    \todo{equivalence proof}

    \begin{itemize}
        \item Full: \ref{prop:DSgHom} and lifting to projective resolutions
        \item Faithful: lemma that $F_0(X)=0\implies X=0$
        \item Essentially surjective: surject $\calE^0\to\calF$, take
            $\calE^1=\ker(\calE^0\to\calF)$, have section $\calE^0\to\calE^1$
            with $d^0d^1=d^1d^0$, need to check $\calE^1$ is locally free. Check
            on stalks by showing $\Ext^i(\calE^1,\O_\pt)=0$ for $i>0$, Gorenstein
            stuff
    \end{itemize}
\end{proof}

\subsection{Kn\"orrer periodicity}

Generalizing the example of $\MF(\A^2,xy)$, given a matrix factorization of a
function $W=MN=NM$, we can produce a factorization of $W+xy$ as follows:
\begin{equation*}
    \begin{pmatrix}
        M & x \\ -y & N
    \end{pmatrix}\begin{pmatrix}
        N & -x \\ y & M
    \end{pmatrix}
        = W+xy =
    \begin{pmatrix}
        N & -x \\ y & M
    \end{pmatrix}\begin{pmatrix}
        M & x \\ -y & N
    \end{pmatrix}
\end{equation*}
In fact this is somehow the only way matrix factorizations of $W+xy$ can arise.

\begin{theorem}{Kn\"orrer periodicity}{}
    $\MF(X\times\A^2,W+xy)\simeq\MF(X,W)$.
\end{theorem}

In terms of derived categories of singularities, we can formulate a more general
statement. Suppose $\calE$ is a vector bundle on $X$, with a section $s$ cutting
out $Y\subset X$ as a regular embedding. The section $s$ induces by linearity a
section of $\O_{\P\calE^\vee}(1)$, cutting out $\tilde Y\subset\P\calE^\vee$. By
adjunction $\calE|_Y=\calN_{Y/X}$, and we get an inclusion
$i:\P\calN^\vee_{Y/X}\hookrightarrow\tilde Y$. Write $q:\tilde Y\to X$ for the
projection.

\begin{theorem}{Orlov \cite{OrlovKnorrer}}{}
    The Fourier--Mukai transform $\Phi_Z=\dR i_*q^*:D^b(Y)\to D^b(\tilde Y)$
    induces an equivalence $D_\Sg(Y)\simeq D_\Sg(\tilde Y)$.
\end{theorem}

\begin{example}{}{}
    Suppose we have two functions $f,g:X\to\A^1$, with $D=\{g=0\}\subset X$
    smooth. As a section of the trivial bundle $\O_X^2$, this gives
    $Y=\{f=g=0\}\subset D$ and $\tilde Y=\{f+tg=0\}\subset X\times\P^1$
    where $t$ is a coordinate on $\P^1$. Then $D_\Sg(Y)\simeq D_\Sg(\tilde Y)$
    implies $\MF(X\times\P^1,f+tg)\simeq\MF(D,f)$. Restricting $f+tg$ to
    $X\times\{\infty\}$ gives $g$, so since $D$ is smooth and $D_\Sg$ only
    depends on a neighbourhood of the singular locus we actually have
    $\MF(X\times\A^1,f+tg)\simeq\MF(D,f)$. From this we can recover the original
    Kn\"orrer periodicity by taking $X=X'\times\A^1_s$ and $g=s$.
    \todo{non-trivial line bundles}
\end{example}

\begin{proof}
    Firstly, we claim that there is a semi-orthogonal decomposition
    \begin{equation*}
        D^b(\tilde Y)
            = \langle\Phi_ZD^b(Y),
                \dL\pi^*D^b(X)\otimes\O_{\tilde Y/X}(1),\ldots,
                \dL\pi^*D^b(X)\otimes\O_{\tilde Y/X}(r-1)\rangle
    \end{equation*}
    where $\pi:\tilde Y\to X$ is the projection, and $\O_{\tilde Y/X}(1)$ is the
    restriction of $\O_{\P\calE^\vee}(1)$.

    \todo{finish proof}
\end{proof}

\todo{cone example, other examples}

\subsection{Graded matrix factorizations and Orlov's theorem}

% kuznetzov component example for orlov's theorem
