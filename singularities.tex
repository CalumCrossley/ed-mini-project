
\subsection{Derived categories of singularities}

When considering smooth projective varieties, we made use of the existence of
bounded locally free resolutions (Hilbert's syszygy theorem), e.g. in defining
derived pullbacks and tensor products. If we want to look at singular varieties,
the failure of this property is an important point.

\begin{theorem}{Serre}{}
    A commutative local ring $A$ is regular iff it has finite global dimension,
    i.e. there is a global bound on the length of projective resolutions of
    $A$-modules.
\end{theorem}

Motivated by this, we are led to the following definition due to Orlov.

\begin{definition}{}{}
    Suppose $X$ is a quasi-projective variety. A complex of sheaves on $X$ is
    perfect if it is quasi-isomorphic to a bounded complex of finite type
    locally free sheaves. This gives a triangulated subcategory
    $\DPerf(X)\subseteq D^b(X)$, by the construction of cones in terms of direct
    sums. The derived category of singularities is the quotient
    $D_\Sg(X)\coloneqq D^b(X)/\DPerf(X)$.
\end{definition}

Recall from Definition~\ref{defn:verdierquotient} that this quotient is given by
inverting all morphisms in $D^b(X)$ whose cones are perfect complexes.

\begin{remark}{}{}
    As noted above, for a smooth variety all bounded complexes of coherent
    sheaves are perfect, so $\DPerf(X)=D^b(X)$ and $D_\Sg(X)=0$.
\end{remark}

\begin{remark}{}{}
    The \emph{underived} category of perfect complexes
    $\Perf(X)\subseteq K^b(\Coh X)$ inherits a differential graded structure
    from $K^b(\Coh X)$. The homotopy category $H^0\Perf(X)$ given by taking 0th
    cohomology of $\Hom^\bullet(-,-)$ for morphisms is equivalent to
    $\DPerf(X)$, as in Theorem~\ref{thm:DfromK}. In the smooth case, where
    $\DPerf(X)=D^b(X)$, this gives a ``dg-enhancement'' of $D^b(X)$. The richer
    structure of a dg-category can be nicer to work with than the triangulated
    category $D^b(X)$, e.g. giving functorial cones.
\end{remark}

\begin{proposition}[label=prop:SgObjects]{}{}
    Any object in $D_\Sg(X)$ is isomorphic to the image of some shifted sheaf.
\end{proposition}

\begin{proof}
    We may take a bounded above projective resolution of an object in $D_\Sg(X)$
    \begin{equation*}
        \begin{tikzcd}
            \cdots \ar[r] & \calP^{a-1} \ar[r] \ar[d] &
            \calP^a \ar[r] \ar[d] & \cdots \ar[r] & \calP^b \ar[d] \\
            \cdots \ar[r] & 0 \ar[r] & \calA^a \ar[r] & \cdots \ar[r] & \calA^b,
        \end{tikzcd}
    \end{equation*}
    and then the truncation
    \begin{equation*}
        \begin{tikzcd}
            \cdots \ar[r] & 0 \ar[r] & \calP^{a-1}/\calP^{a-2} \ar[d] \ar[r] &
            \calP^a \ar[r] \ar[d] & \cdots \ar[r] & \calP^b \ar[d] \\
                & \cdots \ar[r] & 0 \ar[r] &
                \calA^a \ar[r] & \cdots \ar[r] & \calA^b
        \end{tikzcd}
    \end{equation*}
    is also a quasi-isomorphism. This truncated resolution is an extension of
    $\calP^{a-1}/\calP^{a-2}[1-a]$ by the perfect complex
    $\calP^a\to\cdots\to\calP^b$, and so in $D_\Sg(X)$ is isomorphic to
    $\calP^{a-1}/\calP^{a-2}[1-a]$.
\end{proof}

The theory of morphisms in $D_\Sg(X)$ is particularly nice when $X$ is
\emph{Gorenstein}, where we have a bounded derived dual $\RHom(-,\O_X)$ on
$D^b(X)$ satisfying $\RHom(\RHom(\calF,\O_X),\O_X)\simeq\calF$. This holds for
local complete intersections in smooth varieties.

\begin{remark}[label=rmk:SgObjects]{}{}
    When $X$ is Gorenstein, the shifted sheaf $\calF$ in
    Proposition~\ref{prop:SgObjects} may be taken to satisfy
    $\lExt^i(\calF,\O_X)=0$ for $i>0$, since $\RlHom(\calA^\bullet,\O_X)$ is
    bounded and may be assumed within the truncated range.
\end{remark}

\begin{proposition}[label=prop:SgLF]{\cite[Lemma 1.20]{OrlovSingularities}}{}
    Suppose $X$ is Gorenstein. If $\calF$ is a coherent sheaf which is perfect
    as a complex, and $\lExt^i(\calF,\O_X)=0$ for $i>0$, then $\calF$ is locally
    free.
\end{proposition}

\begin{proposition}[label=prop:DSgHom]{\cite[Prop 1.21]{OrlovSingularities}}{}
    If $\calF,\calG\in\Coh(X)$ satisfy
    $\lExt^i(\calF,\O_X)=0$ for $i>0$ and $\lExt^i(\calP,\calG)=0$ for $i>N$ and
    any locally free sheaf $\calP$, then
    \begin{equation*}
        \Hom_{D_\Sg(X)}(\calF,\calG[N])
            \simeq \Hom_{D^b(X)}(\calF,\calG[N])/\calR
    \end{equation*}
    where $\calR$ is the subspace of maps factoring through a locally free
    sheaf.
\end{proposition}

\begin{example}[label=ex:nonreducedpoint]{}{}
    Consider a non-reduced point $X=\Spec\C[t]/t^n$. By the structure theorem
    for finite type modules over a PID, coherent sheaves on $X$ are direct sums
    of the modules $\C[t]/t^i$ for $i=0,\ldots,n$. Moreover, by taking Smith
    normal forms any complex is a direct sum of shifts of these modules. Only
    $i=n$ gives a projective module, and so $D_\Sg(X)$ is generated by
    $\C[t]/t,\C[t]/t^2,\ldots,\C[t]/t^{n-1}$. Note that these all have infinite
    periodic free resolutions as $\C[t]/t^n$-modules, corresponding to the
    factorizations $t^n=t^it^{n-i}$. The corresponding extension
    \begin{equation*}
        0 \to \C[t]/t^{n-i} \xrightarrow{t^i} \C[t]/t^n \to \C[t]/t^i \to 0
    \end{equation*}
    shows that $\C[t]/t^i$ is the shift of $\C[t]/t^{n-i}$ by 1. In particular
    the shift functor is an involution, which we will see holds more generally
    in \ref{subsec:comparison}. By Proposition~\ref{prop:DSgHom} we can see for
    example that the map $\C[t]/t^i\xrightarrow{t^k}\C[t]/t^j$ gives zero in
    $D_\Sg(X)$ if $k\ge n-i$.
\end{example}

One might hope that $D_\Sg(X)$ only depends on the local geometry of $X$ near
its singular locus, and in fact this is true; $D_\Sg(U)\simeq D_\Sg(X)$ for a
formal neighbourhood $U$ of the singular locus. See
\cite[Prop~1.14]{OrlovSingularities} and \cite[\S6]{Shipman}.

\subsection{Matrix factorizations}

Consider the cone $Z=\{xy=z^2\}\subset\A^3$. Looking at the structure sheaf of
the line $L=\{x=z=0\}\subset Z$, we are led to a periodic projective resolution
\begin{equation}\label{eqn:coneMF}
    \cdots \to
    \O_Z^2 \xrightarrow{\begin{pmatrix}
        y & -z \\ -z & x
    \end{pmatrix}}
    \O_Z^2 \xrightarrow{\begin{pmatrix}
        x & z \\ z & y
    \end{pmatrix}}
    \O_Z^2 \xrightarrow{\begin{pmatrix}
        y & -z \\ -z & x
    \end{pmatrix}}
    \O_Z^2 \xrightarrow{\begin{pmatrix}
        x & z
    \end{pmatrix}}
    \O_L\to0,
\end{equation}
arising from a factorization of $xy-z^2$ into matrices:
\begin{equation*}
    \begin{pmatrix}
        y & -z \\ -z & x
    \end{pmatrix}
    \begin{pmatrix}
        x & z \\ z & y
    \end{pmatrix}
        = (xy-z^2)I =
    \begin{pmatrix}
        x & z \\ z & y
    \end{pmatrix}
    \begin{pmatrix}
        y & -z \\ -z & x
    \end{pmatrix}.
\end{equation*}
Viewed over $\A^3$, this is a kind of twisted chain complex, satisfying
$d^2=xy-z^2$ instead of $d^2=0$.

\begin{definition}{}{}
    Suppose $X$ is a smooth quasi-projective variety, and $W:X\to\A^1$ is a
    regular function. A matrix factorization of $W$ on $X$ is a pair of vector
    bundles $\calE^0,\calE^1$ on $X$ with maps
    \begin{equation*}
        \begin{tikzcd}
            \calE^0 \ar[r,"d^0",shift left] & \calE^1 \ar[l,"d^1",shift left]
        \end{tikzcd}
    \end{equation*}
    satisfying $d^1d^0=W\cdot\id_{\calE^0}$, $d^0d^1=W\cdot\id_{\calE^1}$.
\end{definition}

\begin{remark}{}{}
    The differential $d^1$ is a null-homotopy of the multiplication by $W$ map
    on the chain complex $\calE^0\to\calE^1$, and similar for $d^0$. Hence the
    sheaves $\ker d^i$, $\coker d^i$ are supported on $\{W=0\}\subset X$. When
    $W$ is not a zero-divisor we get $\ker d^i=0$, but the cokernels are more
    interesting; see \ref{subsec:comparison}.
\end{remark}

In other words, these are $\Z_2$-graded twisted chain complexes with $d^2=W$.
Given two matrix factorizations $\calE^\bullet,\calF^\bullet$ we obtain a
genuine $\Z_2$-graded complex $\Hom^\bullet(\calE^\bullet,\calF^\bullet)$ with
differential $f\mapsto d\circ f-(-1)^{|f|}f\circ d$ squaring to $W-W=0$. This
gives a differential $\Z_2$-graded category, and as seen earlier we then have a
triangulated homotopy category.

\begin{definition}{}{}
    Taking the 0th cohomology of the complex
    $\Hom^\bullet(\calE^\bullet,\calF^\bullet)$ for morphisms gives a
    triangulated category $\MF(X,W)$ of matrix factorizations.
\end{definition}

\begin{remark}{}{}
    If $W=0$ we recover a weaker version of $\DPerf(X)=D^b(X)$ with only
    $\Z_2$-grading, having a natural forgetful functor $D^b(X)\to\MF(X,0)$.
\end{remark}

The collapse from $\Z$-graded chain complexes when $d^2=0$ to only having the
$\Z_2$-grading when $d^2=W$ is somewhat unfortunate, and one might prefer to
work instead with $\Z$-graded complexes where $W$ somehow has degree 2. This
leads to the notion of \emph{graded matrix factorizations}; see
\ref{subsec:GMF}.

\begin{example}{}{knorrerbase}
     Suppose $X=\A^2$, with $W=xy$. The obvious factorization of $W$ gives an
     object
     \begin{equation*}
         K = \left(\begin{tikzcd}
             \O_X \ar[r,"x",shift left] &
             \O_X \ar[l,"y",shift left]
         \end{tikzcd}\right) \in\MF(X,W).
     \end{equation*}
     Note that
     \begin{equation*}
         \Hom^\bullet(K,K) =
         \left(\begin{tikzcd}[ampersand replacement=\&,column sep=huge]
                 \O_X^2 \ar[shift left]{r}{\begin{pmatrix}
                     x & -x \\ -y & y
                 \end{pmatrix}} \&
                 \O_X^2 \ar[shift left]{l}{\begin{pmatrix}
                     y & x \\ y & x
                 \end{pmatrix}}
         \end{tikzcd}\right),
     \end{equation*}
     with cohomology
     \begin{align*}
         H^0\Hom(K,K) &= \{(f,g):f=g\}/\langle(x,x),(y,y)\rangle
             = \C\cdot(1,1), \\
         H^1\Hom(K,K) &= \{(f,g):yf+xg=0\}/\langle(x,-y)\rangle
             = 0,
     \end{align*}
     so $K$ is an exceptional object. In fact $\langle K\rangle^\perp=0$, so
     $\MF(X,W)\simeq D^b(\pt)$ is generated by $K$. This is the most basic
     example of \emph{Kn\"orrer periodicity}, which we will see in
     \ref{subsec:knorrer}.
\end{example}

\begin{remark}{}{}
    Matrix factorizations first arose in commutative algebra, relating to
    maximal Cohen--Macaulay modules \cite{Eisenbud}. In the geometric setting of
    derived categories they are a natural object of study, e.g. via minimal
    resolutions, and motivated by Orlov's theorem. In the context of mirror
    symmetry, (graded) matrix factorizations arise as the conjectural (due to
    Kontsevich) category of B-branes on a Landau--Ginzburg model $(X,W)$,
    generalizing the case $W=0$ which should give $D^b(X)$. See
    \cite{OrlovTheorem}, \cite{ed_gitlg_2011}.
\end{remark}

\subsection{Comparing categories of singularities and matrix factorizations}
\label{subsec:comparison}

Suppose $Y=\{W=0\}\subset X$ is a Cartier divisor, i.e. $W$ is not a
zero-divisor. We now describe how to obtain a periodic resolution from a matrix
factorization, as in \eqref{eqn:coneMF}, by resolving the cokernel
$\calE^0\to\calE^1\to K\to0$. Note that $K$ is supported on $Y$, since it is
annhilated by $W$:
\begin{equation*}
    \begin{tikzcd}
        K \ar[r,"W"] & K \\
        \calE^1 \ar[r,"W"] \ar[dr,"d^1"'] \ar[u,two heads] &
        \calE^1 \ar[u,two heads] \\
        & \calE^0 \ar[u,"d^0"'] \ar[uu,bend right=60,"0"']
    \end{tikzcd}
\end{equation*}
Restricting the exact sequence to $Y$, we then have
\begin{equation*}
    0 \to d^1\calE^1/W\calE^1
        \to \calE^0/W\calE^0 \xrightarrow{d^0|_Y} \calE^1/W\calE^1 \to K \to 0.
\end{equation*}
Here the kernel of $d^0|_Y$ is the image of $d^1$ since $W=d^1d^0$ and we can
cancel $W$. Repeating the same argument for $d^1|_Y$, we then get a locally free
resolution of $K$ in $\Coh(Y)$:
\begin{equation*}
    \cdots \to \calE^0/W\calE^0 \to \calE^1/W\calE^1
        \to \calE^0/W\calE^0 \to \calE^1/W\calE^1 \to K \to 0.
\end{equation*}

\begin{remark}[label=rmk:MFchainmaps]{}{}
    Recall that maps of the cokernel $K$ correspond to chain maps of
    $\calE^0\to\calE^1$ modulo homotopy. Such chain maps automatically also
    respect the differential $\calE^1\to\calE^0$, by applying $d^1$ to the
    chain map condition and cancelling $W$. Homotopies of these chain maps are a
    special case of homotopies of the maps of matrix factorizations.
\end{remark}

\begin{proposition}{}{}
    Associating $K\in\Coh(Y)$ to the matrix factorization $\calE^\bullet$
    defines a functor $R:\MF(X,W)\to D_\Sg(Y)$ which is an exact equivalence.
\end{proposition}

\begin{remark}{}{}
    We could define a similar functor by taking the cokernel of $d^1$ instead of
    $d^0$. These give distinct objects of $D^b(Y)$ (consider
    Example~\ref{ex:knorrerbase}), but both are given by the same periodic
    locally free resolution up to a shift and an appended perfect complex, so
    the results in $D_\Sg(Y)$ only differ by a shift.
\end{remark}

\begin{proof}
    A map of matrix factorizations $\calE^\bullet\to\calF^\bullet$ is
    null-homotopic iff it factors through
    \begin{equation*}
        \begin{tikzcd}[ampersand replacement=\&,column sep=huge]
            \calF^0\oplus\calF^1 \ar[shift left]{r}{\begin{pmatrix}
                    d^0 & 0 \\ 1 & -d^1
            \end{pmatrix}} \&
            \calF^1\oplus\calF^0, \ar[shift left]{l}{\begin{pmatrix}
                    d^1 & 0 \\ 1 & -d^0
            \end{pmatrix}}
        \end{tikzcd}
    \end{equation*}
    which has cokernel
    \begin{equation*}
        \frac{\calF^1\oplus\calF^0/d^1\calF^1}
            {\langle d^0x\oplus x : x\in\calF^0\rangle}
            \simeq \calF^1/W\calF^1
    \end{equation*}
    since the injection $d^0:\calF^0\to\calF^1$ induces an injection
    $d^0:\calF^0/d^1\calF^1\to\calF^1/d^0d^1\calF^1=\calF^1/W\calF^1$. This is
    the restriction of the locally free sheaf $\calF^1$ to $Y$, and hence gives
    zero in $D_\Sg(Y)$, so $R$ is a well-defined functor which naturally
    respects the triangulated structures. It is full by
    Proposition~\ref{prop:DSgHom} and Remark~\ref{rmk:MFchainmaps}.

    \textbf{Claim:} If $R(\calE^\bullet)=0$ in $D_\Sg(Y)$, then
    $\calE^\bullet=0$ in $\MF(X,W)$.

    \textit{Proof of claim.} If $K$ is a perfect complex on $Y$, then in fact
    $K$ is locally free over $Y$ by Proposition~\ref{prop:SgLF}. Then we have a
    section $Y\to\calE^1|_Y$ of the short exact sequence
    $0\to\calE^0|_Y\to\calE^1|_Y\to K\to0$, which gives a factorization of the
    identity on $\calE^\bullet$ by Remark~\ref{rmk:MFchainmaps}:
    \begin{equation*}
        \begin{tikzcd}
            \calE^0 \ar[shift left]{r}{d^0} \ar[d,dashed] &
            \calE^1 \ar[shift left]{l}{d^1} \ar[d,dashed] \ar[r] &
            K \ar[d] \\
            \calE^1 \ar[shift left]{r}{W} \ar[d,"d^1"] &
            \calE^1 \ar[shift left]{l}{1} \ar[d,"1"] \ar[r] &
            \calE^1|_Y \ar[d] \\
            \calE^0 \ar[shift left]{r}{d^0} &
            \calE^1 \ar[shift left]{l}{d^1} \ar[r] &
            K.
        \end{tikzcd}
    \end{equation*}
    Since $\calE^1\xleftarrow{1}\calE^1$ is acyclic, the middle object is zero
    in $\MF(X,W)$. \qed

    With this claim it is a formal consequence that $R$ is faithful; if an exact
    triangle
    $\calE^\bullet\xrightarrow{f}\calF^\bullet\xrightarrow{g}\calG^\bullet$ has
    $R(f)=0$, then $R(g)$ is an isomorphism and has a retraction of the form
    $R(r)$ since $R$ is full. Now $R(rg)=1$ implies $rf$ is an isomorphism,
    since the cone must be zero, so $g$ is monic and $f=0$.

    To show essential surjectivity, by Remark~\ref{rmk:SgObjects} it suffices to
    consider a sheaf $\calF$ supported on $Y$ with $\lExt^i_Y(\calF,\O_Y)=0$ for
    $i>0$. Take a surjection $\calE^1\to\calF$ with $\calE^1$ a vector bundle on
    $X$.

    \textbf{Claim:} The sheaf $\calE^0=\ker(\calE^1\to\calF)$ is a vector
    bundle.

    \textit{Proof of claim.} This can be checked on stalks by showing
    $\Ext^i_X(\calE^0,\O_\pt)=0$ for $i>0$ and all points in $X$, and from the
    short exact sequence $0\to\calE^0\to\calE^1\to\calF\to0$ it suffices to show
    that $\Ext^i_X(\calF,\O_\pt)=0$ for $i>1$. Since $\Ext^i_Y(\calF,\O_Y)=0$
    for $i>0$, we have a locally free \emph{right} resolution of $\calF$ on $Y$
    by resolving $\calF^\vee$ and dualizing. From the Grothendieck spectral
    sequence, it suffices to show that $\Ext^i_X(Q,\O_\pt)=0$ for $i>1$ when $Q$
    is a locally free sheaf on $Y$. This follows from
    $\RHom_X(Q,\O_\pt)=\RHom_Y(\dL j^*j_*Q,\O_\pt)$, where $j:Y\hookrightarrow X$
    is the inclusion, since $\dL j^*j_*Q$ fits into an exact triangle
    $Q[1]\to\dL j^*j_*Q\to Q$ as in \cite[Cor 11.4]{Huybrechts}. \qed

    Then multiplication by $W$ is null-homotopic on the complex
    $\calE^0\to\calE^1$, since it gives zero on $\calF$, and the homotopy
    $\calE^1\to\calE^0$ makes $\calE^\bullet$ into a matrix factorization with
    cokernel $\calF$ as required.
\end{proof}

\begin{remark}{}{}
    As the shift functor on $\MF(X,W)$ is clearly 2-periodic, it follows that
    the shift functor on $D_\Sg(Y)$ is also 2-periodic, which is not immediately
    obvious.
\end{remark}

\subsection{Kn\"orrer periodicity}\label{subsec:knorrer}

Generalizing the example of $\MF(\A^2,xy)$, given a matrix factorization of a
function $W=MN=NM$, we can produce a factorization of $W+xy$ as follows:
\begin{equation*}
    \begin{pmatrix}
        M & x \\ -y & N
    \end{pmatrix}\begin{pmatrix}
        N & -x \\ y & M
    \end{pmatrix}
        = W+xy =
    \begin{pmatrix}
        N & -x \\ y & M
    \end{pmatrix}\begin{pmatrix}
        M & x \\ -y & N
    \end{pmatrix}.
\end{equation*}
In fact this is somehow the only way matrix factorizations of $W+xy$ can arise.

\begin{theorem}{Kn\"orrer periodicity}{}
    $\MF(X\times\A^2,W+xy)\simeq\MF(X,W)$ by this construction.
\end{theorem}

\begin{example}{}{}
    Consider again the cone $Z=\{z^2+xy=0\}\subset\A^3$. Kn\"orrer periodicity
    gives $D_\Sg(Z)\simeq D_\Sg(\Spec\C[z]/z^2)$, which is generated by
    $\C[z]/z$ as in Example~\ref{ex:nonreducedpoint}. What is the image of this
    generator in $D_\Sg(Z)$? The corresponding matrix factorization is
    $z^2=z\cdot z$, producing
    \begin{equation*}
        \begin{pmatrix}
            z & x \\ -y & z
        \end{pmatrix}\begin{pmatrix}
            z & -x \\ y & z
        \end{pmatrix} = z^2+xy = \begin{pmatrix}
            z & -x \\ y & z
        \end{pmatrix}\begin{pmatrix}
            z & x \\ -y & z
        \end{pmatrix}
    \end{equation*}
    as in \eqref{eqn:coneMF}.
\end{example}

In terms of derived categories of singularities, we can formulate a more general
statement. Suppose $\calE$ is a vector bundle on $X$ of rank $r+1$, with a
section $s$ cutting out $Y\subset X$ as a regular embedding. The section $s$
induces by linearity a section of $\O_{\P\calE^\vee}(1)$, cutting out
$\tilde Y\subset\P\calE^\vee$. By adjunction $\calE|_Y=\calN_{Y/X}$, and we get
an inclusion $i:\P\calN^\vee_{Y/X}\hookrightarrow\tilde Y$. Label the maps as
follows.
\begin{equation*}
    \begin{tikzcd}
        N\coloneqq\P\calN_{Y/X}^\vee \ar[d,"q"] \ar[r,hook,"i"] &
        \tilde Y \ar[r,hook,"j"] \ar[dr,"p"'] &
        \P\calE^\vee \ar[d,"\pi"] \\
        Y \ar[rr,hook] & & X
    \end{tikzcd}
\end{equation*}

\begin{theorem}{\cite[Thm 2.1]{OrlovKnorrer}}{}
    The Fourier--Mukai transform $\Phi_Z=\dR i_*q^*:D^b(Y)\to D^b(\tilde Y)$
    induces an equivalence $D_\Sg(Y)\simeq D_\Sg(\tilde Y)$.
\end{theorem}

\begin{example}{}{}
    Suppose we have two functions $f,g:X\to\A^1$, with $D=\{g=0\}\subset X$
    smooth. As a section of the trivial bundle $\O_X^2$, this gives
    $Y=\{f=g=0\}\subset D$ and $\tilde Y=\{f+tg=0\}\subset X\times\P^1$
    where $t$ is a coordinate on $\P^1$. Then $D_\Sg(Y)\simeq D_\Sg(\tilde Y)$
    implies $\MF(X\times\P^1,f+tg)\simeq\MF(D,f)$. Restricting $f+tg$ to
    $X\times\{\infty\}$ gives $g$, so since $D$ is smooth and $D_\Sg$ only
    depends on a neighbourhood of the singular locus we actually have
    $\MF(X\times\A^1_t,f+tg)\simeq\MF(D,f)$. From this we can recover the
    original Kn\"orrer periodicity by taking $X=X'\times\A^1_s$ and $g=s$.
\end{example}

\begin{proof}
    We reduce to the claim that there is a semi-orthogonal decomposition
    \begin{equation*}
        D^b(\tilde Y)
            = \langle\Phi_ZD^b(Y),
                \dL p^*D^b(X)\otimes\O_p(1),\ldots,
                \dL p^*D^b(X)\otimes\O_p(r)\rangle,
    \end{equation*}
    where the admissible subcategories are equivalent to $D^b(Y)$ and $D^b(X)$,
    i.e. the functors $\Phi_Z$ and $\dL p^*$ are fully faithful.

    It is a formal consequence that this gives a semi-orthogonal decomposition
    of the singularity categories
    \begin{equation*}
        D_\Sg(\tilde Y)
            = \langle\Phi_ZD_\Sg(Y),
                \dL p^*D_\Sg(X)\otimes\O_p(1),\ldots,
                \dL p^*D_\Sg(X)\otimes\O_p(r)\rangle,
    \end{equation*}
    essentially because we can define $D_\Sg(-)$ intrinsically in terms of the
    triangulated category $D^b(-)$, by identifying $\DPerf(-)$ as those objects
    with bounded $\Ext$ groups. Since $X$ is smooth we have $D_\Sg(X)=0$, and so
    it follows that $D_\Sg(\tilde Y)$ is equivalent to $D_\Sg(Y)$ via $\Phi_Z$.

    We now prove the claim.
    \begin{itemize}
        \item $\dL p^*$ is fully faithful and the terms involving $D^b(X)$ are
            orthogonal by the same reasoning as in
            Proposition~\ref{prop:projbundleformula}; we still have
            $\dR p_*\O_{\tilde Y}=\O_X$ from the Koszul resolution, as
            $\dR\pi_*\O_\pi(-1)=0$.

        \item To show that $\Phi_Z$ is fully faithful, first note that $q^*$ is
            fully faithful (part of the projective bundle formula), so
            $\dR q_*q^*\simeq1$. Hence to get
            $\cone(\calF\to\Phi_Z^R\Phi_Z\calF)=0$, where $\Phi_Z^R$ is the
            right adjoint $\dR q_*i^!$, it is enough to show that
            $\calC\coloneqq\cone(q^*\calF\to i^!\dR i_*q^*\calF)$ maps to zero
            under $\dR q_*$. Now
            \begin{align*}
                \dR i_*i^!\dR i_*q^*\calF
                    &\simeq \dR i_*(q^*\calF
                        \otimes\omega_{N/\tilde Y}
                        \otimes i^*\dR i_*\O_N)[-r] \\
                    &\simeq \dR i_*(q^*\calF
                        \otimes\omega_{N/\tilde Y}
                        \otimes\wedge^\bullet\calN_{N/\tilde Y}^\vee)[-r],
            \end{align*}
            and one can conclude that
            $\calH^k(\calC)=q^*\calF\otimes\wedge^k\calN_{N/\tilde Y}$ for
            $k>0$. By the Grothendieck spectral sequence it then suffices to
            show that $\dR q_*(\wedge^k\calN^\vee_{N/\tilde Y})=0$ for $k>0$,
            which can be done by relating $\calN^\vee_{N/\tilde Y}$ to
            $\O_N(1)$. We omit the details; see \cite[Prop 1.17]{OrlovKnorrer}.

        \item To check orthogonality of the $D^b(Y)$ term, for $\calF\in D^b(Y)$
            and $\calG\in D^b(X)$ we have
            \begin{align*}
                \Hom(\dL p^*\calF\otimes\O_p(k),\dR i_*q^*\calG)
                    &= \Hom(q^*\calF\otimes\O_q(k),q^*\calG)
                    \qquad \text{(adjunction)} \\
                    &= \Hom(q^*\calF,q^*\calG\otimes\O_q(-k)) \\
                    &= \Hom(\calF,\dR q_*(q^*\calG\otimes\O_q(-k)))
                    \qquad \text{(adjunction)} \\
                    &= \Hom(\calF,\calG\otimes\dR q_*\O_q(-k))
                    = 0 \qquad \text{(projection formula)}
            \end{align*}
            since $\dR q_*\O_q(-k)=0$.

        \item To check that the sequence generates $D^b(\tilde Y)$ we use the
            spanning class of skyscraper sheaves, so it suffices to show that we
            can generate a Beilinson exceptional collection on each fiber. Over
            $X\setminus Y$ we have a $\P^{r-1}$-bundle given by hyperplanes in
            $\P\calE^\vee$, so the $D^b(X)$ terms give Beilinson exceptional
            collections on the fibers over $X\setminus Y$ by pulling back
            skyscraper sheaves.
            Now consider $y\in Y$. We compute
            \begin{align*}
                \dL p^*\O_y\otimes\O_p(k)
                    &= \dL j^*\pi^*\O_y\otimes\O_p(k) \\
                    &= \dL j^*\O_{\pi^{-1}(y)}\otimes\O_p(k) \\
                    &= \dL j^*j_*\O_{p^{-1}(y)}\otimes\O_p(k),
            \end{align*}
            which has cohomology $\O_{p^{-1}(y)}(k)$ in degree 0 and
            $\O_{p^{-1}(y)}(k-1)$ in degree $-1$ as in
            \cite[Cor 11.4]{Huybrechts}, since $\tilde Y$ is cut out by a
            section of $\O_\pi(1)$. This allows us to inductively generate the
            twists needed for a Beilinson exceptional collection on $p^{-1}(y)$,
            starting with $\O_{p^{-1}(y)}=\Phi_Z\O_y$.
    \end{itemize}
\end{proof}

\begin{remark}{}{}
    This semi-orthogonal decomposition is another variant of the projective
    bundle formula, much like the blowup formula, where we have a hypersurface
    inside a projective bundle. Generalizations can be found in
    \cite{kuznetsov_hpd_2005}.
\end{remark}

\subsection{Graded matrix factorizations and Orlov's theorem}\label{subsec:GMF}

%\begin{example}{}{cubicMF}
%    Consider a cubic curve $C=\{f=0\}\subset\P^2$, and a point $p\in C$. If
%    $p=[0:0:1]$ we can write $f=xP+yQ=\det\begin{pmatrix}
%        x & y \\ -Q & P
%    \end{pmatrix}$, from which we get a matrix factorization
%    \begin{equation*}
%        \begin{tikzcd}[ampersand replacement=\&,column sep=huge]
%            \O_{\A^3}^2 \ar[shift left]{r}{\begin{pmatrix}
%                    x & y \\ -Q & P
%            \end{pmatrix}} \&
%            \O_{\A^3}^2, \ar[shift left]{l}{\begin{pmatrix}
%                    P & -y \\ Q & x
%            \end{pmatrix}}
%        \end{tikzcd}
%    \end{equation*}
%    on the cone $\A^3$ over $\P^2$. (We could view this as a matrix
%    factorization on $\P^2$, but that requires twisting by $\O(1)$, leading to
%    the notion of a \emph{graded} matrix factorization.) This cone $Z$ over $C$
%    has a line $L$ corresponding to $p$, and this gives a resolution
%    \begin{equation*}
%        \cdots \to
%        \O_Z^2 \xrightarrow{\begin{pmatrix}
%                x & y \\ -Q & P
%        \end{pmatrix}}
%        \O_Z^2 \xrightarrow{\begin{pmatrix}
%                P & -y \\ Q & x
%        \end{pmatrix}}
%        \O_Z^2 \xrightarrow{\begin{pmatrix}
%            x & y
%        \end{pmatrix}} \O_Z \to \O_L \to 0.
%    \end{equation*}
%    % TODO: independent of P,Q choice up to homotopy
%    As we vary $p$, and correspondingly $L$, the ideal sheaf deforms
%    non-trivially, so we expect to get matrix factorizations of distinct
%    homotopy types. Note that this factorization of a matrix determinant
%    generalizes via the adjugate construction: $\adj(A)A=A\adj(A)=\det(A)I$.
%\end{example}

% kuznetzov component example for orlov's theorem
