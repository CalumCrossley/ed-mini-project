
\subsection{Derived categories of singularities}

When considering smooth projective varieties, we made use of the existence of
bounded locally free resolutions (Hilbert's syszygy theorem), e.g. in defining
derived pullbacks and tensor products. If we want to look at singular varieties,
the failure of this property is an important point.

\begin{theorem}{Serre}{}
    A commutative local ring $A$ is regular iff it has finite global dimension,
    i.e. there is a global bound on the length of projective resolutions of
    $A$-modules.
\end{theorem}

Motivated by this, one is led to the following definition due to Orlov.

\begin{definition}{}{}
    Suppose $X$ is a quasi-projective variety. A complex of sheaves on $X$ is
    perfect if it is quasi-isomorphic to a bounded complex of finite type
    locally free sheaves. This gives a triangulated subcategory
    $\DPerf(X)\subseteq D^b(X)$, by the construction of cones in terms of direct
    sums. The derived category of singularities is the quotient
    $D_\Sg(X)\coloneqq D^b(X)/\DPerf(X)$.
\end{definition}

Recall from Definition~\ref{defn:verdierquotient} that this quotient is given by
inverting all morphisms in $D^b(X)$ whose cones are perfect complexes.

\begin{remark}{}{}
    As noted above, for a smooth variety all bounded complexes of coherent
    sheaves are perfect, so $\DPerf(X)=D^b(X)$ and $D_\Sg(X)=0$.
\end{remark}

\begin{remark}{}{}
    The \emph{underived} category of perfect complexes
    $\Perf(X)\subseteq K^b(\Coh X)$ inherits a differential graded structure
    from $K^b(\Coh X)$. The homotopy category $H^0\Perf(X)$ given by taking 0th
    cohomology of $\Hom^\bullet(-,-)$ for morphisms is equivalent to
    $\DPerf(X)$, as in Theorem~\ref{thm:DfromK}. In the smooth case, where
    $\DPerf(X)=D^b(X)$, this gives a ``dg-enhancement'' of $D^b(X)$. The richer
    structure of a dg-category can be nicer to work with than the triangulated
    category $D^b(X)$, e.g. giving functorial cones.
\end{remark}

\todo{say what dg categories are}

\begin{proposition}{}{}
    Any object in $D_\Sg(X)$ is isomorphic to the image of some shifted sheaf.
\end{proposition}

\begin{proof}
    We may take a bounded above projective resolution of an object in $D_\Sg(X)$
    \begin{equation*}
        \begin{tikzcd}
            \cdots \ar[r] & \calP^{a-1} \ar[r] \ar[d] &
            \calP^a \ar[r] \ar[d] & \cdots \ar[r] & \calP^b \ar[d] \\
            \cdots \ar[r] & 0 \ar[r] & \calA^a \ar[r] & \cdots \ar[r] & \calA^b,
        \end{tikzcd}
    \end{equation*}
    and then the truncation
    \begin{equation*}
        \begin{tikzcd}
            \cdots \ar[r] & 0 \ar[r] & \calP^{a-1}/\calP^{a-2} \ar[d] \ar[r] &
            \calP^a \ar[r] \ar[d] & \cdots \ar[r] & \calP^b \ar[d] \\
                & \cdots \ar[r] & 0 \ar[r] &
                \calA^a \ar[r] & \cdots \ar[r] & \calA^b
        \end{tikzcd}
    \end{equation*}
    is also a quasi-isomorphism. This truncated resolution projects to
    $\calP^{a-1}/\calP^{a-2}[1-a]$ with cone quasi-isomorphic to
    $\calP^a\to\cdots\to\calP^b$, so in $D_\Sg(X)$ it is isomorphic to
    $\calP^{a-1}/\calP^{a-2}[1-a]$.
\end{proof}

\begin{example}{}{}
    Consider a non-reduced point $X=\Spec\C[t]/t^n$. By the structure theorem
    for finite type modules over a PID, coherent sheaves on $X$ are direct sums
    of the modules $\C[t]/t^i$ for $i=0,\ldots,n$. Moreover, by taking Smith
    normal forms any complex is a direct sum of shifts of these modules. Only
    $i=n$ gives a projective module, and so $D_\Sg(X)$ is generated by
    $\C[t]/t,\C[t]/t^2,\ldots,\C[t]/t^{n-1}$. Note that these all have infinite
    periodic free resolutions as $\C[t]/t^n$-modules, corresponding to the
    factorizations $t^n=t^it^{n-i}$.
\end{example}

One might hope that $D_\Sg(X)$ only depends on the local geometry of $X$ near
its singular locus, and in fact this is true; $D_\Sg(U)\simeq D_\Sg(X)$ for a
formal neighbourhood $U$ of the singular locus. See
\cite[Prop~1.14]{OrlovSingularities} and \cite[\S6]{Shipman}.

\subsection{Matrix factorizations and Kn\"orrer periodicity}

\subsubsection{Triangulated categories of matrix factorizations}

Consider the cone $Z=\{xy=z^2\}\subset\A^3$. Looking at the structure sheaf of
the line $L=\{x=z=0\}\subset Z$, we are led to a periodic projective resolution
\begin{equation*}
    \cdots \to
    \O_Z^2 \xrightarrow{\begin{pmatrix}
        y & -z \\ -z & x
    \end{pmatrix}}
    \O_Z^2 \xrightarrow{\begin{pmatrix}
        x & z \\ z & y
    \end{pmatrix}}
    \O_Z^2 \xrightarrow{\begin{pmatrix}
        y & -z \\ -z & x
    \end{pmatrix}}
    \O_Z^2 \xrightarrow{\begin{pmatrix}
        x & z
    \end{pmatrix}}
    \O_L\to0,
\end{equation*}
arising from a factorization of $xy-z^2$ into matrices:
\begin{equation*}
    \begin{pmatrix}
        y & -z \\ -z & x
    \end{pmatrix}
    \begin{pmatrix}
        x & z \\ z & y
    \end{pmatrix}
        = (xy-z^2)I =
    \begin{pmatrix}
        x & z \\ z & y
    \end{pmatrix}
    \begin{pmatrix}
        y & -z \\ -z & x
    \end{pmatrix}
\end{equation*}
Viewed over $\A^3$, this is a kind of twisted chain complex, satisfying
$d^2=xy-z^2$ instead of $d^2=0$.

\begin{definition}{}{}
    Suppose $X$ is a smooth quasi-projective variety, and $W:X\to\A^1$ is a
    regular function. A matrix factorization of $W$ on $X$ is a pair of vector
    bundles $\calE^0,\calE^1$ on $X$ with maps
    \begin{equation*}
        \begin{tikzcd}
            \calE^0 \ar[r,"d^0",shift left] & \calE^1 \ar[l,"d^1",shift left]
        \end{tikzcd}
    \end{equation*}
    satisfying $d^1d^0=W\cdot\id_{\calE^0}$, $d^0d^1=W\cdot\id_{\calE^1}$.
\end{definition}

In other words, these are $\Z_2$-graded twisted chain complexes with $d^2=W$.
Given two matrix factorizations $\calE^\bullet,\calF^\bullet$ we obtain a
genuine $\Z_2$-graded complex $\Hom^\bullet(\calE^\bullet,\calF^\bullet)$ with
differential $f\mapsto d\circ f+(-1)^{|f|}f\circ d$ squaring to $W-W=0$. This
gives a differential $\Z_2$-graded category, and as seen earlier we then have a
triangulated homotopy category.

\begin{definition}{}{}
    Taking the 0th cohomology of the complex
    $\Hom^\bullet(\calE^\bullet,\calF^\bullet)$ for morphisms gives a
    triangulated category $\MF(X,W)$ of matrix factorizations.
\end{definition}

\begin{remark}{}{}
    If $W=0$ we recover a weaker version of $\DPerf(X)=D^b(X)$ with only
    $\Z_2$-grading, having a natural forgetful functor $D^b(X)\to\MF(X,0)$.
\end{remark}

\begin{example}{}{}
     Suppose $X=\A^2$, with $W=xy$. The obvious factorization of $W$ gives an
     object
     \begin{equation*}
         K = \left(\begin{tikzcd}
             \O_X \ar[r,"x",shift left] &
             \O_X \ar[l,"y",shift left]
         \end{tikzcd}\right) \in\MF(X,W).
     \end{equation*}
     Note that
     \begin{equation*}
         \Hom^\bullet(K,K) =
         \left(\begin{tikzcd}[ampersand replacement=\&,column sep=huge]
                 \O_X^2 \ar[shift left]{r}{\begin{pmatrix}
                     x & -x \\ -y & y
                 \end{pmatrix}} \&
                 \O_X^2 \ar[shift left]{l}{\begin{pmatrix}
                     y & x \\ y & x
                 \end{pmatrix}}
         \end{tikzcd}\right),
     \end{equation*}
     with cohomology
     \begin{align*}
         H^0\Hom(K,K) &= \{(f,g):f=g\}/\langle(x,x),(y,y)\rangle
             = \C\cdot(1,1), \\
         H^1\Hom(K,K) &= \{(f,g):yf+xg=0\}/\langle(x,-y)\rangle
             = 0,
     \end{align*}
     so $K$ is an exceptional object. In fact $\langle K\rangle^\perp=0$, so
     $\MF(X,W)\simeq D^b(\pt)$ is generated by $K$. % TODO: why?
\end{example}

\begin{remark}{}{}
    Matrix factorizations first arose in commutative algebra, relating to
    maximal Cohen--Macaulay modules \cite{Eisenbud}. In the geometric setting of
    derived categories they are a natural object of study, e.g. via minimal
    resolutions, and motivated by Orlov's theorem. In the context of mirror
    symmetry, (graded) matrix factorizations arise as the conjectural (due to
    Kontsevich) category of B-branes on a Landau--Ginzburg model $(X,W)$,
    generalizing the case $W=0$ which should give $D^b(X)$. See
    \cite{OrlovTheorem}, \cite{Ed}.
\end{remark}

% TODO: example resolve point on cubic curve. point moves, MF moves, but remains
% homotopy equivalent

\subsubsection{Comparing categories of singularities and matrix factorizations}

\subsubsection{Kn\"orrer periodicity}

\begin{equation*}
    \begin{pmatrix}
        M & x \\ -y & N
    \end{pmatrix}\begin{pmatrix}
        N & -x \\ y & M
    \end{pmatrix}
        = W+xy =
    \begin{pmatrix}
        N & -x \\ y & M
    \end{pmatrix}\begin{pmatrix}
        M & x \\ -y & N
    \end{pmatrix}
\end{equation*}

\subsection{Orlov's theorem and Landau--Ginzburg models}

% kuznetzov component example for orlov's theorem
