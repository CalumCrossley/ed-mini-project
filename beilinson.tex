cf. Huybrechts
\begin{proof}
[ADD strong collection definition]

First, we check that $\left< \mathcal{O}, \dots, \mathcal{O} (n) \right>$ is an exceptional collection. 

Consider the morphisms between two elements, where $0\leq j\leq i \leq n$ $$
\begin{align}
\text{Hom}_{D(\mathbb{P}^{n})} (\mathcal{O}(i),\mathcal{O}(j)[l]) &= \text{Ext}^{l(\mathcal{O}(i),\mathcal{O}(j))}= \text{RHom}^{l}(\mathcal{O}(i),\mathcal{O}(j)) \\ 
&= R\Gamma^{l}(\mathbb{P}^{n},\mathcal{O}(j-i))  \\
&= H^{l}(\mathbb{P}^{n},\mathcal{O}(j-i)) \\
&= \begin{cases}
\mathbb{C} :\quad l=0, \,j=i \\
0 : \text{else}
\end{cases}
\end{align}
$$
since $H^{l}(\mathbb{P}^{n}, \mathcal{O}(m))=0$ for any $m<0$ or $l \neq 0$. Hence the collection is strong and exceptional. 

Now, it remains to show that the sequence is full. That is, it generates all of $D(\mathbb{P}^{n})$.  Consider the projection of the product to each component 

IMG

We use the notation $\mathcal{O}(-1)\boxtimes \Omega (1) = q^{*}\mathcal{O}(-1)\otimes p^{*}\Omega (1)$. We have from HAR the Euler sequence $$0 \to \Omega(1)\to \mathcal{O}^{n+1}\to \mathcal{O}(1)$$
Noting that $p^{*}\mathcal{O} \simeq q^{*}\mathcal{O} \simeq \mathcal{O}_{\mathbb{P}\times \mathbb{P}}$, we can form the composition of the exact sequence under $p^*$ and $q^*$ respectively, giving $$p^{*}\Omega (1)\to \mathcal{O}_{\mathbb{P}\times \mathbb{P}}^{n+1} \to q^{*}\mathcal{O}(1)$$
Tensoring with $q^{*}\mathcal{O}(-1)$, and the fact that $q^{*}\mathcal{O}(-1) \otimes q^{*}\mathcal{O}(1) \simeq q^{*\mathcal{O}}\simeq \mathcal{O}_{\mathbb{P}\times \mathbb{P}}$, we get the natural composition of morphisms $$\mathcal{O}(-1)\boxtimes \Omega(1) \to \mathcal{O}_{\mathbb{P}^{n}\times \mathbb{P}^n}$$
The fibre of $\mathcal{O}(-1)$ above a point in $\mathbb{P}^n$ is the point considered as a one-dimensional subspace of $\mathbb{C}^{n+1}$, and the fibre of $\Omega (1)$ at a point $l \in \mathbb{P}^n$ is the space of maps $\phi : \mathbb{C}^{n+1}\to \mathbb{C}$ which are zero on the line $l$. Hence, an element of the fibre of  $\mathcal{O}(-1)\boxtimes \Omega (1)$ at $(l,l') \in \mathbb{P}^{n}\times \mathbb{P}^n$ is $(v,\phi)$, where $v \in l$ and $\phi$ vanishes on $l'$. Looking locally over the point $(l,l')$, we have the evaluation map $$ev: \mathcal{O}(-1)\boxtimes \Omega (1) \to \mathcal{O}_{\mathbb{P}^{n}\times \mathbb{P}^{n}}$$ 
sending $(v,\phi)$ to $\phi(v)$. The image if this map cuts out the same locus as the ideal sheaf of the diagonal, so we can use the Koszul construction to get the locally free resolution for $\mathcal{O}_\Delta$$$
0 \to \bigwedge^{n}\mathcal{O}(-1)\boxtimes\Omega(1) \to \dots\to \mathcal{O}(-1)\boxtimes\Omega(1)\to \mathcal{O}_{\mathbb{P}^{n}\times \mathbb{P}^{n}} \to O_{\Delta}\to 0.
$$
Let $\mathcal{E} :=\mathcal{O}(-1)\boxtimes\Omega(1)$. 

We can split the above resolution into a chain of short exact sequences
$$
\begin{align}
0 \to \bigwedge^{n}\mathcal{E} \to &\bigwedge^{n-1}\mathcal{E}\to M_{n-1}\to 0  \\
0\to M_{n-1}\to & \bigwedge^{n-2}\mathcal{E}\to M_{n-2}\to 0  \\  \\
&\dots \\  \\
0\to M_{1}\to &\mathcal{O}_{\mathbb{P}^{n}\times \mathbb{P}^{n}\to}\mathcal{O}_{\Delta}\to 0
\end{align}
$$

with $M_{k-1}:= \text{Im} (\bigwedge^{k}\mathcal{E} \to \bigwedge^{k-1}\mathcal{E})$.  

Let $\mathcal{F} \in D(X)$. Given that $q^*$, $p^*$ and $\otimes$ are exact, $\Phi_{(-)}  (\mathcal{F}): \mathcal{G} \mapsto =q_{*}(p^{*}\mathcal{F} \otimes \mathcal{G})$ is also exact for $\mathcal{G} \in D(\mathbb{P}^{n}\times \mathbb{P}^n)$. This gives us exact triangles
$$
\begin{align}
 \Phi_{\bigwedge^{n}\mathcal{E}} (\mathcal{F}) \to &\Phi_{\bigwedge^{n-1}\mathcal{E}} (\mathcal{F})\to \Phi_{M_{n-1}}(\mathcal{F})\xrightarrow{[1]}   \\
\Phi_{M_{n-1}}(\mathcal{F})\to & \Phi_{\bigwedge^{n-2}\mathcal{E}} (\mathcal{F})\to \Phi_{M_{n-2}}(\mathcal{F})\xrightarrow{[1]}  \\  \\
&\dots \\  \\
 \Phi_{M_{1}}(\mathcal{F})\to &\Phi_{\mathcal{O}_{\mathbb{P}\times \mathbb{P}} }(\mathcal{F})\to \Phi_{\mathcal{O}_{\Delta}} (\mathcal{F})\xrightarrow{[1]}
\end{align}
$$
We can identify that $\bigwedge^{k}\mathcal{O}(-1)\boxtimes\Omega(1)$ with $\mathcal{O}(-k)\boxtimes\Omega^k(k)$. Then, due to the flatness of $p$ and $q$, and the projection formula, we have for each $i \leq n$, $$\begin{align}
 \Phi_{\bigwedge^{n}\mathcal{E}} (\mathcal{F}) &= q_{*}(p^{*}\mathcal{F }\otimes  (\mathcal{O}(-i)\boxtimes \Omega^i(i))) \\
&= q_{*}(p^{*}\mathcal{F }\otimes  (q^{*}\mathcal{O}(-i)\otimes  p^{*}\Omega^i(i))) \\
&= q_{*}(p^{*}(\mathcal{F}\otimes \Omega^{i}(i)) \otimes  q^{*}\mathcal{O}(-i)) \\
\text{(projection formula)} &=q_{*}p^{*}(\mathcal{F}\otimes \Omega^{i}(i) \otimes  \mathcal{O}(-i)\\
\text{(base change)} &= R\Gamma(\mathbb{P}^{n,}\mathcal{F}\otimes \Omega^{i}(i))\otimes \mathcal{O}(-i) \\
&= H^{*}(\mathbb{P}^{n,}\mathcal{F}\otimes \Omega^{i}(i)) \otimes \mathcal{O}(-i)  
\end{align}$$
Hence,  $\Phi_{\bigwedge^{n}\mathcal{E}} (\mathcal{F})$ lies in $<\mathcal{O}(-i)>$ as a tensor product. By closure under exact triangles, then $\Phi_{M_{n-1}}(\mathcal{F})$ also lies in $<\mathcal{O}(-n), \mathcal{O}(-n+1)>$. By induction, $\Phi_{M_{n-i}}(\mathcal{F})$ is generated by $<\mathcal{O}(-n),\mathcal{O}(-n+1),\dots,\mathcal{O}(-n+i)>$. Hence, $\Phi_{\mathcal{O}_\Delta}(\mathcal{F})$ is generated by $\mathcal{O}(-n),\dots,\mathcal{O}$. But $\Phi_{\mathcal{O}_\Delta}(\mathcal{F}) \simeq \mathcal{ F}$.  Tensoring with $\mathcal{O}(n)$, which is exact, we have the desired result. 


\end{proof}